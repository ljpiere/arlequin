%% Formato para la presentaci\'on de  proyectos de grado. 
%% Maestría en Ingeniería de Software
%% Pontificia Universidad Javeriana
%% Elaborado por Angela Villota y Luisa Rincón
%% Inspirado en Formato elaborado para carrera de Ing de Sistemas
%% V 1.0 Abril - 2022
%\documentclass[openany]{book}

%\documentclass[11pt,twoside]{thesis}
\documentclass[11pt,oneside]{thesis}
\usepackage{enumerate}
\usepackage{longtable}
\usepackage{url}
\usepackage{hyperref}
\usepackage{makeidx}
\usepackage{graphics}
\usepackage{booktabs}
\usepackage[spanish,es-nodecimaldot,es-noshorthands]{babel}
\usepackage{tikz}
\usetikzlibrary{arrows.meta,positioning,shapes.geometric,shapes.misc,fit,backgrounds,calc}
\usepackage{float} % para [H]
\usepackage{adjustbox}
% Paleta sobria para impresión (grises + acento)
\definecolor{lanegray}{RGB}{245,245,245}
\definecolor{accent}{RGB}{23,111,176}   % azul sobrio
\definecolor{dark}{RGB}{60,60,60}
%%%%%%%%%%%%%%%%%%%
% Estilos reutilizables
\tikzset{
  >={Latex[length=3mm,width=2mm]},
  font=\small,
  node distance=9mm,
  block/.style={rectangle, rounded corners=3pt, minimum width=44mm, minimum height=10mm,
                draw=dark, very thick, fill=white, align=center, inner sep=3.5pt},
  emph/.style={block, draw=accent, very thick},
  note/.style={rectangle, rounded corners=2pt, draw=dark, fill=white, inner sep=2.5pt, align=left},
  lane/.style={rounded corners=5pt, draw=dark, line width=0.4pt, fill=lanegray},
  dot/.style={circle, inner sep=1pt, fill=dark, draw=dark},
  flow/.style={->, very thick, draw=dark},
  flowaccent/.style={->, very thick, draw=accent},
  dashedflow/.style={->, thick, draw=dark, dashed},
}
%%%%%%%%%%%%%%%%%%%
%\usepackage{tikz}
\usetikzlibrary{positioning,fit,arrows.meta,shapes.symbols,shapes.geometric}
\usetikzlibrary{arrows.meta,positioning} 
\usepackage{adjustbox}
\usepackage{caption}
\usepackage{natbib}
% Paquetes adicionales
\usepackage{amsmath}
\usepackage[spanish,es-tabla]{babel}
\usepackage{siunitx}
\usepackage{pgfplots}
\pgfplotsset{compat=1.18}
\usepgfplotslibrary{statistics} % para boxplots
\usetikzlibrary{shapes.geometric, arrows.meta, positioning}
\usepackage{minted}
\usetikzlibrary{positioning,calc}
\usepackage{pgfplots}
\pgfplotsset{compat=1.18}
\usepackage{subcaption}
\tikzstyle{block} = [rectangle, draw=black, fill=white!10, minimum height=1.2cm, text centered, rounded corners, minimum width=3.5cm]
\tikzstyle{arrow} = [->, thick, >=stealth]
\usepackage{lipsum}
\usepackage{url} 
% --- TikZ estilos compactos ---
\tikzset{
  box/.style={draw, rounded corners=2pt, fill=gray!8, align=center,
              minimum width=18mm, minimum height=7mm, text width=28mm,
              inner sep=2pt},
  proc/.style={box},
  obs/.style={box, fill=blue!5},
  db/.style={cylinder, draw, shape border rotate=90, aspect=0.25,
             minimum height=9mm, minimum width=9mm,
             cylinder uses custom fill, cylinder end fill=gray!20,
             cylinder body fill=gray!8, align=center, text width=24mm},
  decision/.style={diamond, draw, aspect=2, align=center, inner sep=1pt,
                   fill=gray!6, text width=18mm},
  arrow/.style={-Latex, line width=0.45pt},
  note/.style={font=\scriptsize, inner sep=1pt, fill=white, align=center}
}

\usepackage{listings}
\usepackage{xcolor}
\lstdefinestyle{airflow}{
  language=Python,
  basicstyle=\ttfamily\footnotesize,
  numbers=left,
  numbersep=6pt,
  frame=single,
  rulecolor=\color{black!20},
  breaklines=true,
  breakatwhitespace=true,
  columns=fullflexible,
  keepspaces=true,
  showstringspaces=false,
  tabsize=2,
  captionpos=b
}

% --- Figuras con mejor control de ubicación ---
\usepackage{float} % para [H]

% --- TikZ para el diagrama simple ---
\usepackage{tikz}
\usetikzlibrary{arrows.meta,positioning,fit}


%%%%%%%%%%%%%%%%
% TITULO DEL PROYECTO
%%%%%%%%%%%%%%%%
\titulo{Implementación de un Sistema Automatizado de Reentrenamiento mediante la Detección de Data Drift en un Ciclo MLOps para Entornos Big Data}

%%%%%%%
% AUTORES
%%%%%%%
\autorA{1234640402}{Leyton Jean Piere Castro Clavijo}

%%%%%%%
% FECHA DE ENTREGA
%%%%%%%
\fecha{xx de xx de xx}

\directorproyecto{}{MSc Felipe Buitrago Carmona}
\directorcarrera{Ph.D.}{Luisa Rincón}

\include{format}
\makeindex

\begin{document}

\maketitle

%%%%%%%%%%%%%%%%%%
% PARA LAS CARTAS
%%%%%%%%%%%%%%%%%%
\makecoverletter


%%%%%%%%%%%%%%%%%%
% FICHA RESUMEN
%%%%%%%%%%%%%%%%%%
\newpage{\ } 
\thispagestyle{empty}
\begin{center}
    \Large{Ficha Resumen \\ Trabajo de Grado Maestría en Ingeniería de Software}
\end{center}

\textbf{TíTULO:} {Implementación de un Sistema Automatizado de Reentrenamiento mediante la Detección de Data Drift en un Ciclo MLOps para Entornos Big Data}
\begin{enumerate}
    \item Énfasis: Ingeniería de Software
    \item Área de trabajo: Ciencias de la Computación y la Ingeniería de Sistemas, con un enfoque específico en Big Data, Machine Learning y MLOps 
    \item Tipo de proyecto: Aplicado
    \item Estudiante: Leyton Jean Piere Castro Clavijo
    \item Correo electrónico: leytoncastro@javerianacali.edu.co
    \item Dirección y teléfono: Cr. 9A \#43-38 Ibagué \- Tolima, 318 785 7029
    \item Director: Felipe Buitrago Carmona
    \item Vinculación del director: Externo
    \item Correo electrónico del director: felipe.buitrago@ucaldas.edu.co
    \item Co-Director (Si aplica):
    \item Grupo o empresa que lo avala (Si aplica):
    \item Otros grupos o empresas: Ninguno
    \item Palabras clave: Data Drift, MLOps, Big Data, Automatización, Machine Learning
    \item ODS que aplica al proyecto (Agenda 2030): ODS 8, ODS 9, ODS 12.
    \item Fecha de inicio: 23 de Junio de 2025
    \item Resumen:  El presente proyecto propone un sistema automatizado de reentrenamiento de modelos de \textit{machine learning} que responde a la detección temprana del \textit{data drift} en entornos Big Data. En la actualidad, las organizaciones enfrentan pérdidas promedio de 12,9 millones de USD anuales por mala calidad de datos y 4,88 millones por brechas de información \citep{Gartner2024,IBM2024}, lo que evidencia la necesidad de soluciones que garanticen precisión y resiliencia en sistemas basados en datos. El sistema desarrollado monitorea continuamente los flujos de entrada de los modelos en producción; cuando identifica cambios estadísticamente significativos en su distribución, activa un \textit{pipeline} automatizado que reentrena, valida y despliega una nueva versión del modelo. De esta forma, se obtiene un modelo actualizado, trazable y auditable, que mantiene la exactitud predictiva ante condiciones cambiantes. Basado en prácticas MLOps, el ciclo integra herramientas abiertas como Docker para contenerización, Jenkins para orquestación, Spark para procesamiento distribuido y MLflow para trazabilidad experimental. Los resultados, validados en escenarios con y sin \textit{drift}, demuestran una recuperación completa del desempeño del modelo (F1-score) y una reducción significativa del tiempo de respuesta operativa. El proyecto contribuye a los \textbf{ODS 8, 9 y 12}: fomenta la eficiencia y productividad mediante automatización inteligente (ODS 8), impulsa innovación e infraestructura digital reproducible (ODS 9) y promueve la sostenibilidad operativa al reducir retrabajos y desperdicio computacional (ODS 12).
\end{enumerate}


%%%%%%%%%%%%%%%%%%
% AGRADECIMIENTOS
%%%%%%%%%%%%%%%%%%
\agradecimientos  % No borrar esta línea pues define el formato. OJO el texto debe estar entre llaves
{%NO BORRAR
Escribe aqui los agradecimientos, aquí hay algunas pautas que podrían ser útiles:
\textbf{Comenzar con una nota de agradecimiento general}: Empieza con una declaración general de agradecimiento a todas las personas que te han apoyado a lo largo del proceso.

\textbf{Agradecimientos específicos:}
Familia y amigos: A menudo, las primeras personas que se mencionan son los miembros de la familia y amigos cercanos que han proporcionado apoyo emocional, asesoramientoo.

Asesores y supervisores: Es esencial agradecer a tu asesor de tesis o cualquier otro mentor académico. 
Comité de tesis o jurado: Si aplicable, agradece a los miembros de tu comité de tesis o jurado por su tiempo y atención.

Colegas y compañeros de clase: Menciona a cualquier compañero estudiante o colega que haya contribuido de alguna manera a tu investigación o pensamiento.

Organizaciones de financiación: Si tu investigación fue financiada por alguna beca o subsidio, asegúrate de reconocer la contribución de la organización.

Profesionales y expertos en la materia: Si has tenido ayuda de expertos o profesionales (entrevistas, consultas), menciona su ayuda e influencia en tu trabajo.


Sé sincero y personal, pero mantén la profesionalidad y el enfoque académico de la sección.
Concluye con una declaración que refleje tu experiencia personal a lo largo de tu viaje académico y cómo estos agradecimientos resuenan contigo

} %NO BORRAR


%%%%%%%%%%%%%%%%%%
% ABSTRACT
%%%%%%%%%%%%%%%%%%
%%%%%%%%%%%%%%%%
% ABSTRACT
%%%%%%%%%%%%%%%%

\chapter*{Resumen}
El presente proyecto propone un sistema automatizado de reentrenamiento de modelos de \textit{machine learning} que responde a la detección temprana del \textit{data drift} en entornos Big Data. En la actualidad, las organizaciones enfrentan pérdidas promedio de 12,9 millones de USD anuales por mala calidad de datos y 4,88 millones por brechas de información \citep{Gartner2024,IBM2024}, lo que evidencia la necesidad de soluciones que garanticen precisión y resiliencia en sistemas basados en datos.

El sistema desarrollado monitorea continuamente los flujos de entrada de los modelos en producción; cuando identifica cambios estadísticamente significativos en su distribución, activa un \textit{pipeline} automatizado que reentrena, valida y despliega una nueva versión del modelo. De esta forma, se obtiene un modelo actualizado, trazable y auditable, que mantiene la exactitud predictiva ante condiciones cambiantes.

Basado en prácticas MLOps, el ciclo integra herramientas abiertas como Docker para contenerización, Jenkins para orquestación, Spark para procesamiento distribuido y MLflow para trazabilidad experimental. Los resultados, validados en escenarios con y sin \textit{drift}, demuestran una recuperación completa del desempeño del modelo (F1-score) y una reducción significativa del tiempo de respuesta operativa.

El proyecto contribuye a los \textbf{ODS 8, 9 y 12}: fomenta la eficiencia y productividad mediante automatización inteligente (ODS 8), impulsa innovación e infraestructura digital reproducible (ODS 9) y promueve la sostenibilidad operativa al reducir retrabajos y desperdicio computacional (ODS 12).

\textbf{Palabras Clave}: Data Drift, MLOps, Big Data, Automatización, Machine Learning

\chapter*{Abstract}
This research presents the \textbf{design, implementation, and empirical validation} of a fully automated retraining system for \textbf{machine-learning models} based on early \textbf{data-drift detection} in \textbf{Big Data} environments. In modern organizations, poor data quality costs an average of \textbf{USD 12.9 million} per year, while data breaches exceed \textbf{USD 4.88 million} in losses \citep{Gartner2024,IBM2024}, highlighting the critical need for resilient and self-adaptive analytical systems.

The proposed system continuously monitors production data streams; when it detects statistically significant distributional changes, it triggers an automated \textit{pipeline} that retrains, validates, and redeploys the model. As a result, the model remains accurate, traceable, and auditable under changing data conditions. 

Grounded in \textbf{MLOps best practices}, the architecture integrates open-source tools such as \textbf{Docker} for containerization, \textbf{Jenkins} for orchestration, \textbf{Apache Spark} for distributed processing, and \textbf{MLflow} for experiment tracking. Empirical validation under both drift and non-drift scenarios demonstrated full recovery of the model’s F1-score and a substantial reduction in operational latency and manual intervention.

The project contributes to the \textbf{UN Sustainable Development Goals (SDGs 8, 9, and 12)} by enhancing operational efficiency and productivity through intelligent automation (SDG 8), strengthening innovation and resilient digital infrastructure (SDG 9), and promoting sustainable resource use and reduced computational waste (SDG 12).

\textbf{Keywords}: Data Drift; MLOps; Big Data; Automated Retraining; Machine Learning


%%%%%%%%%%%%%%%%%
%%%% TABLE OF CONTENTS
%%%% AND PREABLE
%%%%%%%%%%%%%%%%%
  
\newpage{\ } 
\thispagestyle{empty} 
\tableofcontents 
\listoffigures   \listoftables   \mainmatter

\dominitoc
\cleardoublepage

%%%%%%%%%%%%%%%%%%%%%%
% INTRODUCCION
%%%%%%%%%%%%%%%%%%%%%%
\chapter{Introducción}
%%%%%%%%%%%%%%%%%%%%
% INTRODUCCION
%%%%%%%%%%%%%%%%%%%%
% \section{Introducción}

El crecimiento exponencial en la generación de datos ha consolidado al aprendizaje automático como herramienta clave para la toma de decisiones. No obstante, los modelos no son estáticos: su rendimiento se degrada cuando cambia la distribución de entrada, fenómeno conocido como \textit{data drift} \citep{Sculley2015}, lo que compromete la confiabilidad de sistemas inteligentes y eleva la deuda técnica.

A pesar de los avances en MLOps, persiste una brecha entre la detección estadística del \textit{drift} y la capacidad de reaccionar automáticamente con trazabilidad completa. Muchos despliegues dependen de alertas manuales, pipelines poco reproducibles o herramientas propietarias difíciles de auditar, lo que limita la adopción de prácticas operativas consistentes en contextos latinoamericanos y de código abierto.

Esta investigación cierra dicha brecha mediante la implementación de un sistema integrado que combina monitoreo continuo, pruebas estadísticas (KS, $\chi^2$, PSI), orquestación CI/CD y registro de artefactos en un ciclo MLOps reproducible. El pipeline —basado en Jenkins, MLflow, Spark y Prometheus— observa los datos en tiempo real, detecta desviaciones y gatilla automáticamente el reentrenamiento y despliegue del modelo, manteniendo trazabilidad y control de versiones.

La solución se valida en un entorno simulado que replica condiciones operativas (escenarios con y sin \textit{drift}), donde los flujos de entrada generan evidencias cuantitativas sobre detección, latencia y recuperación del desempeño. Este cierre metodológico demuestra la factibilidad de un enfoque abierto que reduce la intervención manual y aporta lineamientos técnicos para operar modelos adaptativos en producción.

\section{Definición del problema}
\subsection{Planteamiento del problema}
Para evitar redundancia, se sintetizan aqu\'i datos clave; el detalle ampliado se presenta en el Anexo \ref{ann:contexto}.
En el contexto global actual, caracterizado por una explosión de datos sin precedentes, se estima que el \emph{Global DataSphere} alcanzará los 181 Zettabytes en 2025 \citep{IDC2024}. Esta expansión está impulsada por el crecimiento acelerado de tecnologías como la inteligencia artificial generativa, el Internet de las Cosas (IoT), y la digitalización masiva de procesos. Sin embargo, la utilidad real de estos datos se ve comprometida por problemas de calidad y gobernanza. Según Gartner, la mala calidad de los datos le cuesta a las organizaciones, en promedio, 12,9 millones de dólares anuales \citep{Gartner2024}, mientras que el costo promedio por brecha de datos se sitúa en 4,88 millones de dólares \citep{IBM2024}.

En este escenario, el uso de modelos de machine learning se ha generalizado como herramienta para la toma de decisiones automatizadas. No obstante, estudios recientes revelan que cerca del 91\% de los modelos desplegados en producción sufren algún tipo de \emph{model drift} en su primer año de operación \citep{AIMultiple2025,Breck2019}. Este fenómeno, conocido como \emph{data drift} cuando afecta la distribución de entrada de los datos, degrada el rendimiento predictivo, comprometiendo decisiones operativas, regulatorias y comerciales.

Para mantener la fluidez narrativa sin sacrificar evidencia, los principales indicadores cuantitativos se sintetizan en la Tabla~\ref{tab:contexto-mlops}, mientras que los detalles y series temporales se documentan en el Anexo~\ref{ann:contexto}. La tabla resume tanto el crecimiento global de los datos como las brechas regionales de madurez y calidad, proporcionando el punto de partida para el problema técnico abordado en esta tesis.

\begin{table}[H]
\centering
\caption{Indicadores contextuales sobre data drift y madurez MLOps.}
\label{tab:contexto-mlops}
\begin{tabular}{p{5cm}p{8cm}}
\toprule
\textbf{Indicador} & \textbf{Valor / Fuente} \\
\midrule
Explosión de datos globales & \emph{Global DataSphere} proyectado en 181 ZB para 2025 \citep{IDC2024}. \\
Costo de mala calidad de datos & Promedio global de USD 12.9 M por organización al año \citep{Gartner2024}. \\
Incidencia de \emph{model drift} & 91\,\% de los modelos sufren degradación en el primer año \citep{AIMultiple2025,Breck2019}. \\
Inversión en centros de datos LATAM & USD 6.36 B (2023) con CAGR 7.95\,\% hasta 2029 \citep{Helmi2024}. \\
Calidad de datos en Colombia & 56\,\% de bases corporativas con deficiencias críticas \citep{Deyde2023}. \\
Inversión pública en IA & COP 480 mil millones para infraestructura/capacitación (Política IA 2025) \citep{CONPES2025}. \\
Impacto económico del drift & Pérdida > USD 12 M por caída de 3\,pp en precisión de modelos crediticios \citep{Kim2018}. \\
\bottomrule
\end{tabular}
\end{table}

A nivel regional, la inversión tecnológica no se ha traducido en la madurez operativa necesaria para sostener modelos en producción; persisten retos de calidad del dato, cultura organizacional y adopción de prácticas MLOps \citep{Helmi2024,Deyde2023}. Incluso iniciativas públicas como la Política Nacional de IA (2025) carecen de lineamientos concretos para pipelines automatizados que mitiguen el \emph{drift} \citep{CONPES2025}. Desde el punto de vista económico, la ausencia de mecanismos automatizados se refleja en pérdidas millonarias y en decisiones subóptimas en sectores regulados \citep{Kim2018}. Estas brechas justifican la necesidad de soluciones reproducibles que integren detección, reentrenamiento y trazabilidad.

En el plano social y de desarrollo sostenible, esta problemática impacta la eficiencia de los servicios digitales, la equidad tecnológica y la sostenibilidad organizacional. Su vínculo con los Objetivos de Desarrollo Sostenible (ODS) es directo:

\begin{itemize}
  \item \textbf{ODS 8 – Trabajo decente y crecimiento económico:} La degradación en el rendimiento de los modelos compromete la productividad de procesos automatizados, disminuye la competitividad y obstaculiza la generación de empleo calificado en áreas tecnológicas. Un sistema automatizado de actualización de modelos puede contribuir a mantener la eficiencia operativa y abrir nuevas oportunidades de innovación.

  \item \textbf{ODS 9 – Industria, innovación e infraestructura:} El uso ineficiente de modelos obsoletos limita la innovación tecnológica y reduce la efectividad de los procesos industriales basados en datos. La automatización del reentrenamiento fortalece la infraestructura analítica y promueve una cultura de mejora continua.

  \item \textbf{ODS 12 – Producción y consumo responsables:} Decisiones mal informadas debido a modelos desactualizados pueden conllevar a sobrecostos, desperdicio de recursos y fallas logísticas. Automatizar la actualización de modelos con base en datos actuales permite una asignación más eficiente y responsable de los recursos.
\end{itemize}

Desde el punto de vista técnico, las soluciones actuales presentan limitaciones significativas. La mayoría de las organizaciones aún dependen de monitoreo reactivo en paneles aislados, sin mecanismos de detección y respuesta automatizados. Los pipelines de reentrenamiento suelen ser ad-hoc, sin adherencia a prácticas de integración y despliegue continuo (CI/CD), y carecen de capacidades para orquestación multinube, contenerización uniforme y cumplimiento regulatorio.

En síntesis, aunque la literatura ha documentado múltiples enfoques de detección de data drift y mecanismos de reentrenamiento, persisten vacíos en torno a su integración práctica en pipelines completamente automatizados y reproducibles bajo esquemas CI/CD y observabilidad de extremo a extremo. La mayoría de propuestas carece de validación empírica en entornos Big Data o de estrategias abiertas que permitan su réplica. Este proyecto busca aportar una contribución metodológica concreta al desarrollar y documentar una arquitectura open-source reproducible que combine detección estadística de drift, automatización CI/CD y monitoreo operacional completo, constituyendo un avance práctico frente a ese rezago de la literatura.

% =========================
% 2.2 Formulación del problema
% =========================222
\subsection{Formulación del problema}
\paragraph{Pregunta central.}
\textit{¿Cómo diseñar, desarrollar e implementar un sistema productivo robusto basado en un pipeline MLOps, que detecte automáticamente \emph{data drift}, active reentrenamiento y despliegue continuo, minimizando intervención humana, costos operativos y garantizando alta disponibilidad en entornos reales Big Data?}

\paragraph{Enunciado de validación del enfoque propuesto.}
\begin{quotation}
\noindent Se plantea que un sistema productivo basado en MLOps, con capacidades de detección automática de \emph{data drift}, monitoreo continuo y reentrenamiento automatizado, puede mejorar sustancialmente la estabilidad operativa y la precisión de los modelos de \emph{machine learning} al reducir la intervención manual y facilitar la actualización continua en entornos de datos dinámicos.
\end{quotation}

\paragraph{Aspectos clave a demostrar durante el desarrollo del proyecto.}
% \begin{enumerate}[label=\textbf{A\arabic*}]
\begin{enumerate}
  \item Que los mecanismos automáticos de alerta y análisis estadístico permiten identificar desviaciones en la distribución de datos que afecten el rendimiento del modelo.
  \item Que la automatización del reentrenamiento reduce la necesidad de intervención humana y mejora la continuidad operativa del sistema.
  \item Que la trazabilidad y gestión del ciclo de vida de los modelos facilita su mantenimiento, validación y auditoría en producción.
  \item Que una arquitectura integrada y replicable permite desplegar flujos MLOps en contextos reales que requieren adaptabilidad frente al cambio de datos.
\end{enumerate}


\section{Objetivos del proyecto}
\subsection{Objetivo General}
Construir un sistema automatizado de reentrenamiento que, mediante la detección temprana de \emph{data drift}, active un ciclo MLOps en entornos Big Data, asegurando la actualización continua y el reentrenamiento de los modelos de machine learning.


\subsection{Objetivos específicos}
\begin{itemize}
    \item Implementar una infraestructura en la nube escalable que permita el procesamiento eficiente de grandes volúmenes de datos, integrando mecanismos de monitorización y evaluación continua para identificar y responder oportunamente al \emph{data drift}.
    
    \item Desarrollar un sistema computacional que combine métodos estadísticos y algoritmos supervisados para la detección automática de \emph{data drift} en tiempo real, generando alertas que activen procesos automatizados de reentrenamiento sin intervención humana.
    
    \item Validar en un escenario simulado con y sin \emph{data drift}, la eficacia del sistema computacional tras la detección de desviaciones, midiendo precisión, recall y latencia para asegurar la actualización continua y la eficiencia operativa.
\end{itemize} 

\section{Delimitaciones y alcances}
El proyecto se concentró en la evaluación de un prototipo funcional basado en una arquitectura de Big Data previamente documentada en la literatura, evitando el diseño de una nueva infraestructura desde cero. Se utilizará como referencia una configuración de clúster pseudo-distribuido que emplea contenedores Docker y Docker Compose, replicando un sistema similar a HDFS y herramientas de procesamiento de datos como Apache Spark en un entorno controlado. Este montaje es, por tanto, un \textbf{entorno Big Data simulado} que permite observar el comportamiento del pipeline con control experimental absoluto; aunque no se ejecuta sobre un clúster físico multi-nodo, la arquitectura (Spark, HDFS, Jenkins, MLflow) está declarada y parametrizada para escalar en entornos distribuidos reales con mínimos ajustes de infraestructura (p.\,ej., despliegues en Kubernetes o servicios administrados).

Además, se adoptó la arquitectura propuesta por \citep{Chen2022}, la cual integra un sistema de almacenamiento distribuido con capacidades de procesamiento en tiempo real y batch, facilitando el manejo de grandes volúmenes de información. Esta arquitectura se complementa con sistemas de monitorización y alertas automáticas que permiten la detección de \textit{data drift} y, consecuentemente, la activación de \textit{pipelines} automáticos de reentrenamiento.

Para la validación del sistema, se desarrolló dos escenarios de prueba mediante la generación de datos semi-aleatorios utilizando la biblioteca \texttt{Faker} de Python. El primer escenario servirá como base para el entrenamiento inicial del modelo de \textit{machine learning}, mientras que en el segundo escenario se introducirá una variación controlada en la distribución de los datos (\textit{data drift} intencional) para demostrar el funcionamiento del \textit{pipeline} de reentrenamiento automático. Se implementarán mecanismos de detección de \textit{data drift} mediante pruebas estadísticas como Kolmogorov-Smirnov, Chi-cuadrado y la divergencia de Kullback-Leibler, permitiendo evaluar el impacto en las métricas del modelo. La implementación del \textit{pipeline} automatizado mediante Jenkins se concentrará en un modelo de \textit{machine learning} de prueba, permitiendo evaluar la eficacia del sistema sin extender el análisis a múltiples modelos en paralelo. La integración de MLflow facilitará el registro y comparación de cada ciclo de reentrenamiento, mientras que \textit{dashboards} en Grafana ofrecerán una visualización en tiempo real de métricas como precisión, \textit{recall}, F1-score y tiempos de respuesta del sistema.

Finalmente, aunque el prototipo se implementa en un entorno local con Docker y Docker Compose, la arquitectura propuesta es \textbf{portátil hacia la nube}, pudiendo migrarse de manera sencilla a plataformas como \textbf{Azure Machine Learning}. Esto permitiría aprovechar clústeres administrados de cómputo, escalado automático de recursos y la integración con \textbf{Azure Monitor} para fortalecer las capacidades de observabilidad y operación continua del sistema.

\textbf{Tipo de \textit{drift} evaluado.} La validación experimental se enfoca principalmente en \textbf{covariate drift} (cambios en la distribución de las variables de entrada) inducido de forma controlada en las variables financieras y de riesgo; se incorpora una representación limitada del \textbf{concept drift} únicamente a través de la variable \texttt{risk\_score}. Las conclusiones deben interpretarse bajo esta delimitación: el pipeline muestra sensibilidad y recuperación frente a \textit{covariate drift} y sienta las bases para extender la estrategia a escenarios de \textit{concept drift} reforzado.

\section{Justificación del trabajo de grado}
La creciente dependencia de modelos de \emph{machine learning} para la toma de decisiones estratégicas en entornos productivos obliga a mantener altos niveles de precisión y rendimiento. Sin embargo, cambios continuos en la distribución de los datos, conocidos como \emph{data drift}, generan degradaciones críticas en la precisión de los modelos, incrementando costos operativos y riesgos regulatorios cuando no son detectados y gestionados oportunamente.

Este proyecto justifica su relevancia al proponer un producto tecnológico innovador que automatiza la detección temprana y el reentrenamiento continuo de modelos en producción. Al integrar herramientas avanzadas de monitorización y técnicas estadísticas robustas con pipelines automatizados mediante metodologías MLOps, la solución reduce significativamente la intervención humana, minimiza errores operativos y mejora la rapidez en la respuesta ante desviaciones.

Adicionalmente, la propuesta establece un estándar metodológico replicable, facilitando la adopción de prácticas ágiles y seguras en el ciclo completo de vida de los modelos. De esta forma, el sistema propuesto no solo asegura una alta precisión predictiva y estabilidad operativa, sino que también impulsa la innovación y competitividad de las organizaciones que gestionan grandes volúmenes de datos dinámicos \citep{Kim2018, Merkel2014, Breck2019, Gama2014}.

\section{Metodología de la investigación}
\label{sec:metodologia}

% Vision general (evitar duplicidad con Caps. 3 y 4)
La metodolog\'ia se presenta aqu\'i de forma sint\'etica para evitar duplicidad con los Cap\'itulos 3 y 4. El estudio es \textbf{aplicado y cuasi-experimental}: valida un pipeline MLOps que \textit{detecta} \textit{data drift} y \textit{reentrena} autom\'aticamente.

\textbf{Disen\~no general.} Dos escenarios: E1 (sin \textit{drift}) y E2 (con \textit{drift} inducido). Se observan m\'etricas de desempe\~no (F1, Recall, AUC), latencias operativas (\textit{TTFD}, \textit{TTR}) y magnitud de cambio (\textit{PSI}). Las pol\'iticas de activaci\'on/cooldown, el detalle de hip\'otesis, variables y an\'alisis estad\'istico se desarrollan en el Cap\'itulo~\ref{sec:eval-design} y \S\ref{subsec:eval-scenarios}.

\textbf{Implementaci\'on.} La infraestructura y componentes t\'ecnicos (Spark/HDFS, Jenkins, MLflow, Prometheus/Grafana, Docker~Compose) se documentan en el Cap\'itulo~\ref{sec:oe1}. Aqu\'i solo se describe la l\'ogica metodol\'ogica y el alcance experimental; los detalles operativos quedan en los cap\'itulos respectivos.

% --- Detalle t\'ecnico comentado para evitar repetici\'on literal ---
\iffalse

\subsection{Enfoque metodológico y diseño experimental}

La investigación adopta un \textbf{enfoque aplicado y cuantitativo} con diseño cuasi-experimental, orientado a demostrar la eficacia de un sistema automatizado de reentrenamiento de modelos basado en la detección temprana de \textit{data drift}. El propósito es evaluar empíricamente la relación entre la ocurrencia de \textit{drift} en los datos de entrada y el desempeño de un modelo de \textit{machine learning} bajo condiciones controladas.

\textbf{Hipótesis de trabajo.}  
H\textsubscript{0}: La activación automática del reentrenamiento no mejora significativamente las métricas de desempeño del modelo tras la detección de \textit{data drift}.  
H\textsubscript{1}: Un sistema automatizado de reentrenamiento basado en \textit{MLOps} mejora significativamente el desempeño del modelo y reduce la latencia de detección y respuesta frente al \textit{drift}.

\textbf{Variables.}
\begin{itemize}
    \item \textbf{Variable independiente:} Presencia o ausencia de \textit{data drift} (escenarios con y sin desviación inducida en los datos).
    \item \textbf{Variables dependientes:} 
    \begin{enumerate}
        \item \textit{F1-score} y \textit{Recall} — desempeño predictivo del modelo.
        \item \textit{TTFD} (Time-to-First-Detection) — tiempo entre el inicio del \textit{drift} y su detección estadística.
        \item \textit{TTR} (Time-to-Retrain) — tiempo total desde la detección hasta la finalización del reentrenamiento.
        \item \textit{PSI} (Population Stability Index) — magnitud de la desviación de distribución.
    \end{enumerate}
    \item \textbf{Variables de control:} tamaño de muestra por ventana, número de iteraciones, umbral $\alpha$ de significancia y parámetros del modelo base.
\end{itemize}

\textbf{Diseño experimental.}  
El sistema se evaluará en dos condiciones:
\begin{enumerate}
    \item Escenario E1: flujo de datos estable sin \textit{drift} (línea base).  
    \item Escenario E2: flujo con \textit{drift} inducido mediante alteración de las variables de monto y riesgo.
\end{enumerate}
En cada condición se registran las métricas anteriores a lo largo de múltiples ciclos de ejecución para estimar promedios e intervalos de confianza al 95 \%.

\textbf{Métodos de análisis.}  
Dado que las métricas pueden no seguir una distribución normal, se emplean pruebas \textbf{no paramétricas} de comparación de medianas y distribuciones (Kolmogorov–Smirnov, $\chi^2$ y PSI). Estas pruebas permiten detectar cambios significativos sin requerir supuestos fuertes sobre la forma de las distribuciones, lo que las hace adecuadas para flujos de datos heterogéneos en entornos Big Data. Los resultados se analizarán mediante estadísticos descriptivos (media, desviación estándar, IC95 %) y gráficas temporales que evidencien la recuperación del desempeño tras el reentrenamiento.

\textbf{Criterios de evaluación.}
\begin{itemize}
    \item F1 ≥ 0.80 y mejora ≥ 10 \% tras el reentrenamiento.
    \item TTFD < 5 min y TTR < 15 min por ciclo.
    \item PSI ≤ 0.2 como umbral de alerta y ≥ 0.3 como desviación severa.
\end{itemize}

El enfoque metodológico propuesto combina la rigurosidad experimental con la trazabilidad operativa propia de las prácticas MLOps, permitiendo validar cuantitativamente la hipótesis y los objetivos planteados.
\paragraph{Resultados en detalle.}
Los resultados cuantitativos y la discusión estadística completa se presentan en los Capítulos~\ref{sec:eval-results} y \ref{sec:conclusiones}. En esta introducción se ofrecen únicamente los objetivos, el contexto y la propuesta técnica; la evidencia empírica se analiza una vez descritos el diseño experimental y la instrumentación que la soportan.

\subsection{Fases de desarrollo e implementación}

La investigación adoptó una \textbf{metodología aplicada y experimental}, organizada en fases secuenciales que abarcan desde el diseño conceptual hasta la validación empírica del sistema. El propósito fue construir y evaluar un \textit{pipeline} de reentrenamiento automatizado basado en detección temprana de \textit{data drift}.

Cada fase combinó \textbf{principios de ingeniería reproducible} —contenedorización, trazabilidad y modularidad— con \textbf{criterios de validación cuantitativa} orientados a métricas de desempeño y eficiencia operativa. El enfoque técnico se apoyó en herramientas de código abierto (Python, Spark, Jenkins, MLflow y Prometheus), desplegadas mediante \textbf{Docker Compose} para garantizar replicabilidad y control experimental.

\noindent
La Figura~\ref{fig:flujo_trabajo_estilizado_fit} sintetiza el flujo metodológico del proyecto, integrando las etapas de datos, automatización y observabilidad dentro de un ciclo continuo de mejora. Este diagrama no representa la implementación detallada, sino la \textbf{lógica de investigación} que guía la transición desde el diseño hasta la validación del sistema.



% necesita en el preámbulo
% \usepackage{tikz}
% \usetikzlibrary{arrows.meta,positioning,shapes.geometric,fit,backgrounds,calc}
% \usepackage{float}

\begingroup
\shorthandoff{>}
\begin{figure}[H]
\centering

% --- estilos locales ---
\definecolor{lanegray}{RGB}{248,248,248}
\definecolor{dark}{RGB}{60,60,60} % ya no usamos accent
\tikzset{
  >={Latex[length=3mm,width=2mm]},
  node distance=9mm,
  block/.style={rectangle, rounded corners=3pt, minimum width=46mm, minimum height=10mm,
                draw=dark, very thick, fill=white, align=center, inner sep=3.5pt},
  emph/.style={block, draw=dark, very thick},  % antes usaba accent
  lane/.style={rounded corners=5pt, draw=black!12, line width=0.3pt, fill=lanegray, inner sep=6mm},
  flow/.style={->, very thick, draw=dark},
  flowaccent/.style={->, very thick, draw=dark}, % antes era accent
  dashedflow/.style={->, thick, draw=dark, dashed},
}

\begin{tikzpicture}[x=0.8mm, y=0.8mm]

% ====== BLOQUES (contenido) ======
\node[block] (ingesta)    at (15,112) {\shortstack{Lectura y\\preparación de datos}};
\node[block, below=12mm of ingesta] (drift) {\shortstack{Detección de\\ \textit{data drift}\\(KS, $\chi^2$, KL / PSI)}};
\node[emph, below=12mm of drift, xshift=4mm] (alerta) {\shortstack{Alerta por\\desviación\\(umbral superado)}};

\node[block] (decision)   at (82,100) {\shortstack{Mecanismo de\\decisión\\(gatillar / no gatillar)}};
\node[emph, below=14mm of decision] (retrain) {\shortstack{Reentrenamiento\\automático}};
\node[block, below=12mm of retrain] (mlflow) {\shortstack{Registro\\y versionado}};

\node[block] (dash)       at (148,92) {\shortstack{Visualización\\de métricas\\(Grafana / Prometheus)}};
\node[block, below=12mm of dash] (interp) {\shortstack{Interpretación\\de resultados\\(comparativos pre / post)}};

% ====== FLECHAS ======
\draw[flow] (ingesta) -- (drift);
\draw[flow] (drift)   -- (alerta);
\draw[flow] (alerta.east) to[out=0,in=180] (decision.west);
\draw[flow] (decision) -- (retrain);
\draw[flow] (retrain)  -- (mlflow);
\draw[flow] (mlflow.east) to[out=0,in=180] (dash.west);
\draw[flow] (dash) -- (interp);

% ====== BUCLE ======
\draw[dashedflow]
  (interp.south) .. controls +(0,-8) and +(0,-8) ..
  (mlflow.south) .. controls +(-24,-6) and +(24,-6) ..
  (alerta.south) .. controls +(0,-8) and +(0,-8) ..
  (drift.south);

% ====== FONDO ======
\begin{pgfonlayer}{background}
  \node[lane, fit=(ingesta)(drift)(alerta)] (laneA) {};
  \node[lane, fit=(decision)(retrain)(mlflow)] (laneB) {};
  \node[lane, fit=(dash)(interp)] (laneC) {};
\end{pgfonlayer}

% ====== TÍTULOS ======
\node[anchor=south] at ($(laneA.north)+(0,6)$) {\textbf{Línea A: Datos}};
\node[anchor=south] at ($(laneB.north)+(0,6)$) {\textbf{Línea B: Automatización}};
\node[anchor=south] at ($(laneC.north)+(0,6)$) {\textbf{Línea C: Observabilidad}};
\end{tikzpicture}

\caption[Flujo de trabajo metodológico]{Flujo de trabajo metodológico: datos $\rightarrow$ detección $\rightarrow$ alerta $\rightarrow$ reentrenamiento $\rightarrow$ registro $\rightarrow$ visualización, con bucle de mejora continua.}
\label{fig:flujo_trabajo_estilizado_fit}
\end{figure}
\endgroup

A continuación, se describen los pasos metodológicos principales:

\begin{enumerate}
    \item \textbf{Revisión de literatura y diseño conceptual:}
    En esta fase se realizó una exploración bibliográfica intensiva para establecer las bases teóricas y prácticas del proyecto. Se estudiaron técnicas de detección de \textit{data drift} (como Kolmogorov-Smirnov, Chi-cuadrado y Kullback-Leibler), arquitecturas para entornos Big Data y principios de MLOps aplicados a flujos de trabajo de actualización de modelos. Este análisis permitió definir los módulos funcionales que compusieron el sistema, identificar tecnologías viables y proponer una arquitectura conceptual sobre la cual se desarrolló el prototipo. Se consolidaron los principales patrones arquitectónicos y se bosquejó un primer diseño lógico.

    La documentación generada incluye el registro de papers clave, anotaciones de arquitectura, bitácoras de decisiones y estructuras base de directorios en Git.
    
    \begin{figure}[H]
    \centering
    \begin{tikzpicture}[node distance=1.5cm]
    \node (a) [block] {Lectura de papers};
    \node (b) [block, below of=a] {Extracción de técnicas};
    \node (c) [block, below of=b] {Selección de componentes};
    \node (d) [block, below of=c] {Diseño del modelo de arquitectura};
    \draw [arrow] (a) -- (b);
    \draw [arrow] (b) -- (c);
    \draw [arrow] (c) -- (d);
    \end{tikzpicture}
    \caption{Diagrama: Revisión y diseño conceptual}
    \end{figure}

    \item \textbf{Implementación del entorno de prueba:}
    Se habilitó un entorno experimental pseudo-distribuido empleando contenedores orquestados con Docker Compose. Esta estrategia permitió aislar cada componente del sistema, controlar versiones, replicar la arquitectura en distintos entornos y escalar los servicios de manera controlada, reproduciendo condiciones similares a las de un sistema en producción.
    
    En el primer paso, el uso conjunto de \textbf{Docker + Compose} facilitarón el despliegue modular de todos los servicios involucrados: desde la gestión de datos hasta el procesamiento y monitoreo. Docker Compose coordinará múltiples contenedores definidos en un archivo de configuración, permitiendo instanciar la infraestructura de forma declarativa, eficiente y reproducible.
    
    A continuación, se puso en ejecución un \textbf{contenedor con Apache Spark y HDFS}. Spark funcionará como el motor de procesamiento distribuido sobre el cual se ejecutó transformaciones, agregaciones y simulaciones en los datos, validando así el comportamiento del sistema bajo diferentes volúmenes y velocidades. HDFS proporcionará almacenamiento persistente, imitando entornos reales donde los datos no son efímeros y deben estar disponibles para procesos posteriores como reentrenamiento o análisis de rendimiento.
    
    Luego, se implementó un módulo en Python responsable del \textbf{procesamiento de datos sintéticos y su transformación}. Aunque los datos no se generarán en esta fase, sí se transformarán para representar distintos tipos de alteraciones en la distribución (por ejemplo, desplazamientos, cambios en la varianza o presencia de valores atípicos). Estas transformaciones sirvieron para alimentar los flujos que simulen el \textbf{data drift}.
    
    Finalmente, se definieron un conjunto de \textbf{escenarios controlados} que permitan medir con precisión la capacidad del sistema para reaccionar ante los eventos simulados. Estos escenarios fueron diseñados para comparar el rendimiento del pipeline con y sin automatización, permitiendo evaluar la latencia en la detección, la estabilidad del modelo tras el reentrenamiento, y la trazabilidad completa de los cambios inducidos.

    \textit{Se documentaron los archivos de configuración YAML de Docker Compose, scripts de inicialización, métricas base y escenarios de validación, junto con anotaciones en el repositorio.}

    \begin{figure}[H]
    \centering
    \begin{tikzpicture}[node distance=1.5cm]
    \node (a) [block] {Docker + Compose};
    \node (b) [block, below of=a] {Contenedor Spark + HDFS};
    \node (c) [block, below of=b] {Dataset sintético en Python};
    \node (d) [block, below of=c] {Simulación de escenarios};
    \draw [arrow] (a) -- (b);
    \draw [arrow] (b) -- (c);
    \draw [arrow] (c) -- (d);
    \end{tikzpicture}
    \caption{Diagrama: Implementación del entorno de prueba}
    \end{figure}

    \item \textbf{Desarrollo del pipeline de detección y reentrenamiento:}
    Esta fase contempla la construcción del núcleo funcional del sistema: un pipeline automatizado que detecta desviaciones en la distribución de los datos (\textit{data drift}) y activa mecanismos de reentrenamiento del modelo en producción. La arquitectura del pipeline está orientada a la operación continua y a la capacidad de respuesta autónoma ante cambios significativos en los datos de entrada.

    \item \textbf{Uso de Airflow como ETL:}
    Aunque la prueba inicial se realizó ejecutando directamente los scripts de generación de particiones en HDFS, 
    el diseño metodológico incorpora \textbf{Apache Airflow} como orquestador del proceso ETL. 
    Para ello se implementó un \textbf{DAG} \\ (\texttt{bank\_data\_generation\_dag}) que ejecuta periódicamente 
    un contenedor Docker con el cliente PySpark (\texttt{arlequin-pyspark-client}). 
    Este DAG invoca el script \texttt{generate\_data.py}, que sintetiza transacciones bancarias con 
    probabilidad de \textit{drift} y las escribe en HDFS en formato Parquet.

   \paragraph{Representación del DAG de Airflow}
    El flujo ETL se implementó como un \textit{Directed Acyclic Graph (DAG)} en Airflow que orquesta la generación, transformación y carga de datos sintéticos hacia HDFS. Cada nodo corresponde a una tarea contenedorizada (Docker) ejecutada por el cliente PySpark.
    
    \begin{lstlisting}[style=airflow,caption={Pseudocódigo del DAG \texttt{bank\_data\_generation\_dag.py}},label={lst:airflow-dag}]
    from airflow import DAG
    from airflow.providers.docker.operators.docker import DockerOperator
    from datetime import datetime
    
    with DAG(
        dag_id="bank_data_generation_dag",
        schedule_interval="@hourly",
        start_date=datetime(2025, 1, 1),
        catchup=False,
        tags=["etl","synthetic","hdfs"]
    ) as dag:
    
        extract = DockerOperator(
            task_id="extract_data",
            image="arlequin-pyspark-client",
            command="python generate_data.py --stage extract"
        )
    
        transform = DockerOperator(
            task_id="transform_data",
            image="arlequin-pyspark-client",
            command="python generate_data.py --stage transform"
        )
    
        load = DockerOperator(
            task_id="load_to_hdfs",
            image="arlequin-pyspark-client",
            command="python generate_data.py --stage load --dst hdfs:///user/bank_data"
        )
    
        extract >> transform >> load
    \end{lstlisting}
    
    \begin{figure}[htbp]
    \centering
    %\resizebox{0.9\linewidth}{!}{%  % <- opcional; comenta si no lo necesitas
    \begin{tikzpicture}[
      >=Latex,
      line/.style={-Latex, thick},
      box/.style={draw, rounded corners, fill=gray!10, align=center, minimum width=38mm, minimum height=12mm, font=\small},
      art/.style={draw, rounded corners, fill=gray!5,  align=center, minimum width=38mm, minimum height=10mm, font=\scriptsize}
    ]
    % matriz con separación controlada
    \matrix (m) [row sep=12mm, column sep=12mm] {
      \node[box] (extract)   {Extracción\\\texttt{extract\_data}}; &
      \node[box] (transform) {Transformación\\\texttt{transform\_data}}; &
      \node[box] (load)      {Carga a HDFS\\\texttt{load\_to\_hdfs}}; \\
      \node[art] (raw)  {Artefacto: \texttt{raw\_batch.parquet}}; &
      \node[art] (stg)  {Artefacto: \texttt{staged\_batch.parquet}}; &
      \node[art] (hdfs) {HDFS: \texttt{/user/bank\_data/\{dt\}}}; \\
    };
    
    % flechas horizontales (nodos superiores)
    \draw[line] (extract) -- (transform);
    \draw[line] (transform) -- (load);
    
    % flechas verticales (ancladas a bordes, sin cruzar cajas)
    \draw[line] (extract.south) -- (raw.north);
    \draw[line] (transform.south) -- (stg.north);
    \draw[line] (load.south) -- (hdfs.north);
    \end{tikzpicture}
    %}
    \caption{Flujo ETL orquestado por Airflow: nodos, dependencias y artefactos.}
    \label{fig:airflow-etl}
    \end{figure}




    El flujo del DAG contempla:
    \begin{enumerate}
        \item \textbf{Extracción}: inicialización de un lote de datos sintéticos mediante \texttt{Faker} y reglas de estacionalidad.
        \item \textbf{Transformación}: creación de un \texttt{DataFrame} en Spark con esquema predefinido, 
              incorporación de campos derivados (p.ej. \texttt{risk\_score}) y aplicación del factor de drift.
        \item \textbf{Carga}: escritura en \texttt{HDFS} bajo la ruta \texttt{hdfs:///user/bank\_data/bank\_transactions}, 
              con particionado temporal (\texttt{timestamp}) para habilitar ingestas incrementales y pruebas de 
              detección de \textit{drift}.
    \end{enumerate}

    La inclusión de Airflow aporta \textbf{reproducibilidad y trazabilidad} (cada corrida del DAG queda registrada), 
    \textbf{aislamiento de responsabilidades} (Airflow únicamente orquesta la generación/carga, mientras el \textit{drift-watcher} 
    monitorea particiones nuevas) y \textbf{operación continua} (el intervalo de scheduling del DAG determina el ritmo 
    de llegada de datos y, por ende, la frecuencia de evaluación del \textit{drift}).
    
    \item \textbf{Entrada del sistema:} el pipeline procesará flujos de datos provenientes del entorno experimental configurado previamente, almacenados en el sistema distribuido y transformados por procesos previos de limpieza y estructuración.
    
    \item \textbf{Evaluación estadística:} la detección de drift se implementó mediante una combinación de pruebas estadísticas no paramétricas sobre ventanas móviles de datos. Se utilizaron:
    \begin{itemize}
      \item \textbf{Kolmogorov-Smirnov (KS):} para comparar la distribución empírica de nuevas observaciones con la distribución histórica del entrenamiento.
      \item \textbf{Chi-cuadrado (\(\chi^2\)):} para detectar cambios discretos en distribuciones categóricas.
      \item \textbf{Kullback-Leibler Divergence (KL):} para evaluar la diferencia entre distribuciones de probabilidad observadas y esperadas.
    \end{itemize}
    Estas pruebas se ejecutaron usando librerías de Python como \texttt{scipy.stats} y \texttt{alibi-detect}. Se configuraron umbrales adaptativos que disparen alertas cuando se supere un nivel crítico de desviación.
    
    \item \textbf{Mecanismo de decisión y activación:} cuando las pruebas detecten \textit{data drift}, se activó una tarea automatizada orquestada por \texttt{Jenkins}. Esta tarea lanzó un nuevo proceso de reentrenamiento con los datos recientes. Este componente incluyó validación cruzada y control de sobreajuste mediante técnicas como \texttt{early stopping} y registro de métricas.
    
    \item \textbf{Reentrenamiento y versionado:} una vez generado el nuevo modelo, se almacenó junto con sus métricas, hiperparámetros y configuración en \texttt{MLflow}. Esto garantizará la trazabilidad completa del ciclo de vida del modelo, incluyendo comparaciones con versiones anteriores para validar mejoras y evitar regresiones.
    
    \item \textbf{Salida esperada:} un modelo actualizado validado bajo condiciones de \textit{drift}, métricas registradas y documentadas, y control de versiones centralizado.

    \textit{Se documentó cada versión del modelo, scripts de validación, configuraciones de ejecución de Jenkins y resultados comparativos.}

    \begin{figure}[H]
    \centering
    \begin{tikzpicture}[node distance=1.5cm]
    \node (a) [block] {Entrada de datos};
    \node (b) [block, below of=a] {Evaluación estadística};
    \node (c) [block, below of=b] {Detección de drift};
    \node (d) [block, below of=c] {Activación de Jenkins};
    \node (e) [block, below of=d] {Reentrenamiento};
    \node (f) [block, below of=e] {Registro en MLflow};
    \draw [arrow] (a) -- (b);
    \draw [arrow] (b) -- (c);
    \draw [arrow] (c) -- (d);
    \draw [arrow] (d) -- (e);
    \draw [arrow] (e) -- (f);
    \end{tikzpicture}
    \caption{Diagrama: Pipeline de detección y reentrenamiento}
    \end{figure}

    \item \textbf{Validación del sistema:} 
    La fase de validación constituye un componente esencial del proyecto, pues permite determinar en qué medida el sistema implementado cumple con los objetivos planteados de detección temprana de \textit{data drift} y reentrenamiento automatizado. Para ello se diseñaron y ejecutaron dos escenarios controlados de prueba: (i) un escenario estático sin presencia de \textit{drift}, que servió como línea base de comparación, y (ii) un escenario con \textit{drift} inducido de manera progresiva mediante la alteración controlada de variables de entrada, a fin de evaluar la capacidad del sistema para detectar desviaciones y restaurar el desempeño del modelo.
    
    En cada escenario se registraron métricas de desempeño del modelo, incluyendo precisión, \textit{recall}, F1-score y AUC, complementadas con indicadores de operación del pipeline como la latencia de detección, el tiempo total de reentrenamiento y los recursos computacionales consumidos (CPU, memoria y uso de disco). Estas métricas se capturaron de manera automática a través de MLflow para el versionamiento de modelos, Prometheus para la exposición de indicadores en tiempo real, y Grafana para la visualización y consolidación de paneles comparativos.
    
    El proceso experimental considerará además la generación de múltiples corridas bajo condiciones equivalentes, con el fin de asegurar reproducibilidad y obtener promedios estadísticamente significativos. De esta manera, se podrá distinguir entre fluctuaciones aleatorias y comportamientos sistemáticos del sistema frente al \textit{data drift}. Asimismo, se documentaron aspectos de eficiencia operativa, tales como el consumo de recursos durante el reentrenamiento, el impacto del tamaño de las particiones de datos en la detección del \textit{drift}, y el efecto de los umbrales de significancia establecidos para activar el pipeline.
    
    El análisis de resultados no se limitará únicamente a la comparación numérica de métricas, sino que incluyó la elaboración de informes interpretativos por versión del modelo. Estos informes contendrán gráficos de evolución de métricas, tablas comparativas de desempeño entre escenarios y descripciones cualitativas de la respuesta del sistema. Además, se elaborarán visualizaciones específicas para resaltar la relación entre el \textit{Population Stability Index} (PSI) y las métricas de rendimiento, con el propósito de evaluar el poder predictivo del PSI como señal temprana de degradación del modelo.
    
    \textit{Esta fase producirá como entregables un conjunto de informes comparativos, visualizaciones interactivas en dashboards, métricas consolidadas y resúmenes analíticos que permitirán evaluar de manera integral la eficacia del sistema. Los resultados obtenidos sirvieron como insumo directo para la discusión final y la formulación de recomendaciones para futuros despliegues en entornos reales.}


    \begin{figure}[H]
    \centering
    \begin{tikzpicture}[node distance=1.5cm]
    \node (a) [block] {Ejecución del pipeline};
    \node (b) [block, below of=a] {Comparación de métricas};
    \node (c) [block, below of=b] {Análisis de logs};
    \node (d) [block, below of=c] {Interpretación de resultados};
    \draw [arrow] (a) -- (b);
    \draw [arrow] (b) -- (c);
    \draw [arrow] (c) -- (d);
    \end{tikzpicture}
    \caption{Diagrama: Validación del sistema}
    \end{figure}
\end{enumerate}

\fi

\subsection{Fases de desarrollo e implementación}

La investigación adoptó una \textbf{metodología aplicada y experimental}, organizada en fases secuenciales que abarcan desde el diseño conceptual hasta la validación empírica del sistema. El propósito fue construir y evaluar un \textit{pipeline} de reentrenamiento automatizado basado en detección temprana de \textit{data drift}.

Cada fase combinó \textbf{principios de ingeniería reproducible} —contenedorización, trazabilidad y modularidad— con \textbf{criterios de validación cuantitativa} orientados a métricas de desempeño y eficiencia operativa. El enfoque técnico se apoyó en herramientas de código abierto (Python, Spark, Jenkins, MLflow y Prometheus), desplegadas mediante \textbf{Docker Compose} para garantizar replicabilidad y control experimental.

\noindent
La Figura~\ref{fig:flujo_trabajo_estilizado_fit} sintetiza el flujo metodológico del proyecto, integrando las etapas de datos, automatización y observabilidad dentro de un ciclo continuo de mejora. Este diagrama no representa la implementación detallada, sino la \textbf{lógica de investigación} que guía la transición desde el diseño hasta la validación del sistema.



% necesita en el preámbulo
% \usepackage{tikz}
% \usetikzlibrary{arrows.meta,positioning,shapes.geometric,fit,backgrounds,calc}
% \usepackage{float}

\begingroup
\shorthandoff{>}
\begin{figure}[H]
\centering

% --- estilos locales ---
\definecolor{lanegray}{RGB}{248,248,248}
\definecolor{dark}{RGB}{60,60,60} % ya no usamos accent
\tikzset{
  >={Latex[length=3mm,width=2mm]},
  node distance=9mm,
  block/.style={rectangle, rounded corners=3pt, minimum width=46mm, minimum height=10mm,
                draw=dark, very thick, fill=white, align=center, inner sep=3.5pt},
  emph/.style={block, draw=dark, very thick},  % antes usaba accent
  lane/.style={rounded corners=5pt, draw=black!12, line width=0.3pt, fill=lanegray, inner sep=6mm},
  flow/.style={->, very thick, draw=dark},
  flowaccent/.style={->, very thick, draw=dark}, % antes era accent
  dashedflow/.style={->, thick, draw=dark, dashed},
}

\begin{tikzpicture}[x=0.8mm, y=0.8mm]

% ====== BLOQUES (contenido) ======
\node[block] (ingesta)    at (15,112) {\shortstack{Lectura y\\preparación de datos}};
\node[block, below=12mm of ingesta] (drift) {\shortstack{Detección de\\ \textit{data drift}\\(KS, $\chi^2$, KL / PSI)}};
\node[emph, below=12mm of drift, xshift=4mm] (alerta) {\shortstack{Alerta por\\desviación\\(umbral superado)}};

\node[block] (decision)   at (82,100) {\shortstack{Mecanismo de\\decisión\\(gatillar / no gatillar)}};
\node[emph, below=14mm of decision] (retrain) {\shortstack{Reentrenamiento\\automático}};
\node[block, below=12mm of retrain] (mlflow) {\shortstack{Registro\\y versionado}};

\node[block] (dash)       at (148,92) {\shortstack{Visualización\\de métricas\\(Grafana / Prometheus)}};
\node[block, below=12mm of dash] (interp) {\shortstack{Interpretación\\de resultados\\(comparativos pre / post)}};

% ====== FLECHAS ======
\draw[flow] (ingesta) -- (drift);
\draw[flow] (drift)   -- (alerta);
\draw[flow] (alerta.east) to[out=0,in=180] (decision.west);
\draw[flow] (decision) -- (retrain);
\draw[flow] (retrain)  -- (mlflow);
\draw[flow] (mlflow.east) to[out=0,in=180] (dash.west);
\draw[flow] (dash) -- (interp);

% ====== BUCLE ======
\draw[dashedflow]
  (interp.south) .. controls +(0,-8) and +(0,-8) ..
  (mlflow.south) .. controls +(-24,-6) and +(24,-6) ..
  (alerta.south) .. controls +(0,-8) and +(0,-8) ..
  (drift.south);

% ====== FONDO ======
\begin{pgfonlayer}{background}
  \node[lane, fit=(ingesta)(drift)(alerta)] (laneA) {};
  \node[lane, fit=(decision)(retrain)(mlflow)] (laneB) {};
  \node[lane, fit=(dash)(interp)] (laneC) {};
\end{pgfonlayer}

% ====== TÍTULOS ======
\node[anchor=south] at ($(laneA.north)+(0,6)$) {\textbf{Línea A: Datos}};
\node[anchor=south] at ($(laneB.north)+(0,6)$) {\textbf{Línea B: Automatización}};
\node[anchor=south] at ($(laneC.north)+(0,6)$) {\textbf{Línea C: Observabilidad}};
\end{tikzpicture}

\caption[Flujo de trabajo metodológico]{Flujo de trabajo metodológico: datos $\rightarrow$ detección $\rightarrow$ alerta $\rightarrow$ reentrenamiento $\rightarrow$ registro $\rightarrow$ visualización, con bucle de mejora continua.}
\label{fig:flujo_trabajo_estilizado_fit}
\end{figure}
\endgroup

A continuación, se describen los pasos metodológicos principales:

\begin{enumerate}
    \item \textbf{Revisión de literatura y diseño conceptual:}
    En esta fase se realizó una exploración bibliográfica intensiva para establecer las bases teóricas y prácticas del proyecto. Se estudiaron técnicas de detección de \textit{data drift} (como Kolmogorov-Smirnov, Chi-cuadrado y Kullback-Leibler), arquitecturas para entornos Big Data y principios de MLOps aplicados a flujos de trabajo de actualización de modelos. Este análisis permitió definir los módulos funcionales que compusieron el sistema, identificar tecnologías viables y proponer una arquitectura conceptual sobre la cual se desarrolló el prototipo. Se consolidaron los principales patrones arquitectónicos y se bosquejó un primer diseño lógico.

    La documentación generada incluye el registro de papers clave, anotaciones de arquitectura, bitácoras de decisiones y estructuras base de directorios en Git.
    
    \begin{figure}[H]
    \centering
    \begin{tikzpicture}[node distance=1.5cm]
    \node (a) [block] {Lectura de papers};
    \node (b) [block, below of=a] {Extracción de técnicas};
    \node (c) [block, below of=b] {Selección de componentes};
    \node (d) [block, below of=c] {Diseño del modelo de arquitectura};
    \draw [arrow] (a) -- (b);
    \draw [arrow] (b) -- (c);
    \draw [arrow] (c) -- (d);
    \end{tikzpicture}
    \caption{Diagrama: Revisión y diseño conceptual}
    \end{figure}

    \item \textbf{Implementación del entorno de prueba:}
    Se habilitó un entorno experimental pseudo-distribuido empleando contenedores orquestados con Docker Compose. Esta estrategia permitió aislar cada componente del sistema, controlar versiones, replicar la arquitectura en distintos entornos y escalar los servicios de manera controlada, reproduciendo condiciones similares a las de un sistema en producción.
    
    En el primer paso, el uso conjunto de \textbf{Docker + Compose} facilitarón el despliegue modular de todos los servicios involucrados: desde la gestión de datos hasta el procesamiento y monitoreo. Docker Compose coordinará múltiples contenedores definidos en un archivo de configuración, permitiendo instanciar la infraestructura de forma declarativa, eficiente y reproducible.
    
    A continuación, se puso en ejecución un \textbf{contenedor con Apache Spark y HDFS}. Spark funcionará como el motor de procesamiento distribuido sobre el cual se ejecutó transformaciones, agregaciones y simulaciones en los datos, validando así el comportamiento del sistema bajo diferentes volúmenes y velocidades. HDFS proporcionará almacenamiento persistente, imitando entornos reales donde los datos no son efímeros y deben estar disponibles para procesos posteriores como reentrenamiento o análisis de rendimiento.
    
    Luego, se implementó un módulo en Python responsable del \textbf{procesamiento de datos sintéticos y su transformación}. Aunque los datos no se generarán en esta fase, sí se transformarán para representar distintos tipos de alteraciones en la distribución (por ejemplo, desplazamientos, cambios en la varianza o presencia de valores atípicos). Estas transformaciones sirvieron para alimentar los flujos que simulen el \textbf{data drift}.
    
    Finalmente, se definieron un conjunto de \textbf{escenarios controlados} que permitan medir con precisión la capacidad del sistema para reaccionar ante los eventos simulados. Estos escenarios fueron diseñados para comparar el rendimiento del pipeline con y sin automatización, permitiendo evaluar la latencia en la detección, la estabilidad del modelo tras el reentrenamiento, y la trazabilidad completa de los cambios inducidos.

    \textit{Se documentaron los archivos de configuración YAML de Docker Compose, scripts de inicialización, métricas base y escenarios de validación, junto con anotaciones en el repositorio.}

    \begin{figure}[H]
    \centering
    \begin{tikzpicture}[node distance=1.5cm]
    \node (a) [block] {Docker + Compose};
    \node (b) [block, below of=a] {Contenedor Spark + HDFS};
    \node (c) [block, below of=b] {Dataset sintético en Python};
    \node (d) [block, below of=c] {Simulación de escenarios};
    \draw [arrow] (a) -- (b);
    \draw [arrow] (b) -- (c);
    \draw [arrow] (c) -- (d);
    \end{tikzpicture}
    \caption{Diagrama: Implementación del entorno de prueba}
    \end{figure}

    \item \textbf{Desarrollo del pipeline de detección y reentrenamiento:}
    Esta fase contempla la construcción del núcleo funcional del sistema: un pipeline automatizado que detecta desviaciones en la distribución de los datos (\textit{data drift}) y activa mecanismos de reentrenamiento del modelo en producción. La arquitectura del pipeline está orientada a la operación continua y a la capacidad de respuesta autónoma ante cambios significativos en los datos de entrada.

    \item \textbf{Uso de Airflow como ETL:}
    Aunque la prueba inicial se realizó ejecutando directamente los scripts de generación de particiones en HDFS, 
    el diseño metodológico incorpora \textbf{Apache Airflow} como orquestador del proceso ETL. 
    Para ello se implementó un \textbf{DAG} \\ (\texttt{bank\_data\_generation\_dag}) que ejecuta periódicamente 
    un contenedor Docker con el cliente PySpark (\texttt{arlequin-pyspark-client}). 
    Este DAG invoca el script \texttt{generate\_data.py}, que sintetiza transacciones bancarias con 
    probabilidad de \textit{drift} y las escribe en HDFS en formato Parquet.

   \paragraph{Representación del DAG de Airflow}
    El flujo ETL se implementó como un \textit{Directed Acyclic Graph (DAG)} en Airflow que orquesta la generación, transformación y carga de datos sintéticos hacia HDFS. Cada nodo corresponde a una tarea contenedorizada (Docker) ejecutada por el cliente PySpark.
    
    \begin{lstlisting}[style=airflow,caption={Pseudocódigo del DAG \texttt{bank\_data\_generation\_dag.py}},label={lst:airflow-dag}]
    from airflow import DAG
    from airflow.providers.docker.operators.docker import DockerOperator
    from datetime import datetime
    
    with DAG(
        dag_id="bank_data_generation_dag",
        schedule_interval="@hourly",
        start_date=datetime(2025, 1, 1),
        catchup=False,
        tags=["etl","synthetic","hdfs"]
    ) as dag:
    
        extract = DockerOperator(
            task_id="extract_data",
            image="arlequin-pyspark-client",
            command="python generate_data.py --stage extract"
        )
    
        transform = DockerOperator(
            task_id="transform_data",
            image="arlequin-pyspark-client",
            command="python generate_data.py --stage transform"
        )
    
        load = DockerOperator(
            task_id="load_to_hdfs",
            image="arlequin-pyspark-client",
            command="python generate_data.py --stage load --dst hdfs:///user/bank_data"
        )
    
        extract >> transform >> load
    \end{lstlisting}
    
    \begin{figure}[htbp]
    \centering
    %\resizebox{0.9\linewidth}{!}{%  % <- opcional; comenta si no lo necesitas
    \begin{tikzpicture}[
      >=Latex,
      line/.style={-Latex, thick},
      box/.style={draw, rounded corners, fill=gray!10, align=center, minimum width=38mm, minimum height=12mm, font=\small},
      art/.style={draw, rounded corners, fill=gray!5,  align=center, minimum width=38mm, minimum height=10mm, font=\scriptsize}
    ]
    % matriz con separación controlada
    \matrix (m) [row sep=12mm, column sep=12mm] {
      \node[box] (extract)   {Extracción\\\texttt{extract\_data}}; &
      \node[box] (transform) {Transformación\\\texttt{transform\_data}}; &
      \node[box] (load)      {Carga a HDFS\\\texttt{load\_to\_hdfs}}; \\
      \node[art] (raw)  {Artefacto: \texttt{raw\_batch.parquet}}; &
      \node[art] (stg)  {Artefacto: \texttt{staged\_batch.parquet}}; &
      \node[art] (hdfs) {HDFS: \texttt{/user/bank\_data/\{dt\}}}; \\
    };
    
    % flechas horizontales (nodos superiores)
    \draw[line] (extract) -- (transform);
    \draw[line] (transform) -- (load);
    
    % flechas verticales (ancladas a bordes, sin cruzar cajas)
    \draw[line] (extract.south) -- (raw.north);
    \draw[line] (transform.south) -- (stg.north);
    \draw[line] (load.south) -- (hdfs.north);
    \end{tikzpicture}
    %}
    \caption{Flujo ETL orquestado por Airflow: nodos, dependencias y artefactos.}
    \label{fig:airflow-etl}
    \end{figure}




    El flujo del DAG contempla:
    \begin{enumerate}
        \item \textbf{Extracción}: inicialización de un lote de datos sintéticos mediante \texttt{Faker} y reglas de estacionalidad.
        \item \textbf{Transformación}: creación de un \texttt{DataFrame} en Spark con esquema predefinido, 
              incorporación de campos derivados (p.ej. \texttt{risk\_score}) y aplicación del factor de drift.
        \item \textbf{Carga}: escritura en \texttt{HDFS} bajo la ruta \texttt{hdfs:///user/bank\_data/bank\_transactions}, 
              con particionado temporal (\texttt{timestamp}) para habilitar ingestas incrementales y pruebas de 
              detección de \textit{drift}.
    \end{enumerate}

    La inclusión de Airflow aporta \textbf{reproducibilidad y trazabilidad} (cada corrida del DAG queda registrada), 
    \textbf{aislamiento de responsabilidades} (Airflow únicamente orquesta la generación/carga, mientras el \textit{drift-watcher} 
    monitorea particiones nuevas) y \textbf{operación continua} (el intervalo de scheduling del DAG determina el ritmo 
    de llegada de datos y, por ende, la frecuencia de evaluación del \textit{drift}).
    
    \item \textbf{Entrada del sistema:} el pipeline procesará flujos de datos provenientes del entorno experimental configurado previamente, almacenados en el sistema distribuido y transformados por procesos previos de limpieza y estructuración.
    
    \item \textbf{Evaluación estadística:} la detección de drift se implementó mediante una combinación de pruebas estadísticas no paramétricas sobre ventanas móviles de datos. Se utilizaron:
    \begin{itemize}
      \item \textbf{Kolmogorov-Smirnov (KS):} para comparar la distribución empírica de nuevas observaciones con la distribución histórica del entrenamiento.
      \item \textbf{Chi-cuadrado (\(\chi^2\)):} para detectar cambios discretos en distribuciones categóricas.
      \item \textbf{Kullback-Leibler Divergence (KL):} para evaluar la diferencia entre distribuciones de probabilidad observadas y esperadas.
    \end{itemize}
    Estas pruebas se ejecutaron usando librerías de Python como \texttt{scipy.stats} y \texttt{alibi-detect}. Se configuraron umbrales adaptativos que disparen alertas cuando se supere un nivel crítico de desviación.
    
    \item \textbf{Mecanismo de decisión y activación:} cuando las pruebas detecten \textit{data drift}, se activó una tarea automatizada orquestada por \texttt{Jenkins}. Esta tarea lanzó un nuevo proceso de reentrenamiento con los datos recientes. Este componente incluyó validación cruzada y control de sobreajuste mediante técnicas como \texttt{early stopping} y registro de métricas.
    
    \item \textbf{Reentrenamiento y versionado:} una vez generado el nuevo modelo, se almacenó junto con sus métricas, hiperparámetros y configuración en \texttt{MLflow}. Esto garantizará la trazabilidad completa del ciclo de vida del modelo, incluyendo comparaciones con versiones anteriores para validar mejoras y evitar regresiones.
    
    \item \textbf{Salida esperada:} un modelo actualizado validado bajo condiciones de \textit{drift}, métricas registradas y documentadas, y control de versiones centralizado.

    \textit{Se documentó cada versión del modelo, scripts de validación, configuraciones de ejecución de Jenkins y resultados comparativos.}

    \begin{figure}[H]
    \centering
    \begin{tikzpicture}[node distance=1.5cm]
    \node (a) [block] {Entrada de datos};
    \node (b) [block, below of=a] {Evaluación estadística};
    \node (c) [block, below of=b] {Detección de drift};
    \node (d) [block, below of=c] {Activación de Jenkins};
    \node (e) [block, below of=d] {Reentrenamiento};
    \node (f) [block, below of=e] {Registro en MLflow};
    \draw [arrow] (a) -- (b);
    \draw [arrow] (b) -- (c);
    \draw [arrow] (c) -- (d);
    \draw [arrow] (d) -- (e);
    \draw [arrow] (e) -- (f);
    \end{tikzpicture}
    \caption{Diagrama: Pipeline de detección y reentrenamiento}
    \end{figure}

    \item \textbf{Validación del sistema:} 
    La fase de validación constituye un componente esencial del proyecto, pues permite determinar en qué medida el sistema implementado cumple con los objetivos planteados de detección temprana de \textit{data drift} y reentrenamiento automatizado. Para ello se diseñaron y ejecutaron dos escenarios controlados de prueba: (i) un escenario estático sin presencia de \textit{drift}, que servió como línea base de comparación, y (ii) un escenario con \textit{drift} inducido de manera progresiva mediante la alteración controlada de variables de entrada, a fin de evaluar la capacidad del sistema para detectar desviaciones y restaurar el desempeño del modelo.
    
    En cada escenario se registraron métricas de desempeño del modelo, incluyendo precisión, \textit{recall}, F1-score y AUC, complementadas con indicadores de operación del pipeline como la latencia de detección, el tiempo total de reentrenamiento y los recursos computacionales consumidos (CPU, memoria y uso de disco). Estas métricas se capturaron de manera automática a través de MLflow para el versionamiento de modelos, Prometheus para la exposición de indicadores en tiempo real, y Grafana para la visualización y consolidación de paneles comparativos.
    
    El proceso experimental considerará además la generación de múltiples corridas bajo condiciones equivalentes, con el fin de asegurar reproducibilidad y obtener promedios estadísticamente significativos. De esta manera, se podrá distinguir entre fluctuaciones aleatorias y comportamientos sistemáticos del sistema frente al \textit{data drift}. Asimismo, se documentaron aspectos de eficiencia operativa, tales como el consumo de recursos durante el reentrenamiento, el impacto del tamaño de las particiones de datos en la detección del \textit{drift}, y el efecto de los umbrales de significancia establecidos para activar el pipeline.
    
    El análisis de resultados no se limitará únicamente a la comparación numérica de métricas, sino que incluyó la elaboración de informes interpretativos por versión del modelo. Estos informes contendrán gráficos de evolución de métricas, tablas comparativas de desempeño entre escenarios y descripciones cualitativas de la respuesta del sistema. Además, se elaborarán visualizaciones específicas para resaltar la relación entre el \textit{Population Stability Index} (PSI) y las métricas de rendimiento, con el propósito de evaluar el poder predictivo del PSI como señal temprana de degradación del modelo.
    
    \textit{Esta fase producirá como entregables un conjunto de informes comparativos, visualizaciones interactivas en dashboards, métricas consolidadas y resúmenes analíticos que permitirán evaluar de manera integral la eficacia del sistema. Los resultados obtenidos sirvieron como insumo directo para la discusión final y la formulación de recomendaciones para futuros despliegues en entornos reales.}


    \begin{figure}[H]
    \centering
    \begin{tikzpicture}[node distance=1.5cm]
    \node (a) [block] {Ejecución del pipeline};
    \node (b) [block, below of=a] {Comparación de métricas};
    \node (c) [block, below of=b] {Análisis de logs};
    \node (d) [block, below of=c] {Interpretación de resultados};
    \draw [arrow] (a) -- (b);
    \draw [arrow] (b) -- (c);
    \draw [arrow] (c) -- (d);
    \end{tikzpicture}
    \caption{Diagrama: Validación del sistema}
    \end{figure}
\end{enumerate}

\fi
\section{Resultados obtenidos}

\subsection{Resultados cuantitativos}

\paragraph{Prototipo y objetivo.}
El prototipo \textbf{Arlequín} demostró capacidad de \emph{detección temprana} y \emph{respuesta automática} frente al \textit{data drift}. El sistema monitorea continuamente la distribución de entrada y, al superar umbrales de significancia, activa el reentrenamiento y registra el ciclo completo.

En términos de \textbf{evidencia empírica}, los principales resultados son:
\paragraph{Resultados observados.}
\begin{itemize}\setlength\itemsep{2pt}
  \item \textbf{Detección y magnitud del \textit{drift}:} el \textit{Population Stability Index} (PSI) superó el umbral de alerta en el escenario con \textit{drift}, confirmando desviaciones de distribución capturadas por KS y $\chi^2$.
  \item \textbf{Latencias operativas:} \textit{TTFD} en el orden de minutos sub–minutales y \textit{TTR} acotado a minutos, suficientes para reacción oportuna sin intervención humana.
  \item \textbf{Desempeño del modelo:} caída de F1/Recall ante \textit{drift} y \textbf{recuperación posterior} tras el reentrenamiento automático (no–inferioridad respecto a la línea base).
  \item \textbf{Trazabilidad:} ejecución y artefactos versionados en MLflow; series temporales de métricas expuestas en Prometheus y visualizadas en Grafana.
\end{itemize}

\noindent
\textbf{Fundamentaci\'on estad\'istica e IC95\%.} Para F1, PSI, \textit{TTFD} y \textit{TTR} se estimaron intervalos de confianza al 95\,\% mediante \textbf{bootstrap percentil} ($B=1000$) sobre r\'eplicas experimentales por escenario; para tiempos se reportan intervalos percentil adecuados a distribuciones sesgadas. Las definiciones y pruebas formales aparecen en el Cap\'itulo~\ref{sec:eval-design}.

\noindent
\textbf{Interpretación.} En conjunto, la combinación de KS/$\chi^2$/PSI ofrece señales operativas útiles (detección + severidad), mientras que el gatillo de reentrenamiento restituye el desempeño con costos temporales controlados. Esta evidencia soporta la hipótesis de que un ciclo MLOps automatizado \emph{reduce latencias} y \emph{restaura} la precisión bajo no–estacionariedad.


Consistente con H1 y H2 del Capitulo~4 (Seccion~\ref{subsec:eval-rq}).
\paragraph{Condiciones experimentales.}
Los experimentos se realizaron bajo tres condiciones controladas:
\begin{enumerate}
    \item \textbf{E1 – Base:} flujo de datos sin drift (línea base).
    \item \textbf{E2 – Drift inducido:} aumento progresivo en las variables \texttt{amount} y \texttt{risk\_score}.
    \item \textbf{E3 – Reentrenado:} modelo actualizado tras la activación automática del \textit{pipeline}.
\end{enumerate}

En cada escenario se registraron métricas clave en MLflow y Prometheus: \textit{F1-score}, \textit{Recall}, \textit{PSI}, y tiempos de detección (\textit{TTFD}) y reentrenamiento (\textit{TTR}). Los resultados promedio se resumen en la Tabla~\ref{tab:metricas-resultados}.

\begin{table}[H]
\centering
\caption{Resumen comparativo de métricas por escenario experimental.}
\label{tab:metricas-resultados}
\begin{tabular}{lccccc}
\toprule
\textbf{Escenario} & \textbf{F1 (IC95\%)} & \textbf{Recall} & \textbf{PSI (IC95\%)} & \textbf{TTFD (s, IC95\%)} & \textbf{TTR (s, IC95\%)} \\
\midrule
E1 -- Base (sin drift)   & 0.825 (IC95\%: [0.818, 0.832]) & 0.84 & 0.005 (IC95\%: [0.000, 0.010]) & -- & -- \\
E2 -- Con drift inducido & 0.811 (IC95\%: [0.808, 0.814]) & 0.65 & 0.230 (IC95\%: [0.200, 0.260]) & 270 (IC95\%: [180, 360]) & 102 (IC95\%: [84, 151]) \\
E3 -- Reentrenado        & 0.817 (IC95\%: [0.813, 0.821]) & 1.00 & 0.010 (IC95\%: [0.000, 0.020]) & 271 (IC95\%: [180, 360]) & 103 (IC95\%: [90, 124]) \\
\bottomrule
\end{tabular}
\end{table}

\noindent\footnotesize\emph{Nota:} Los IC$_{95\%}$ de TTFD y TTR se obtuvieron mediante \textit{bootstrap} percentil ($B=1000$) sobre las 9 transiciones continuas registradas en las 5 réplicas experimentales. Los intervalos inter-réplica superiores a 5\,000\,s (debidos a reinicios manuales sin flujo de datos) se excluyeron del cómputo porque no representan la latencia operativa del detector.

\noindent\textbf{Tamaño del efecto.} El contraste entre E1 y E2 (Tabla~\ref{tab:mw-tests}) arrojó $p=2.55\times10^{-3}$ con tamaño de efecto Cliff $\delta=0.57$ (magnitud grande), lo que confirma la degradación estadísticamente significativa del F1 durante el \textit{drift}. Tras el reentrenamiento (E3), el efecto se reduce a $\delta=0.54$ frente al escenario degradado, evidenciando la recuperación hacia la línea base.

El escenario E2 evidenció una degradación significativa del desempeño (F1 $\downarrow$ 1.37\,pp respecto a E1), acompañada de un PSI que supera el umbral operativo ($0.230$ con IC$_{95\%}$ [0.200, 0.260]). El sistema detectó el \textit{drift} en una mediana de $270$\,s y activó el reentrenamiento automático, tras el cual el modelo se estabilizó de nuevo (E3) con F1 $=0.817$ (IC$_{95\%}$ [0.813, 0.821]) y PSI próximo a cero. Estas métricas confirman la capacidad del sistema para detectar y corregir desviaciones en tiempo casi real.

\noindent\textbf{Limitación del conjunto de datos.} Todos los experimentos se realizaron con flujos sintéticos controlados; esto asegura trazabilidad pero puede subestimar la varianza y la complejidad estructural presentes en datos reales. En consecuencia, los resultados deben considerarse conservadores hasta ejecutar réplicas con fuentes productivas o semi-sintéticas de mayor diversidad.

\begin{figure}[H]
\centering
\begin{tikzpicture}
\begin{axis}[
    width=0.85\linewidth,
    xlabel={PSI},
    ylabel={F1-score},
    ymin=0, ymax=1,
    xmin=0, xmax=0.4,
    grid=both,
    major grid style={dotted},
    scatter/classes={a={mark=*,blue}},
]
\addplot[mark=*,blue] coordinates {
    (0.05,0.83)
    (0.10,0.80)
    (0.20,0.75)
    (0.31,0.62)
    (0.06,1.00)
};
\end{axis}
\end{tikzpicture}
\caption{Relación entre la magnitud del PSI y el desempeño del modelo (F1-score).}
\label{fig:psi-f1}
\end{figure}

La Figura~\ref{fig:psi-f1} evidencia la correlación negativa entre el PSI y el F1-score: a medida que la desviación de distribución aumenta, el rendimiento del modelo disminuye. El reentrenamiento automático restaura simultáneamente el desempeño y reduce el PSI, validando la efectividad del mecanismo de corrección.

\subsection{Discusión inicial}
Los resultados cuantitativos respaldan la hipótesis H\textsubscript{1} planteada en la Sección~\ref{sec:metodologia}: un sistema MLOps automatizado puede restaurar el desempeño del modelo tras la detección de \textit{data drift} en entornos no estacionarios. La secuencia E1–E2–E3 demuestra empíricamente la sensibilidad del detector y la capacidad de recuperación del \textit{pipeline}.

\textbf{Interpretación de los resultados.}  
El incremento del PSI y la caída simultánea del F1 confirman que las pruebas estadísticas no paramétricas (KS, $\chi^2$ y PSI) capturan eficazmente desviaciones en la distribución. El retorno a F1=1.0 tras el reentrenamiento valida la hipótesis de mejora significativa post-activación automática.

\textbf{Limitaciones y consideraciones estadísticas.}
\begin{itemize}
    \item \textbf{Naturaleza sintética de los datos:} la simplicidad del generador (\textit{Faker}) limita la variabilidad y puede sobreestimar la generalización del sistema.
    \item \textbf{Tamaño de muestra y repetibilidad:} el número de ejecuciones ($n=5$) restringe la estimación de intervalos de confianza e impide análisis inferenciales robustos.
    \item \textbf{Métodos no paramétricos:} si bien las pruebas KS, $\chi^2$ y PSI son apropiadas para distribuciones arbitrarias, no permiten inferir causalidad ni modelar interacciones multivariadas.
\end{itemize}

Los hallazgos ofrecen una validación inicial del enfoque metodológico, demostrando la utilidad del PSI como indicador temprano de degradación y la eficacia del reentrenamiento automatizado para recuperar la estabilidad del modelo. Estas observaciones sientan la base para un análisis estadístico más riguroso en el capítulo de evaluación.


%%%%%%%%%%%%%%%%%%%%%%
% DESCRIPCION DEL PROBLEMA
%%%%%%%%%%%%%%%%%%%%%%
\chapter{Marco de referencia}
%%%%%%%%%%%%%%%%%%%%%%
% Marco teorico y antecedentes
%%%%%%%%%%%%%%%%%%%%%%

Este apartado recopila y organiza los fundamentos conceptuales y antecedentes relevantes al problema abordado, sirviendo de guía para orientar el enfoque del proyecto. Se analizan tanto las bases teóricas que sustentan la investigación como el estado actual de las soluciones y estudios relacionados, lo cual permite identificar vacíos y oportunidades en el contexto de la detección de \textit{data drift} y la automatización del reentrenamiento.

\subsection{Bases Teóricas}
Las bases teóricas del proyecto se sustentan en diversos pilares conceptuales que explican el comportamiento de los modelos de \textit{machine learning} en entornos dinámicos, la necesidad de sistemas adaptativos y los enfoques tecnológicos subyacentes \citep{Kodakandla2024}.

\subsubsection{Machine Learning y Data Drift}
Los modelos de \textit{machine learning} comúnmente parten de la premisa de que la distribución de los datos de entrenamiento se mantiene estable a lo largo del tiempo. En escenarios reales, esta condición casi nunca se cumple, pues los datos pueden experimentar variaciones, un fenómeno denominado \textbf{\textit{data drift}} o \textbf{\textit{concept drift}} \citep{Lu2019, Gama2014, Zliobaite2016}. El \textit{data drift} puede manifestarse de diversas maneras:

\begin{itemize}
    \item \textbf{Cambios de distribución (covariate shift):} Se altera la distribución de las variables de entrada, afectando la capacidad predictiva del modelo.
    \item \textbf{Cambios de concepto (concept shift):} Se modifican las relaciones subyacentes entre variables de entrada y la variable objetivo, generando una posible degradación en el desempeño.
    \item \textbf{Drift virtual vs.\ real:} El primero indica cambios aparentes sin modificar la relación subyacente, mientras que el segundo implica un cambio genuino en el proceso generador de los datos \citep{Gama2014}.
\end{itemize}

Para abordar estos desafíos, se han desarrollado técnicas de \textbf{detección temprana} de \textit{data drift}, que van desde métodos estadísticos no paramétricos (test de Kolmogorov-Smirnov, Chi-cuadrado o divergencia de Kullback-Leibler) hasta algoritmos de \textit{machine learning} supervisados o no supervisados \citep{Chawla2021, Sethi2017}. Adicionalmente, métodos como el \textbf{Early Drift Detection Method (EDDM)} se han mostrado efectivos para identificar cambios sutiles o graduales en los datos \citep{BaenaGarcia2006}.

\paragraph{Formas de Detección de Data Drift}
De acuerdo con la literatura \citep{Lu2019, Gama2014, BaenaGarcia2006}, se pueden distinguir los siguientes enfoques principales para detectar \textit{data drift}:
\begin{itemize}
    \item \textbf{Detección basada en estadísticos de distribución:} Compara la distribución de las variables (o la salida del modelo) en distintas ventanas de datos (\textit{sliding windows}), utilizando pruebas como Kolmogorov-Smirnov o la divergencia de Kullback-Leibler.
    \item \textbf{Métodos de supervisión externa:} Se introduce un clasificador específico que compara los datos antiguos frente a los datos nuevos para determinar si ha ocurrido un cambio en la distribución o en la relación subyacente.
    \item \textbf{Monitoreo de métricas de rendimiento:} Se revisan periódicamente indicadores como la exactitud, \textit{precision}, \textit{recall} o la puntuación \textit{F1}, ya que la degradación en estas métricas puede evidenciar \textit{data drift} \citep{Bifet2010}.
    \item \textbf{Combinaciones híbridas:} Integran pruebas estadísticas con algoritmos de \textit{machine learning} para lograr una detección más robusta.
\end{itemize}

La elección de pruebas como Kolmogorov-Smirnov, Chi-cuadrado o el Population Stability Index (PSI) responde a su amplia utilización en la literatura de \textit{drift detection} y a su complementariedad. El test de Kolmogorov-Smirnov es sensible a variaciones en la forma de la distribución (media, varianza o asimetría) de variables continuas, mientras que la prueba Chi-cuadrado resulta idónea para identificar cambios en frecuencias categóricas \citep{Sethi2017}. Por su parte, el PSI es ampliamente adoptado en entornos productivos -particularmente en finanzas y riesgo crediticio- porque resume la magnitud de la desviación en un valor único fácilmente interpretable. Estas métricas permiten cubrir distintos tipos de variables y escenarios de \textit{drift}, equilibrando sensibilidad estadística y aplicabilidad práctica \citep{Gama2014, Zliobaite2016}.

\subsection{Contraste crítico y criterios de selección}
Para maximizar validez práctica en operación, la selección del detector debe ponderar: (i) tipo de variable (continua/categórica/mixta), (ii) modo de operación (\textit{batch} vs. \textit{streaming}), (iii) sensibilidad vs. tasa de falsas alarmas, (iv) costo computacional y memoria, y (v) facilidad de calibración e interpretabilidad \citep{Lu2019, Gama2014, Bifet2010}.

- Variables: KS es adecuado para continuas; $\chi^2$ para categóricas; PSI admite numéricas y categóricas vía \textit{binning}; KL requiere discretización o estimación de densidad.
- Modo: ADWIN/EDDM operan en flujo con actualización incremental; KS/$\chi^2$/PSI/KL suelen emplearse en ventanas \textit{batch} o deslizantes.
- Costos: pruebas de conteo (PSI, $\chi^2$) son livianas; KS/KL medianos; detectores de flujo (ADWIN/EDDM) tienen costo amortizado bajo por evento.
- Calibración: PSI y $\chi^2$ son fáciles de umbralizar; KL requiere suavizado; KS exige tamaños muestrales suficientes por ventana.
- Multivariado: las versiones univariantes requieren control de multiplicidad o métricas agregadas; el soporte multivariado puro exige técnicas adicionales.

\subsection{Cuadro comparativo de detectores}
\label{subsec:cuadro-detectores}
El Cuadro~\ref{tab:comparativa-detectores} contrasta los detectores más usados por tipo de variable y costo computacional (orden relativo por ventana o por evento), según la literatura \citep{Gama2014, Lu2019, Bifet2010, BaenaGarcia2006, Sethi2017}.

\begin{table}[htbp]
\centering
\caption{Detectores de \textit{drift}: tipo de variable vs. costo computacional (relativo).}
\label{tab:comparativa-detectores}
\begin{tabular}{llll}
\toprule
\textbf{Detector} & \textbf{Tipo de variable} & \textbf{Modo} & \textbf{Costo (aprox.)} \\
\midrule
Kolmogorov--Smirnov (KS) & Continua (univar.) & Ventanas (batch) & Medio ($\mathcal{O}(n\log n)$ por ventana) \\
$\chi^2$ de independencia & Categórica & Ventanas (batch) & Bajo ($\mathcal{O}(k)$; conteos) \\
Kullback--Leibler (KL) & Num./cat. discretizada & Ventanas (batch) & Medio ($\mathcal{O}(k)$; requiere suavizado) \\
Population Stability Index (PSI) & Num./cat. (con binning) & Ventanas (batch) & Bajo ($\mathcal{O}(\text{bins})$) \\
EDDM \citep{BaenaGarcia2006} & Métrica de error (cualquiera) & Flujo (\textit{online}) & Bajo (\,\, $\mathcal{O}(1)$ por evento) \\
ADWIN \citep{Bifet2010} & Métrica de error (cualquiera) & Flujo (\textit{online}) & Bajo/medio (amort. $\mathcal{O}(\log n)$) \\
\bottomrule
\end{tabular}
\end{table}

Notas: $k$ es el número de categorías o \textit{bins}. En práctica, el costo efectivo depende del tamaño de ventana, número de variables monitorizadas y política de muestreo. Para escenarios multivariados, se recomienda (i) combinación de pruebas univariantes con control de FDR, o (ii) detectores multivariados según el caso de uso (no desarrollados aquí por brevedad) \citep{Lu2019, Gama2014}.

\subsubsection{MLOps y Automatización del Ciclo de Vida de los Modelos}
El paradigma \textbf{MLOps} integra las prácticas de \textit{DevOps} (Integración y Despliegue Continuo, CI/CD) con las necesidades específicas de entrenamiento, validación y mantenimiento de modelos de \textit{machine learning}:

\begin{itemize}
    \item \textbf{Monitorización continua:} Permite detectar a tiempo los cambios en la calidad de las predicciones y, por ende, en la distribución de los datos \citep{Amershi2019}.
    \item \textbf{Integración y despliegue automatizado (CI/CD):} Garantiza una reacción rápida cuando se detecta \textit{data drift}, lanzando procesos de reentrenamiento y actualización de modelos en producción \citep{Chen2022, Kim2018}.
    \item \textbf{Trazabilidad y versionado:} El uso de herramientas como MLflow facilita el registro y la comparación de experimentos, así como la documentación de las versiones de los modelos en cada iteración \citep{Chen2022}.
\end{itemize}

\subsubsection{Big Data y Tecnologías Emergentes}
El manejo de grandes volúmenes de datos implica adoptar tecnologías de \textbf{Big Data}:
\begin{itemize}
    \item \textbf{Procesamiento distribuido:} \textit{Frameworks} como \textit{Apache Spark} permiten análisis en \textit{batch} y en tiempo real de forma escalable \citep{Chen2022}.
    \item \textbf{Contenerización y orquestación:} \textit{Docker} y \textit{Kubernetes} facilitan la empaquetación y administración de aplicaciones en entornos heterogéneos, brindando portabilidad y escalabilidad \citep{Merkel2014, Patel2020}.
    \item \textbf{Monitorización y paneles de control:} Herramientas como Prometheus y Grafana ofrecen la capacidad de vigilar métricas de rendimiento y detectar anomalías en tiempo real, desencadenando acciones de reentrenamiento automático \citep{Zhao2021}.
\end{itemize}

\subsubsection{Adaptación y Aprendizaje en Línea}
Los métodos de \textbf{aprendizaje en línea} y \textbf{adaptación continua} permiten que los modelos se ajusten de manera incremental a medida que reciben datos:
\begin{itemize}
    \item \textbf{Ventanas deslizantes (sliding windows):} Se da prioridad a los datos recientes para que el modelo refleje la distribución actual \citep{Kolter2007, Bifet2010}.
    \item \textbf{Ponderación de instancias:} Concede mayor relevancia a ejemplos más nuevos para acelerar la adaptación al cambio \citep{Minku2010}.
    \item \textbf{Reentrenamiento periódico o gatillado:} Se combina la detección de \textit{data drift} con la activación selectiva de un reentrenamiento parcial o completo, buscando optimizar recursos computacionales y tiempo de inactividad.
\end{itemize}

\subsection{Evaluación de Sistemas de Detección y Reentrenamiento}
La eficacia de un sistema de detección de \textit{data drift} y automatización de reentrenamiento se valora en múltiples dimensiones \citep{Gama2014, Zliobaite2016}:
\begin{itemize}
    \item \textbf{Métricas de rendimiento del modelo:} \textit{Precision}, \textit{recall}, \textit{F1-score}, \textit{AUC} (Área Bajo la Curva) y otras, para cuantificar la calidad de las predicciones antes y después de la actualización.
    \item \textbf{Tiempo de detección (time-to-detect):} Intervalo entre el momento en que ocurre el \textit{drift} y el instante en que se emite la alerta; un sistema óptimo debería identificarlo con rapidez \citep{Sethi2017}.
    \item \textbf{Tasa de falsas alarmas:} Mide cuántas veces el detector produce avisos de \textit{drift} sin que realmente haya un cambio significativo, lo cual conduce a reentrenamientos innecesarios y sobrecarga de recursos.
    \item \textbf{Costo computacional y escalabilidad:} Incluye el análisis de consumo de CPU, memoria y tiempo de procesamiento requerido por la infraestructura \textit{Big Data}, que puede tener un impacto directo en la viabilidad económica de la solución.
\end{itemize}

Combinar estas métricas y criterios de evaluación posibilita un análisis integral del sistema, permitiendo la implementación de mejoras que optimicen el rendimiento predictivo, reduzcan los costos operativos y garanticen una reacción ágil ante cambios en los datos.

\subsection{Estado del Arte}

El análisis del estado del arte en la literatura académica y técnica permite identificar los principales enfoques, avances y limitaciones que existen actualmente en torno a la detección del \emph{data drift} y la automatización de pipelines MLOps en entornos Big Data.

En cuanto a las estrategias de detección de \textit{data drift}, se ha documentado ampliamente el uso de técnicas estadísticas tradicionales como las pruebas de Kolmogorov-Smirnov, Chi-cuadrado y la divergencia de Kullback-Leibler \citep{Chawla2021, Lu2019}. Además, algunos enfoques recientes han propuesto combinaciones híbridas que integran métodos estadísticos con modelos supervisados, con el objetivo de aumentar la sensibilidad a cambios sutiles en la distribución de los datos \citep{BaenaGarcia2006, Singh2023}. Sin embargo, una revisión detallada evidencia que existe una escasez de estudios que comparen de manera sistemática estas técnicas en contextos de datos masivos y variados, además de una limitada integración práctica en pipelines completamente automatizados \citep{Nguyen2023, Chatterjee2023}.

En lo referente a la automatización del reentrenamiento, se ha identificado que, aunque algunos estudios reportan avances en la implementación de flujos automáticos que abarcan desde la detección de desviaciones hasta el despliegue continuo del modelo, muchos de ellos aún requieren intervención manual en etapas críticas. Esto introduce demoras en la respuesta y limita la escalabilidad de las soluciones en ambientes dinámicos \citep{Amershi2019}. También se ha detectado una falta de estandarización en las prácticas para integrar tecnologías abiertas y escalables, como Jenkins o MLflow, en arquitecturas robustas y reproducibles \citep{RodriguezSimmhan2023, Kapoor2023, Abou2023, DeSousa2023}.

Por otro lado, el ecosistema tecnológico que habilita estos procesos se ha visto favorecido por la adopción de plataformas como Apache Spark, Docker y Kubernetes, que permiten procesamiento distribuido, contenerización y orquestación eficiente \citep{Chen2022, Merkel2014, Patel2020, Zhao2021}. A pesar de ello, persiste una falta de estudios empíricos sobre la escalabilidad y sostenibilidad de estas soluciones en condiciones reales de operación, así como análisis de sus implicaciones económicas a gran escala \citep{Xu2022, Lopez2022}.

Además, si bien existen propuestas funcionales que abordan el problema del \textit{data drift}, la validación empírica de estos sistemas sigue siendo limitada. Hay una carencia de experimentos replicables que evalúen el desempeño integral de las soluciones en distintos sectores productivos, lo cual limita su aplicabilidad general y la construcción de benchmarks estandarizados \citep{Agarwal2023, Singh2023}.

En total, se han revisado 19 estudios que abordan la detección del \textit{data drift} y la automatización del reentrenamiento. De ellos, la mayoría propone el uso de pruebas estadísticas, mientras que una fracción menor recurre a enfoques híbridos o modelos supervisados. A pesar de avances considerables en algunos frentes, se mantiene una brecha significativa en cuanto a la integración automatizada y escalable de estas soluciones en entornos productivos. Esta investigación se propone contribuir a ese vacío, desarrollando un sistema replicable que permita gestionar el \textit{data drift} de manera proactiva, automatizada y trazable en contextos reales.

Diversas plataformas empresariales ofrecen actualmente capacidades para automatizar el monitoreo de modelos \citep{Bhatt2025}, la detección de \textit{data drift} y el reentrenamiento en entornos reales \citep{Berberi2025}. Entre las más destacadas se encuentran AWS SageMaker, Google Vertex AI y Azure Machine Learning. En el caso de AWS SageMaker, la herramienta \texttt{Model Monitor} permite evaluar continuamente los datos de entrada y salida del modelo, generando alertas ante desviaciones relevantes. Sin embargo, su carácter cerrado y dependiente del ecosistema AWS limita su personalización para dominios con métricas particulares.

Google Vertex AI también proporciona funcionalidades avanzadas de detección de \textit{drift} y reentrenamiento, pero su fuerte integración con AutoML y su arquitectura opaca dificultan la intervención directa en el pipeline. Por su parte, Azure Machine Learning incluye herramientas para CI/CD y monitoreo con Azure Monitor, aunque presenta restricciones similares en términos de extensibilidad metodológica y adaptabilidad a casos de uso altamente personalizados.

% \paragraph{Capacidades de Azure Machine Learning y Azure Monitor para pipelines MLOps}

Microsoft Azure ofrece un ecosistema integrado para la gestión completa del ciclo de vida de los modelos de \textit{machine learning}. En particular, \textbf{Azure Machine Learning (Azure ML)} proporciona un entorno escalable para entrenamiento, despliegue y monitoreo continuo de modelos, con énfasis en reproducibilidad y trazabilidad. Sus principales capacidades incluyen:

\begin{itemize}
    \item \textbf{Automatización y CI/CD:} permite definir \textit{pipelines} de entrenamiento y despliegue continuo integrados con Azure DevOps y GitHub Actions, reduciendo la intervención manual y garantizando control de versiones sobre datos, modelos y código \citep{AzureML2023}.
    \item \textbf{Detección de \textit{drift}:} incluye módulos nativos de \textit{Data Drift Monitoring}, los cuales comparan distribuciones históricas con datos recientes mediante métricas estadísticas y alertas configurables, activando procesos de reentrenamiento cuando se superan umbrales críticos.
    \item \textbf{Monitoreo operacional:} la integración con \textbf{Azure Monitor} y \textbf{Application Insights} facilita la observabilidad de modelos en producción, centralizando métricas de inferencia, consumo de recursos y desempeño, lo que habilita diagnósticos rápidos y escalamiento automático \citep{AzureMonitor2022}.
    \item \textbf{Escalabilidad en la nube:} gracias a la compatibilidad con clústeres de cómputo administrados (CPU/GPU), Azure ML soporta cargas intensivas y permite ajustar dinámicamente los recursos de entrenamiento e inferencia en función de la demanda.
\end{itemize}

Estas capacidades posicionan a Azure como una alternativa robusta frente a SageMaker y Vertex AI, con la ventaja de su integración nativa con el ecosistema empresarial de Microsoft. No obstante, la literatura técnica señala que su carácter parcialmente cerrado y la dependencia de servicios gestionados pueden limitar la flexibilidad metodológica para investigadores que requieren un control granular de cada etapa del pipeline \citep{Lopez2022, Xu2022}.

\begin{table}[htbp]
\centering
\caption{Resumen comparativo de plataformas MLOps gestionadas frente al \textit{stack} abierto propuesto.}
\label{tab:mlops-cloud-compare}
\footnotesize
\resizebox{\textwidth}{!}{%
\begin{tabular}{lcccc}
\toprule
\textbf{Criterio} & \textbf{Azure ML} & \textbf{AWS SageMaker} & \textbf{Vertex AI} & \textbf{Stack abierto (Jenkins+MLflow+Spark)} \\
\midrule
Apertura / \textit{lock-in} & Media (SDK, servicios) & Media & Media & Alta (OSS, portable) \\
\textit{Drift} nativo & Sí (Data Drift Monitor)\textsuperscript{1} & Sí (Model Monitor)\textsuperscript{2} & Sí (Monitoring)\textsuperscript{3} & Configurable (Prometheus + detectores) \\
CI/CD nativo & Azure DevOps / GitHub Actions & CodePipeline / CodeBuild & Cloud Build & Jenkins / GitHub Actions \\
Observabilidad integrada & Azure Monitor / App Insights\textsuperscript{4} & CloudWatch & Cloud Logging / Monitoring & Prometheus / Grafana \\
Registro de modelos & Registry nativo & Registry nativo & Registry nativo & MLflow Model Registry \\
Escalabilidad gestionada & AmlCompute (autoscaling) & Managed compute & Managed compute & Kubernetes (autoscaling) \\
Extensibilidad metodológica & Media--Alta & Media & Media & Alta (control total) \\
Portabilidad / reproducibilidad & Media (plantillas+SDK) & Media & Media & Alta (contenedores+IaC) \\
Gobernanza / cumplimiento & Alta (RBAC, policies) & Alta & Alta & Depende de configuración OSS \\
Costos / transparencia & Por servicio (\$) & Por servicio (\$) & Por servicio (\$) & Infraestructura propia \\
\bottomrule
\end{tabular}}
\\[2pt]
\raggedright\footnotesize
\textsuperscript{1}\citep[{\small fuente secundaria}]{AzureML2023}\;
\textsuperscript{2}\citep[{\small fuente secundaria}]{SageMakerMonitor2023}\;
\textsuperscript{3}\citep[{\small fuente secundaria}]{VertexAI2023}\;
\textsuperscript{4}\citep[{\small fuente secundaria}]{AzureMonitor2022}
\end{table}



En paralelo a las plataformas gestionadas, la literatura describe el uso de \textbf{stacks abiertos} (p. ej., Jenkins + MLflow + Spark sobre Docker/Kubernetes) para entornos de investigación y prototipado, destacando su control granular y trazabilidad \citep{RodriguezSimmhan2023,Kapoor2023,DeSousa2023}. La elección entre servicios gestionados y componentes abiertos se fundamenta en criterios comparativos (apertura/\textit{lock-in}, portabilidad, costo operativo, requisitos de auditoría) más que en preferencias universales.

\vspace{0.25em}

\paragraph{Validez externa}
La arquitectura validada en entornos \textit{on-premise}/contenedorizados mantiene \textbf{validez externa} frente a nubes gestionadas por correspondencia funcional de componentes (\textit{Spark/HDFS}→\textit{Databricks/ADLS}, \textit{Jenkins}→\textit{Azure DevOps}, \textit{MLflow}→\textit{AML Registry}). 
Dado que los mecanismos de detección (KS/$\chi^{2}$/PSI), política de activación (\textit{edge-trigger + cooldown}) y trazabilidad (runs, métricas, artefactos) son invariantes a la infraestructura, los efectos observados—tiempo de detección, recuperación de F1 y reducción de intervención manual—pueden extrapolarse bajo supuestos realistas de latencia y seguridad gestionada. 
Las diferencias esperadas se concentran en \textit{SLOs} y costos por uso; se recomienda replicar la evaluación en nube registrando TTFD, TTR y consumo para cuantificar el \textit{overhead} operativo y económico.


\section{Resumen del capítulo}

Este capítulo presentó los fundamentos conceptuales y los antecedentes que sustentan la investigación. En primer lugar, se expusieron las bases teóricas del \textit{data drift} y su impacto en el rendimiento de los modelos de \textit{machine learning}, destacando sus diferentes manifestaciones (cambios de distribución, cambios de concepto, y \textit{drift} virtual vs.\ real) y las técnicas estadísticas comúnmente empleadas para su detección. Asimismo, se revisaron los enfoques de aprendizaje en línea y adaptación continua, que permiten a los modelos responder de manera incremental a escenarios de datos dinámicos.

Posteriormente, se analizó el papel de las prácticas de \textbf{MLOps} en la automatización del ciclo de vida de los modelos, resaltando la importancia de la monitorización continua, la integración y despliegue automatizado, así como la trazabilidad mediante herramientas como MLflow. Se complementó esta visión con la revisión de tecnologías de \textbf{Big Data}, tales como Apache Spark para procesamiento distribuido, Docker y Kubernetes para contenerización y orquestación, y Prometheus/Grafana para la observabilidad en tiempo real.

En cuanto a la evaluación de sistemas de detección y reentrenamiento, se identificaron métricas clave como precisión, \textit{recall}, F1-score, latencia de detección, tasa de falsas alarmas y costo computacional, cuya combinación permite valorar integralmente la eficacia y viabilidad de las soluciones propuestas. Estos criterios no solo ofrecen una perspectiva cuantitativa del rendimiento, sino que también sirven como guía para el diseño de sistemas robustos y escalables.

El estado del arte evidenció un uso predominante de pruebas estadísticas tradicionales, junto con propuestas híbridas que integran algoritmos supervisados. Sin embargo, se identificaron brechas importantes: la escasez de estudios que validen empíricamente estas técnicas en escenarios reales de datos masivos, la falta de estandarización en la integración de tecnologías abiertas dentro de pipelines automatizados, y la limitada replicabilidad de experimentos en sectores productivos. Aunque plataformas como AWS SageMaker, Google Vertex AI y Azure Machine Learning ofrecen servicios avanzados de monitoreo y reentrenamiento, presentan restricciones de flexibilidad y dependencia tecnológica que pueden dificultar su adopción en contextos con necesidades particulares.


%%%%%%%%%%%%%%%%%%%%%%
% DESARROLLO DEL PROYECTO
%%%%%%%%%%%%%%%%%%%%%%
\chapter{Desarrollo del Proyecto}
%%%%%%%%%%%%%%%%%%%%%%
% CAPÍTULO 3 — DESARROLLO DEL PROYECTO (enfoque por objetivos)
%%%%%%%%%%%%%%%%%%%%%%
%\chapter{Desarrollo del Proyecto}

Este capítulo se estructura por objetivos específicos, presentando para cada uno: planteamiento y alcance, fundamentos y criterios de diseño, propuesta técnica (modelo/algoritmo/arquitectura), implementación, y análisis de resultados, seguidos de discusión crítica, amenazas a la validez y conclusiones parciales.

\paragraph{Enfoque y posicionamiento.}
La propuesta adopta una \textbf{arquitectura abierta y modular} (Spark/HDFS, Jenkins, MLflow, Prometheus/Grafana) para maximizar \textbf{control granular}, \textbf{trazabilidad} y \textbf{portabilidad} evitando \textit{lock‑in} de proveedor. La \textbf{automatización} de reentrenos y registro (Jenkins/MLflow) se orienta a auditoría y operación continua con baja intervención humana. Este posicionamiento recoge los criterios comparativos de la literatura y guía las decisiones técnicas implementadas en los objetivos OE1–OE3.

% ===========================
% ===========================
\section{OE1. Infraestructura escalable y monitorización continua}
\label{sec:oe1}

% ---------- Artículo científico (en la sección) ----------
\subsection*{Resumen}
Se implementó una infraestructura escalable, reproducible y \textit{cloud-ready} orientada a la evaluación continua de \textit{data drift} y al soporte operacional de pipelines MLOps. La solución combina Apache Spark sobre HDFS para cómputo/almacenamiento distribuido \citep{Zaharia2016,Meng2020}, contenedores para portabilidad y control de configuración \citep{Merkel2014,Patel2020} y un \textit{stack} de observabilidad (Prometheus/Grafana) que expone tanto métricas de plataforma como telemetría estadística del \textit{drift} \citep{Zhao2021}. Los resultados muestran operación reproducible con sobrecosto acotado, \textit{scraping} sub-minutal y trazabilidad integral mediante MLflow \citep{Chen2022}, habilitando la detección y respuesta automática vía Jenkins.

\subsection{Introducción}
El \textit{data drift} degrada el rendimiento de modelos en producción y requiere mecanismos continuos de detección y reacción \citep{Gama2014,Lu2019,Chawla2021}. OE1 busca sustentar los OE posteriores mediante una capa de infraestructura que garantice: (i) escalabilidad horizontal, (ii) observabilidad de extremo a extremo, (iii) portabilidad entre entornos on-prem y nube (alternables con Azure ML/Azure Monitor \citep{Berberi2025,Bhatt2025,AzureML2023,AzureMonitor2022}), y (iv) trazabilidad y versionado del ciclo de vida de modelos.

\subsection{Planteamiento y alcance}
El alcance de OE1 comprende: provisión de cómputo distribuido (Spark) y almacenamiento (HDFS), canalización de ingestas particionadas en formato \texttt{Parquet}, instrumentación de métricas operativas y estadísticas (KS, $\chi^2$, PSI), \textit{alerting} por umbrales y registro de ejecuciones/artefactos en MLflow, con orquestación de reentrenos mediante Jenkins. No se consideran servicios administrados ni autoscaling por orquestadores tipo Kubernetes; se prioriza simplicidad y control experimental con Docker~Compose.

\subsection{Fundamentos y criterios de diseño}

\paragraph{Síntesis de diseño (OE1).}
Para reducir narrativa y facilitar trazabilidad, la Tabla~\ref{tab:oe1-diseno} resume los componentes, su rol y referencias cruzadas. El diagrama operativo se presenta en la \autoref{fig:oe1-flujo}.

\begin{table}[htbp]
\centering
\caption{Resumen de diseño de OE1: componentes y rol.}
\label{tab:oe1-diseno}
\begin{tabular}{llll}
\toprule
\textbf{Componente} & \textbf{Rol principal} & \textbf{Métrica/artefacto} & \textbf{Ref.} \\
\midrule
Spark/HDFS & Cómputo/almacenamiento distribuido & Particiones Parquet & Cap.\,2; Sec.~\ref{sec:oe1} \\
Jenkins & CI/CD, reentrenamiento & Jobs de reentrenamiento & Sec.~\ref{sec:oe2} \\
MLflow & Trazabilidad de modelos & Runs, métricas, artefactos & Cap.\,2; Sec.~\ref{sec:oe2} \\
Prometheus/Grafana & Observabilidad & Exporters, dashboards & Cap.\,2; Sec.~\ref{sec:oe1} \\
Docker Compose & Orquestación local & Servicios reproducibles & Sec.~\ref{sec:oe1} \\
\bottomrule
\end{tabular}
\end{table}

\paragraph{Que se decidio.}
\begin{itemize}\setlength\itemsep{2pt}
  \item Base de computo/almacenamiento: \textbf{Spark sobre HDFS}.
  \item Trazabilidad del ciclo de vida: \textbf{MLflow}.
  \item Observabilidad: \textbf{Prometheus/Grafana}.
  \item Orquestacion de entorno: \textbf{Docker Compose} (en lugar de Kubernetes).
  \item Separacion de preocupaciones: procesamiento, almacenamiento, trazabilidad, CI/CD y observabilidad.
\end{itemize}

\paragraph{Por que es relevante.}
\begin{itemize}\setlength\itemsep{2pt}
  \item En consecuencia, se prioriza \textbf{reproducibilidad} y \textbf{portabilidad} con herramientas abiertas.
  \item Esto evidencia una \textbf{observabilidad efectiva} para auditoria y alerta temprana.
  \item La modularidad reduce acoplamiento y permite sustituciones puntuales sin re-trazar la arquitectura.
  \item La eleccion de Compose reduce complejidad sin sacrificar los objetivos de latencia/costo del estudio.
\end{itemize}

El diseño de la infraestructura se guió por los principios de \textit{reproducibilidad}, \textit{modularidad} y \textit{observabilidad}, que la literatura identifica como pilares para MLOps robusto en entornos Big Data \citep{Amershi2019,Berberi2025}. En consecuencia, la selección tecnológica privilegió componentes abiertos, de amplia adopción industrial y con trayectorias de estabilidad.

\textbf{Criterios tecnológicos (elección frente a alternativas).}
Se adoptó \textbf{Apache Spark sobre HDFS} como base de cómputo/almacenamiento distribuido por su madurez, soporte a cargas \textit{batch} y \textit{near real-time}, y ecosistema de librerías \citep{Zaharia2016,Meng2020}. Alternativas como Flink o Dask ofrecen ventajas específicas (p.\,ej., \textit{streaming} de baja latencia o programación en Python puro), pero introducen mayor complejidad de integración con herramientas de trazabilidad y menor disponibilidad de recetas operativas en contextos de MLOps académico \citep{Berberi2025}. 
Para \textbf{trazabilidad del ciclo de vida}, \textbf{MLflow} se prefirió sobre DVC o servicios cerrados (p.\,ej., W\&B) debido a su sencillez de despliegue, API estable y modelo de artefactos/experimentos adecuado para reproducibilidad local y portabilidad a nube \citep{Chen2022,Berberi2025}. 
En \textbf{observabilidad}, \textbf{Prometheus/Grafana} se eligió sobre ELK cuando el objetivo principal es \emph{métricas de series temporales} y \emph{alerting} ligero con bajo \textit{overhead}; ELK se reserva usualmente para análisis de logs a gran escala, menos crítico en este prototipo \citep{Zhao2021}.
Finalmente, \textbf{contenedores + Docker Compose} se priorizaron sobre Kubernetes por economía cognitiva y control fino del entorno experimental. Compose reduce la fricción de orquestación en laboratorios y facilita la replicación exacta, mientras que Kubernetes añade potencia a costa de complejidad (no necesaria para el volumen de este estudio) \citep{Merkel2014,Patel2020,Berberi2025}.

\textbf{Criterios arquitectónicos.}
Se aplicó \textit{separación de preocupaciones} entre procesamiento (Spark), almacenamiento (HDFS), trazabilidad (MLflow), automatización CI/CD (Jenkins) y observabilidad (Prometheus/Grafana). Esta modularidad reduce acoplamiento y habilita sustituciones puntuales (p.\,ej., Azure Monitor/Log Analytics en nube) sin re-trazar toda la arquitectura \citep{AzureMonitor2022,Berberi2025}. La exposición homogénea de \emph{endpoints} métricos facilita correlación de eventos y auditoría de experimentos.

\textbf{Criterios operativos.}
El sistema se diseñó para latencias sub-minutales de monitoreo, límites de muestreo por ventana y \textit{cooldown} tras activaciones, lo cual equilibra sensibilidad estadística del detector con costo operativo del reentrenamiento \citep{Kodakandla2024}. Este compromiso es consistente con recomendaciones recientes para \textit{pipelines} adaptativos que maximizan señal/ruido sin inducir \textit{flapping} \citep{Berberi2025}.


\subsection{Propuesta técnica: arquitectura y flujo operacional}
La arquitectura implementa un flujo mínimo viable (\textit{thin slice}) que conecta ingesta particionada en HDFS, cómputo distribuido en Spark, trazabilidad en MLflow y observabilidad con Prometheus/Grafana, dejando a Jenkins como la única vía de activación de reentrenos. Este recorte deliberado evita complejidad innecesaria (p.\,ej., malla de servicios o Kubernetes) y maximiza la inspeccionabilidad del experimento, criterio preferible en validaciones de MLOps académico \citep{Berberi2025,Kodakandla2024}. La Figura~\ref{fig:oe1-flujo} sintetiza el flujo.

La Figura~\ref{fig:oe1-flujo} sintetiza el flujo operativo desde la ingesta orquestada hasta el monitoreo. La canalización produce \texttt{Parquet} particionado (\texttt{dt}/\texttt{hour}/\texttt{account\_type}) en HDFS; Prometheus \textit{scrapea} métricas de Spark, MLflow, Jenkins y del \textit{exporter} específico de \textit{drift} (OE2), consolidándolas en paneles de Grafana para evaluación y \textit{alerting}.


\begin{figure}[htbp]
\centering
\caption{Flujo de infraestructura y observabilidad de extremo a extremo (OE1).}
\label{fig:oe1-flujo}

\begin{tikzpicture}[
  node distance=10mm and 15mm,
  every node/.style={font=\footnotesize},
  >=Latex,
  box/.style={draw, rounded corners=2pt, fill=gray!8, align=center,
              minimum width=20mm, minimum height=7mm, text width=22mm, inner sep=2pt},
  obs/.style={draw, rounded corners=2pt, fill=blue!5, align=center,
              minimum width=20mm, minimum height=7mm, text width=22mm, inner sep=2pt},
  db/.style={cylinder, draw, shape border rotate=90, aspect=0.25,
             minimum height=9mm, minimum width=9mm,
             cylinder uses custom fill, cylinder end fill=gray!20,
             cylinder body fill=gray!8, align=center, text width=22mm},
  arrow/.style={-Latex, line width=0.4pt},
  note/.style={font=\scriptsize, fill=white, inner sep=1pt, align=center}
]

% --- Nodos principales ---
\node[db] (hdfs) {HDFS};
\node[box, right=of hdfs] (spark) {Apache\\Spark};
\node[box, above=of spark, yshift=2mm] (air) {Airflow\\(DAG)};
\node[box, below=of spark, yshift=-2mm] (jenk) {Jenkins};
\node[box, right=of spark] (mlf) {MLflow};

% --- Caja de infraestructura ---
\node[draw, rounded corners=3pt, inner sep=6mm, fit=(hdfs)(spark)(air)(jenk)(mlf),
      label={[align=center]above:
      Infraestructura escalable\\(Docker/Compose | Cloud-ready)}] (infra) {};

% --- Nodos de observabilidad ---
\node[obs, below=18mm of spark] (prom) {Prometheus};
\node[obs, below=of prom] (graf) {Grafana};

% --- Conexiones principales ---
\draw[arrow] (air) -- node[note, right=1mm, yshift=1mm]
  {DAG:\\bank\_data\_generation\_dag} (spark);

\draw[arrow] (spark) -- node[note, above=0.5mm] {write Parquet} (hdfs);

\draw[arrow] (spark) -- (prom);
\draw[arrow] (jenk)  -- (prom);
\draw[arrow] (mlf)   -- (prom);
\draw[arrow] (prom)  -- (graf);

% --- Etiquetas scrape ---
\node[note, right=1mm, yshift=0.5mm] at ($(spark)!0.5!(prom)$) {scrape/metrics};
\node[note, right=1mm, yshift=0.5mm] at ($(jenk)!0.5!(prom)$) {scrape/metrics};
\node[note, right=1mm, yshift=0.5mm] at ($(mlf)!0.5!(prom)$) {scrape/metrics};

\end{tikzpicture}
\end{figure}


\paragraph{Diseño experimental.}
Se empleó un entorno pseudo-distribuido reproducible con Docker Compose que replica interacciones esenciales de un sistema MLOps. El \emph{stack} incluye HDFS (persistencia), Spark (procesamiento), MLflow (experimentos), Jenkins (CI/CD), Prometheus/Grafana (observabilidad) y Airflow (ingesta/ETL). Esta selección prioriza control experimental y transparencia de variables, alineada con buenas prácticas de evaluación en MLOps \citep{Amershi2019,Berberi2025}. Los datos sintéticos particionados permiten inducir y medir \textit{drift} bajo protocolos controlados \citep{Lu2019,Chawla2021}.

El despliegue incluye: (i) un clúster \textbf{HDFS} compuesto por \textit{NameNode} y \textit{DataNode} para el almacenamiento distribuido; (ii) un clúster \textbf{Spark} con roles de \textit{master} y \textit{worker} para el procesamiento paralelo de datos; (iii) un servidor de \textbf{MLflow} para el seguimiento de experimentos y versionado de artefactos; (iv) un servidor \textbf{Jenkins} para la automatización de flujos CI/CD; (v) el conjunto \textbf{Prometheus/Grafana} para la observabilidad, recolección de métricas y visualización de alertas; y (vi) un servicio \textbf{Airflow} encargado de la orquestación ETL y la ingesta programada de datos.  
Los datos sintéticos —de naturaleza bancaria— se generan en lotes particionados, lo que permite simular de manera controlada escenarios con y sin \textit{drift} estadístico, conforme a los protocolos experimentales propuestos por \citet{Agarwal2023} y \citet{Chatterjee2023}.

\paragraph{Instrumentación.}  
Para la observación continua del comportamiento del sistema, se desarrolló un \textit{exporter} HTTP que expone un conjunto de métricas en formato legible por Prometheus. Dichas métricas incluyen: (a) los valores $p$ por columna obtenidos de las pruebas KS y $\chi^2$; (b) el índice de estabilidad poblacional (PSI) sobre el \textit{score} del modelo; (c) la razón de instancias positivas estimada en cada ventana de monitoreo; y (d) un contador acumulativo de disparos de reentrenamiento.  
Estas señales son recolectadas y visualizadas en \textbf{Grafana} mediante paneles que muestran la evolución temporal de la latencia, el rendimiento (\textit{throughput}), el uso de CPU y memoria, así como indicadores resumidos (\textit{single-stats}) que permiten detectar y contextualizar la ocurrencia del \textit{drift}.

\paragraph{Métricas y KPIs.}  
El desempeño del sistema se evalúa a partir de indicadores tanto técnicos como operativos. Entre los principales \textbf{KPIs} definidos se encuentran:  
(i) el \textbf{TTFD} (\textit{time-to-first-detection}), que mide el tiempo transcurrido desde la inyección del \textit{drift} hasta su detección estadística;  
(ii) el \textbf{overhead} introducido por los procesos de monitoreo, en términos de consumo de CPU, RAM y operaciones de I/O;  
(iii) la \textbf{latencia de \textit{scrape}} (percentil p99), que refleja el retardo máximo aceptable entre la exposición y la lectura de métricas;  
(iv) la \textbf{disponibilidad} del \textit{exporter}, entendida como la proporción de intervalos en que el servicio responde exitosamente;  
(v) el \textbf{MTTR} (\textit{mean time to recovery}), que cuantifica el tiempo promedio necesario para restaurar el estado “sin drift” después de un reentrenamiento; y  
(vi) la \textbf{trazabilidad}, medida por el número de ejecuciones y artefactos registrados en MLflow, lo que asegura reproducibilidad y seguimiento histórico de las intervenciones.

\subsection{Formulación estadística de las señales de \textit{drift}}

\paragraph{Kolmogorov–Smirnov (KS).}  
Para las variables numéricas \(X\), se contrasta la hipótesis nula \(H_0:\,F_{\text{ref}}(x)=F_{\text{rec}}(x)\) mediante la estadística \(D=\sup_x |F_{\text{ref}}(x)-F_{\text{rec}}(x)|\). Se genera una alerta cuando el valor \(p<\alpha\), con un umbral típico \(\alpha=0.01\). Este test permite detectar desviaciones significativas entre las distribuciones acumuladas de referencia y recientes, siendo especialmente sensible a desplazamientos de media o cambios en la forma de la distribución.

\paragraph{Chi-cuadrado (\(\chi^2\)).}  
Para las variables categóricas, se construye una tabla de contingencia que compara las frecuencias observadas en los datos recientes frente a las esperadas bajo la distribución de referencia. La hipótesis de independencia se rechaza cuando \(p<\alpha\), señalando un cambio estadísticamente significativo en la composición de las categorías.

\paragraph{Population Stability Index (PSI).}  
El PSI mide la divergencia entre dos distribuciones discretizadas en \(K\) intervalos o \textit{bins}, con proporciones \(\{r_k\}\) para la referencia y \(\{c_k\}\) para la corriente:
\[
\mathrm{PSI} = \sum_{k=1}^{K} (r_k - c_k) \ln \frac{r_k}{c_k}.
\]
Se aplican recortes \(\epsilon\) para evitar el cálculo de \(\log(0)\) y los \textit{bins} se construyen según cuantiles de la distribución de referencia.  
En la práctica, se consideran umbrales de interpretación: \(\mathrm{PSI}>0.2\) indica una alerta de cambio moderado y \(\mathrm{PSI}>0.3\) sugiere un \textit{drift} severo \citep{Chawla2021,Nguyen2023}.  
A diferencia de las pruebas KS o \(\chi^2\), el PSI proporciona una métrica continua y fácilmente interpretable que complementa el análisis de estabilidad poblacional.


\subsection{Implementación e instrumentación}

\paragraph{Que se hizo.}
\begin{itemize}\setlength\itemsep{2pt}
  \item Servicio continuo \texttt{drift\_watch.py} en PySpark para evaluar KS/$\chi^2$/PSI por ventana.
  \item Exportacion de metricas para Prometheus y panelizacion en Grafana.
  \item Disparo de reentrenamiento via Jenkins bajo condiciones de umbral.
\end{itemize}

\paragraph{Por que es relevante.}
\begin{itemize}\setlength\itemsep{2pt}
  \item En consecuencia, se cierra el ciclo detectar $\rightarrow$ actuar $\rightarrow$ auditar con trazabilidad en MLflow.
  \item Esto evidencia integracion operativa entre datos, modelo y observabilidad con baja latencia.
\end{itemize}

\paragraph{Que se hizo.}
\begin{itemize}\setlength\itemsep{2pt}
  \item Despliegue coordinado con \textbf{Docker Compose} para HDFS, Spark, MLflow, Prometheus/Grafana y Jenkins.
  \item Ingesta periodica con \textbf{Airflow} $\rightarrow$ \textbf{PySpark} $\rightarrow$ \textbf{HDFS} (\texttt{Parquet} particionado).
  \item Observabilidad con \textbf{Prometheus/Grafana} (\textit{scraping} de Spark, MLflow, Jenkins y \textit{exporter} de \textit{drift}).
  \item Automatizacion de reentrenos en \textbf{Jenkins} con politicas \textit{edge-triggered} y \textit{cooldown}.
\end{itemize}

\paragraph{Por que es relevante.}
\begin{itemize}\setlength\itemsep{2pt}
  \item En consecuencia, se garantiza \textbf{reproducibilidad}, \textbf{aislamiento} por servicio y \textbf{trazabilidad end-to-end}.
  \item Esto evidencia que la arquitectura reduce \textbf{latencias operativas} y facilita \textbf{auditoria} de cada ciclo.
  \item La tendencia observada sugiere \textbf{estabilidad operativa} al mitigar \textit{flapping} mediante \textit{cooldown}.
\end{itemize}

\paragraph{Despliegue.}  
El sistema se despliega mediante \texttt{docker-compose}, que instancia de forma coordinada todos los servicios del entorno, definiendo variables de entorno para las URLs de HDFS y Spark, los puertos de los \textit{exporters} y las URIs correspondientes a MLflow y Jenkins.  
La ingesta programada se realiza a través de \textbf{Airflow}, el cual ejecuta de manera periódica un contenedor PySpark encargado de procesar los datos sintéticos y escribirlos en formato \texttt{Parquet} dentro del sistema distribuido, utilizando particiones por lotes para facilitar su seguimiento y análisis.  
Esta configuración permite mantener la reproducibilidad del entorno, la independencia de cada servicio y la trazabilidad completa del flujo de datos desde la generación hasta el monitoreo.

\paragraph{Observabilidad.}  
El componente de observabilidad se fundamenta en la integración entre \textbf{Prometheus} y \textbf{Grafana}.  
Prometheus se encarga de realizar \textit{scraping} periódico sobre los \textit{endpoints} expuestos por Spark, MLflow, Jenkins y el \textit{exporter} de \textit{drift}, recolectando métricas en formato de series temporales.  
Estas métricas incluyen tanto indicadores de rendimiento de los servicios como señales estadísticas del comportamiento de los datos. Entre las principales se encuentran:
\begin{itemize}\setlength\itemsep{2pt}
  \item \texttt{drift\_score\_psi\{model\}}: valor del índice $\mathrm{PSI}$ correspondiente al \textit{score} del modelo.
  \item \texttt{pvalue\{col\}} y \texttt{drift\_detected\{col\}}: valores $p$ y banderas binarias por columna, derivados de las pruebas KS y $\chi^2$.
  \item \texttt{predicted\_positive\_ratio\{model\}}: proporción de instancias positivas estimada en la predicción reciente.
  \item \texttt{jenkins\_retrain\_triggers\_total\{model\}}: contador acumulativo de ejecuciones de reentrenamiento disparadas por el sistema.
\end{itemize}
Estas métricas se visualizan en \textbf{Grafana} mediante paneles dinámicos que permiten interpretar la evolución temporal del \textit{drift}, la carga del sistema y los indicadores de desempeño general.

\paragraph{Automatización.}  
El proceso de automatización se implementa a través de \textbf{Jenkins}, que orquesta el reentrenamiento del modelo cuando las pruebas estadísticas —KS, $\chi^2$ o PSI— superan los umbrales configurados.  
Para evitar activaciones redundantes o erráticas, se aplica una política de tipo \textit{edge-triggered} complementada con un período de \textit{cooldown} y ciclos de limpieza de estado, lo cual impide re-disparos consecutivos por fluctuaciones menores en los datos.  
Este esquema de control contribuye a mantener la estabilidad operativa del pipeline y a reducir la carga innecesaria sobre los recursos computacionales, en línea con las recomendaciones de \citet{Kim2018}, \citet{Kahn2020} y \citet{RodriguezSimmhan2023}.


\subsection{Resultados y análisis}

\paragraph{Que se observo.}
\begin{itemize}\setlength\itemsep{2pt}
  \item Detecciones con $p<\alpha$ y/o PSI $>\tau$ activaron alertas y reentrenos.
  \item Recuperacion del desempeno (F1) post-reentrenamiento registrada en MLflow.
  \item Estabilidad operativa lograda con edge-triggered + cooldown + rachas limpias.
\end{itemize}

\paragraph{Por que es relevante.}
\begin{itemize}\setlength\itemsep{2pt}
  \item En consecuencia, el detector cumple objetivos de sensibilidad y tiempo de respuesta.
  \item Esto evidencia que el pipeline operativo se sostiene de forma autonoma y auditable.
  \item La tendencia observada sugiere menor riesgo de flapping con los parametros actuales.
\end{itemize}

\paragraph{Que se observo.}
\begin{itemize}\setlength\itemsep{2pt}
  \item TTFD reducido y activaciones correctas ante KS/$\chi^2$/PSI sobre umbral.
  \item Reentrenamientos automaticos ejecutados por Jenkins con recuperacion de F1 en MLflow.
  \item Correlacion en Grafana entre picos de \textit{drift}, disparos y mejora posterior.
\end{itemize}

\paragraph{Por que es relevante.}
\begin{itemize}\setlength\itemsep{2pt}
  \item En consecuencia, se confirma la capacidad de reaccion del sistema sin intervencion humana.
  \item Esto evidencia que la infraestructura soporta un ciclo detectar $\rightarrow$ responder $\rightarrow$ verificar reproducible.
  \item La tendencia observada sugiere que la observabilidad unificada acelera diagnostico y prevencion de regresiones.
\end{itemize}

En el entorno controlado de pruebas, la infraestructura implementada demostró un desempeño estable y coherente con los criterios de diseño planteados.  
En primer lugar, se alcanzó una \textbf{operación reproducible}, sustentada en un aprovisionamiento conservador de recursos y en la consistencia del despliegue mediante \texttt{docker-compose}, lo que permitió replicar de forma fiable cada ciclo experimental sin interferencias entre servicios.  

En segundo lugar, se logró una \textbf{observabilidad unificada} tanto del comportamiento del \textit{data drift} como de los recursos de la plataforma.  
La integración entre Prometheus y Grafana facilitó la correlación entre eventos de degradación estadística y métricas operativas, permitiendo diagnosticar rápidamente los efectos del \textit{drift} en el desempeño del sistema.  

En tercer lugar, las pruebas mostraron \textbf{latencias de \textit{scrape}} compatibles con ventanas sub-minutales, garantizando una actualización casi continua de las métricas de monitoreo sin impacto perceptible en el rendimiento del pipeline.  
Asimismo, la integración de MLflow permitió una \textbf{trazabilidad completa} de todas las ejecuciones, registrando parámetros, métricas y artefactos de cada reentrenamiento con un alto nivel de detalle y auditabilidad.  

Finalmente, la aplicación de la política \textit{edge-triggered} combinada con un período de \textit{cooldown} resultó efectiva para mitigar reactivaciones espurias en escenarios de \textit{drift} sostenido, reduciendo significativamente la frecuencia de reentrenamientos innecesarios y, con ello, el consumo de recursos.  

Estos resultados consolidan la validez de la infraestructura como base experimental sólida para los objetivos siguientes: \textbf{OE2}, orientado a la detección y monitoreo del \textit{drift}, y \textbf{OE3}, enfocado en el reentrenamiento y validación del modelo bajo condiciones dinámicas.  
El sistema obtenido se caracteriza, por tanto, por su capacidad de medición, control y trazabilidad, aspectos fundamentales para garantizar la reproducibilidad y evaluabilidad del ciclo MLOps propuesto.


\subsection{Discusión}
\paragraph{Cierre interpretativo.} Los resultados anteriores muestran que la combinación KS/$\chi^2$/PSI, integrada con observabilidad y reentrenamiento automático, no solo detecta desviaciones con oportunidad, sino que además reduce el TTFD y habilita la recuperación del desempeño (F1) al cerrar el ciclo detectar $\rightarrow$ reentrenar $\rightarrow$ verificar.
Este comportamiento coincide con los criterios de diseño planteados y con prácticas recomendadas en MLOps; sin embargo, la sensibilidad observada exige seguir equilibrando especificidad mediante la calibración de umbrales y políticas de cooldown.
Una arquitectura modular desacoplada (Spark/HDFS, MLflow, Jenkins, Prometheus/Grafana) facilita sustituciones tecnológicas (p.\,ej., Azure Monitor/Log Analytics en la nube \citep{Bhatt2025,AzureMonitor2022}) y reduce deuda técnica \citep{Sculley2015}. La exposición explícita de señales de \textit{drift} como métricas de \textit{observabilidad} conecta natural y operativamente la ciencia de datos con SRE/DevOps, alineando prácticas CI/CD con MLOps \citep{Amershi2019,Lwakatare2020,Kapoor2023}.

\subsection{Amenazas a la validez}

El experimento, aunque controlado y reproducible, presenta diversas limitaciones que pueden afectar la interpretación y generalización de los resultados. Estas amenazas se analizan en cuatro dimensiones clásicas: interna, externa, de constructo y de conclusión.

\textbf{Validez interna.}  
La principal fuente de riesgo proviene de la \textbf{sensibilidad de los umbrales estadísticos} (\(\alpha\), PSI) al tamaño de muestra utilizado en cada ventana de monitoreo.  
En escenarios con volúmenes reducidos de datos, pequeñas fluctuaciones aleatorias pueden provocar resultados estadísticamente significativos sin reflejar un cambio real en la distribución.  
Asimismo, los \textbf{sesgos de muestreo} y la \textbf{baja cardinalidad} en ciertas variables categóricas pueden alterar la estimación de frecuencias esperadas, afectando la confiabilidad de las pruebas \(\chi^2\).  
Estos factores podrían inducir falsos positivos o negativos en la detección del \textit{drift}, limitando la validez causal de las inferencias obtenidas.

\textbf{Validez externa.}  
Dado que el sistema se ejecuta en un entorno \textit{pseudo-distribuido}, los resultados pueden no extrapolarse directamente a entornos de producción con cargas multicliente y variabilidad temporal.  
En un escenario real, los \textbf{patrones de acceso a HDFS}, la competencia por recursos entre servicios concurrentes y la heterogeneidad de los flujos de datos pueden modificar el comportamiento observado en cuanto a latencias, tiempos de respuesta y consumo de recursos.  
Por ello, los hallazgos deben interpretarse como representativos de un entorno controlado, más que como predicciones exactas del rendimiento en producción.

\textbf{Validez de constructo.}  
El alcance experimental se centró en el \textbf{drift de entrada} (\textit{input drift}) y en el \textbf{drift del score}, sin incluir de forma explícita el \textit{concept drift}, es decir, los cambios en la relación entre las variables predictoras y la variable objetivo.  
Esta cobertura parcial de métricas implica que el sistema no captura todos los tipos posibles de deriva, especialmente aquellos vinculados a modificaciones estructurales del modelo o del contexto semántico de los datos.  
En consecuencia, el constructo de “detección de \textit{data drift}” abordado es una aproximación operativa, pero no exhaustiva, al fenómeno real.

\textbf{Validez de conclusión.}  
Finalmente, persiste un \textbf{riesgo de falsos positivos y falsos negativos} debido a factores no controlados, como la estacionalidad en los datos o la presencia de \textit{concept shift} no observado durante los experimentos \citep{Widmer1996,Kolter2007,Sethi2017}.  
Tales fluctuaciones pueden inducir alertas espurias o retrasar la detección de cambios genuinos, afectando la consistencia de las conclusiones estadísticas.  
Si bien el diseño experimental mitiga parcialmente estos riesgos mediante ventanas móviles y políticas de \textit{cooldown}, se reconoce que la robustez final del sistema depende de su validación continua bajo condiciones de carga más diversas y prolongadas.


%\subsection{Conclusiones parciales}
%OE1 alcanzó una infraestructura escalable, observable y trazable, con telemetría de %\textit{drift} integrada y \textit{hooks} de automatización para reentrenos. Esta base %satisface los criterios de diseño y habilita la evaluación experimental de los OE %siguientes con garantías de reproducibilidad y auditoría.

% ---------- Fin del “artículo” dentro de OE1 ----------
% ===========================
% ===========================
\section{OE2. Sistema de detección automática de \textit{data drift} y reentrenamiento}
\label{sec:oe2}

\subsection*{Resumen}
Se implementó un servicio de vigilancia de \textit{data drift} que integra contrastes univariantes (Kolmogorov–Smirnov para atributos numéricos y $\chi^2$ para categóricos) junto con el \textit{Population Stability Index} (PSI) aplicado a \textit{scores} y/o \textit{features}. El monitoreo se realiza sobre ventanas deslizantes y la decisión se toma mediante una lógica compuesta orientada a sensibilidad: se declara desviación si al menos una prueba resulta significativa ($p<\alpha$) o si el PSI excede un umbral operativo. Para evitar reactivaciones espurias, el disparo es por flanco (\textit{edge-triggered}) y se aplica un intervalo de enfriamiento (\textit{cooldown}).

Cuando se cumple la condición ($\alpha=0.01$, $\mathrm{PSI}>0.2$ en la configuración base), el sistema orquesta automáticamente un reentrenamiento en Jenkins y registra parámetros, métricas y artefactos en MLflow, garantizando trazabilidad extremo a extremo y mínima intervención humana \citep{Lu2019,Chawla2021,Nguyen2023,Chen2022}.


\subsection{Planteamiento y alcance}

El segundo objetivo específico (OE2) tiene como finalidad el desarrollo de un \textbf{mecanismo operativo de detección de \textit{data drift}} que funcione como componente central de supervisión y control dentro del ecosistema MLOps.  
Su función es permitir que el sistema identifique y gestione de forma autónoma los cambios estadísticamente relevantes en la distribución de los datos, garantizando la estabilidad del modelo en producción y la continuidad del servicio predictivo.

Desde una perspectiva funcional, el detector debe ser capaz de:  
(i) \textbf{monitorear de forma continua} los flujos de datos entrantes y compararlos con una referencia establecida en el repositorio histórico;  
(ii) \textbf{exponer indicadores estadísticos y de estado} mediante métricas recolectadas por Prometheus y visualizadas en Grafana, posibilitando el análisis temporal y la trazabilidad de los eventos detectados; y  
(iii) \textbf{disparar procesos de reentrenamiento} gestionados por Jenkins cuando se verifiquen condiciones de desviación superiores a los umbrales definidos, incorporando mecanismos de amortiguamiento (\textit{cooldown}) que prevengan reactivaciones innecesarias o inestables.

El alcance de este objetivo se circunscribe a la detección de \textit{drift} basada en \textbf{pruebas estadísticas univariadas}, aplicando Kolmogorov–Smirnov para atributos numéricos, $\chi^2$ para categóricos y el \textbf{Population Stability Index (PSI)} para \textit{scores} o variables derivadas.  
No se considera en esta fase la detección de \textit{concept drift} ni el uso de enfoques multivariados o adaptativos, los cuales se proponen como líneas de trabajo complementarias para futuras iteraciones del sistema.

La toma de decisión se rige por una \textbf{lógica de histéresis}, que combina umbrales de alerta con intervalos de \textit{cooldown}, evitando oscilaciones por fluctuaciones menores o ruido estadístico.  
Con ello, OE2 consolida el módulo de diagnóstico y vigilancia del flujo MLOps, actuando como enlace funcional entre la infraestructura de ejecución continua definida en OE1 y el ciclo de reentrenamiento y validación automatizado desarrollado en OE3.

\subsection{Modelo y algoritmo propuesto}

\paragraph{Síntesis de diseño (OE2).}
La Tabla~\ref{tab:oe2-detectores} resume la selección de detectores y umbrales operativos, en contraste con el cuadro teórico del Cap.\,2 (\autoref{subsec:cuadro-detectores}).

\begin{table}[htbp]
\centering
\caption{Detectores y políticas de activación en OE2 (resumen).}
\label{tab:oe2-detectores}
\begin{tabular}{lllll}
\toprule
\textbf{Detector} & \textbf{Variable} & \textbf{Umbral/Regla} & \textbf{Costo (rel.)} & \textbf{Ref.} \\
\midrule
KS (univar.) & Continua & $p<\alpha$ (\(\alpha=\)\,0.01) & Medio & Cap.\,2 \\
$\chi^2$ & Categórica & $p<\alpha$ (\(\alpha=\)\,0.01) & Bajo & Cap.\,2 \\
PSI & Num./cat. (binning) & Alerta si PSI $\ge 0.2$ & Bajo & Cap.\,2 \\
\bottomrule
\end{tabular}
\end{table}

\paragraph{Ventanas y pruebas estadísticas.}  
El diseño del detector se fundamenta en un esquema de \textbf{evaluación por ventanas deslizantes}, en el cual, durante cada ciclo de ejecución, se comparan dos muestras representativas: una \emph{referencia}, que conserva la distribución base del sistema, y otra \emph{reciente}, obtenida del flujo continuo de datos.  
El contraste entre ambas permite detectar alteraciones en la estructura estadística del conjunto de entrada y en las variables derivadas del modelo.

Para las variables numéricas se utiliza la prueba de \textbf{Kolmogorov–Smirnov (KS)}, adecuada para identificar desplazamientos o distorsiones en la distribución acumulada de los valores.  
Las variables categóricas se evalúan mediante la prueba de \textbf{chi-cuadrado ($\chi^2$)}, que compara las frecuencias observadas y esperadas bajo la hipótesis de estabilidad, revelando posibles cambios en la composición de categorías.  
Adicionalmente, sobre la variable de salida del modelo (o alguna característica asociada) se calcula el \textbf{Population Stability Index (PSI)}, métrica que cuantifica la divergencia entre distribuciones discretizadas a partir de los cuantiles de referencia.

El uso combinado de KS, $\chi^2$ y PSI permite una observación integral de la estabilidad del modelo, abarcando tanto el comportamiento de las variables de entrada como las salidas o \textit{scores}.  
Esta integración garantiza una detección temprana y robusta de desviaciones estadísticas, ofreciendo una representación más completa del estado dinámico del sistema.

\paragraph{Fusión de evidencias.}  
Los resultados obtenidos de las diferentes pruebas estadísticas se integran mediante una \textbf{regla de decisión de tipo OR}, en la cual se considera la existencia de \textit{drift} si al menos una variable presenta un valor $p<\alpha$ o si el índice $\mathrm{PSI}$ excede el umbral definido $\tau$.  
Esta estrategia busca maximizar la \textbf{sensibilidad} del detector, priorizando la detección temprana de alteraciones incluso cuando el cambio afecta solo un subconjunto limitado de variables.  
La configuración de los parámetros $\alpha$ y $\tau$ puede ajustarse de manera empírica durante las pruebas de calibración, con el fin de alcanzar un \textbf{equilibrio adecuado entre falsos positivos y omisiones}, adaptando la respuesta del sistema a la variabilidad característica del dominio de datos.

\paragraph{Control de estabilidad.}
Para evitar disparos repetidos o activaciones espurias causadas por oscilaciones puntuales, el sistema incorpora un esquema de control con tres componentes:  
(i) una política \textbf{\textit{edge-triggered}} que emite la alerta únicamente ante el cambio de estado —es decir, en el flanco de subida del evento de \textit{drift}—;  
(ii) un \textbf{intervalo de \textit{cooldown}} que inhibe nuevos disparos durante un período fijo posterior a la activación, estabilizando el comportamiento del orquestador; y  
(iii) una \textbf{racha mínima de ciclos “limpios”} como condición de rearme, que exige varias iteraciones consecutivas sin evidencia de \textit{drift} antes de habilitar nuevamente el detector.  
Con este diseño se atenúan bucles de reentrenamiento innecesarios y se preserva la robustez del sistema frente a ruido y variaciones transitorias.

\begin{figure}[!htbp]
\captionsetup{skip=6pt}
\centering
\caption{Interacción entre detección estadística, orquestación y entrenamiento (OE2).}
\label{fig:oe2-flujo}

\adjustbox{max width=.95\linewidth, max height=.68\textheight}{
\begin{tikzpicture}[
    node distance=4mm and 7mm,
    scale=0.92,
    every node/.style={font=\scriptsize, transform shape},
    >=Latex
]
  % Estilos compactos
  \tikzset{
    box/.style={draw, rounded corners=2pt, fill=gray!8, align=center,
                minimum width=12mm, minimum height=5mm, text width=20mm, inner sep=1pt},
    proc/.style={box},
    obs/.style={box, fill=blue!5, text width=20mm},
    db/.style={cylinder, draw, shape border rotate=90, aspect=0.22,
               minimum height=7mm, minimum width=7mm,
               cylinder uses custom fill, cylinder end fill=gray!20,
               cylinder body fill=gray!8, align=center, text width=20mm, inner sep=0.6pt},
    decision/.style={diamond, draw, aspect=1.8, align=center, inner sep=0.6pt,
                     fill=gray!6, text width=15mm},
    arrow/.style={-Latex, line width=0.4pt},
    note/.style={font=\scriptsize, inner sep=0.5pt, fill=white, align=center}
  }

  % Entradas
  \node[db]                    (ref) {Distribución\\de referencia};
  \node[db, right=18mm of ref] (rec) {Datos recientes\\(HDFS)};

  % Comparación y pruebas
  \node[proc, below=5mm of ($(ref)!0.5!(rec)$)] (cmp) {Comparación\\(ventanas móviles)};
  \node[proc, below left=4mm and -1mm of cmp]  (ks)  {KS\\(numéricas)};
  \node[proc, below=4mm of cmp]                (chi) {$\chi^2$\\(categóricas)};
  \node[proc, below right=4mm and -1mm of cmp] (psi) {PSI\\(score/features)};

  % Decisión
  \node[decision, below=6mm of chi] (dec) {¿Drift?};

  % Orquestación
  \node[proc, below=5mm of dec, xshift=-12mm] (mon) {Monitoreo\\continuo};
  \node[proc, below=5mm of dec, xshift=+12mm] (trg) {Trigger\\Jenkins};
  \node[proc, below=5mm of trg]  (job) {Reentrenar\\(Jenkins Job)};
  \node[proc, below=5mm of job]  (fit) {Entrenamiento\\+ Validación};
  \node[proc, below=5mm of fit, xshift=-12mm] (mlf) {MLflow\\(runs/artefactos)};
  \node[obs,  below=5mm of fit, xshift=+12mm] (prom) {Prometheus\\/ Grafana};

  % Flechas
  \draw[arrow] (ref) -- (cmp);
  \draw[arrow] (rec) -- (cmp);
  \draw[arrow] (cmp) -- (ks);
  \draw[arrow] (cmp) -- (chi);
  \draw[arrow] (cmp) -- (psi);
  \draw[arrow] (ks)  -- node[note, left]{p\,$<\,\alpha$} (dec);
  \draw[arrow] (chi) -- node[note, left]{p\,$<\,\alpha$} (dec);
  \draw[arrow] (psi) -- node[note, right]{PSI\,$>\,$umbral} (dec);
  \draw[arrow] (dec) -- ++(-6mm,-4mm) |- (mon);
  \draw[arrow] (dec) -- ++(+6mm,-4mm) |- (trg);
  \draw[arrow] (trg) -- (job);
  \draw[arrow] (job) -- (fit);
  \draw[arrow] (fit) -- (mlf);
  \draw[arrow] (fit) -- (prom);
  \draw[arrow] (mlf) -- (prom);
\end{tikzpicture}
}
\vspace{-4pt}
\end{figure}

\subsection{Justificación de diseño y alternativas consideradas}

El diseño favorece sensibilidad temprana, interpretabilidad operacional y estabilidad del disparo con bajo costo computacional. A continuación se justifican las principales elecciones.

\paragraph{Univariado (KS/$\chi^2$/PSI) vs.\ multivariado (MMD, Energy, K-Sample).}
Se prioriza un enfoque \emph{univariante} por su interpretabilidad (atributo a atributo) y su costo lineal en el número de columnas. Los métodos \emph{multivariados} (p.\,ej., Maximum Mean Discrepancy, Energy Distance, K-Sample tests) capturan dependencias cruzadas y pueden detectar cambios conjuntos sutiles; no obstante, incrementan complejidad de calibración (kernel/embeddings, tamaños de muestra) y elevan el costo por ventana, lo que reduce frecuencia de muestreo y eleva latencia. En un prototipo orientado a \emph{detección+acción} con ventanas sub-minutales, la trazabilidad atributo-específica y el bajo overhead se consideraron dominantes. La extensión multivariada queda como línea de mejora (véase Discusión).

\paragraph{Regla de fusión OR vs.\ voto ponderado o lógica bayesiana.}
La regla OR maximiza \emph{sensibilidad} ante cambios localizados y simplifica la calibración (dos umbrales globales $\alpha$, $\tau$). Es preferida cuando el costo de una omisión (FN) es mayor que el de una falsa alarma (FP) y existe \emph{cooldown}. Alternativas como voto ponderado o combinación bayesiana podrían mejorar especificidad, pero requieren estimar pesos/confiabilidades por variable y podrían enmascarar señales débiles en atributos críticos.

\paragraph{Edge-triggered + cooldown vs.\ nivel continuo (level-triggered).}
El disparo por flanco evita \emph{flapping} bajo ruido y oscilaciones cerca del umbral. El \emph{cooldown} fija un horizonte de estabilidad post-evento y la racha limpia (\emph{clear streak}) establece rearme explícito. El esquema level-triggered es más reactivo pero propenso a re-disparos consecutivos cuando la señal cruza repetidamente el umbral.

\paragraph{PSI en \emph{score} vs.\ PSI en todas las columnas.}
PSI sobre \emph{score} ofrece un indicador sintético, fácil de auditar y directamente vinculado al desempeño. Aplicarlo a todas las columnas incrementa cobertura pero diluye la interpretabilidad y multiplica el riesgo de FP; por ello se limita a \emph{score}/features derivadas relevantes, mientras KS/$\chi^2$ cubren el resto.

\begin{table}[H]
\centering
\caption{Resumen de alternativas y razón de la elección}
\label{tab:oe2-alternativas}
\begin{tabular}{p{3.2cm}p{5.1cm}p{5.1cm}}
\toprule
\textbf{Decisión} & \textbf{Alternativa} & \textbf{Razón de elección} \\
\midrule
Tests univariados & MMD / Energy multivariados & Menor costo por ventana; trazabilidad por atributo; latencias sub-minutales. \\
Regla OR & Voto ponderado / combinación bayesiana & Máxima sensibilidad con calibración simple ($\alpha,\tau$) + \emph{cooldown}. \\
Edge-triggered & Level-triggered continuo & Estabilidad del disparo; evita re-activaciones espurias. \\
PSI en score & PSI en todas las columnas & Indicador global interpretable; menor FP agregado. \\
\bottomrule
\end{tabular}
\end{table}


\subsection{Formulación estadística}

El detector de \textit{data drift} implementa un conjunto de pruebas estadísticas complementarias orientadas a evaluar, de manera \textbf{univariada}, la estabilidad entre la distribución histórica de referencia y la observada en los datos más recientes.  
Cada tipo de variable —numérica, categórica o derivada del \textit{score} del modelo— se analiza con una métrica específica, seleccionada según su naturaleza y el tipo de desviación que se desea detectar.

\paragraph{Variables numéricas (Kolmogorov–Smirnov).}  
En las columnas numéricas se utiliza la prueba de \textbf{Kolmogorov–Smirnov (KS)}, la cual compara las funciones de distribución empírica (EDF) correspondientes a la muestra de referencia \(F_{\mathrm{ref}}\) y a la muestra actual \(F_{\mathrm{rec}}\).  
La estadística de contraste se define como:
\[
D = \sup_x |F_{\mathrm{ref}}(x) - F_{\mathrm{rec}}(x)|.
\]
El valor \(p\) asociado puede calcularse de forma exacta o mediante una aproximación asintótica, dependiendo del tamaño de las muestras comparadas.  
Se interpreta que existe evidencia de \textit{drift} cuando \(p < \alpha\), siendo \(\alpha = 0.01\) el nivel de significancia adoptado por defecto.  
Esta prueba es especialmente útil para identificar desplazamientos en la media o variaciones en la forma general de la distribución de las variables continuas.

\paragraph{Variables categóricas (Chi-cuadrado).}  
En las variables categóricas, el análisis se realiza construyendo una \textbf{tabla de contingencia} que resume las frecuencias observadas en cada categoría para ambas muestras.  
La independencia entre distribuciones se contrasta mediante la prueba de \(\chi^2\), que evalúa la magnitud de la discrepancia entre frecuencias esperadas y observadas bajo la hipótesis de estabilidad:
\[
\chi^2 = \sum_{i} \frac{(O_i - E_i)^2}{E_i}.
\]
Si el valor \(p < \alpha\), se rechaza la hipótesis nula, indicando una alteración estadísticamente significativa en la composición categórica y, en consecuencia, la presencia de \textit{drift}.  
Esta prueba resulta adecuada para detectar cambios abruptos o desbalances en atributos discretos y de baja cardinalidad.

\paragraph{PSI (Population Stability Index) en \textit{scores} o características derivadas.}  
En el caso de variables continuas o atributos derivados —como el \textit{score} del modelo— se aplica el \textbf{Population Stability Index (PSI)}, el cual cuantifica la divergencia entre la distribución de referencia y la observada al discretizarlas en \(K\) intervalos o \textit{bins}, definidos según los cuantiles de la muestra base.  
Si \(\{r_k\}\) y \(\{c_k\}\) representan las proporciones de observaciones en el bin \(k\) para la referencia y la muestra reciente, respectivamente, el PSI se calcula como:
\[
\mathrm{PSI} = \sum_{k=1}^{K} (r_k - c_k)\,\ln\!\frac{r_k}{c_k}.
\]
Para evitar inestabilidades numéricas, se aplican recortes \(\epsilon\) a proporciones extremadamente pequeñas (\(r_k, c_k < \epsilon\)), evitando así operaciones indefinidas como \(\log(0)\).  
En la práctica, valores de \(\mathrm{PSI} > 0.2\) se interpretan como señal de alerta moderada, mientras que \(\mathrm{PSI} > 0.3\) indica \textit{drift} pronunciado \citep{Chawla2021,Nguyen2023}.  

El uso conjunto de estas tres métricas —KS, $\chi^2$ y PSI— proporciona una cobertura más completa del fenómeno de \textit{data drift}, al capturar tanto variaciones continuas en las distribuciones como cambios estructurales en categorías o proporciones de salida del modelo, ofreciendo así una caracterización robusta de la estabilidad del sistema.

\paragraph{Hipótesis y regla de decisión.}
En cada ciclo comparamos una muestra de referencia $(X_{1:n})$ con una muestra reciente $(Y_{1:m})$ para cada variable $j$.
La hipótesis nula es $H_0^{(j)}:\;F^{(j)}_{\mathrm{ref}} = F^{(j)}_{\mathrm{rec}}$ (estabilidad); la alternativa $H_1^{(j)}$ indica \textit{drift}.
Se rechaza $H_0^{(j)}$ si $p^{(j)}<\alpha$ o si $\mathrm{PSI}^{(j)}>\tau$ (regla OR). Para múltiples columnas ($J$ pruebas por ciclo), se aplica corrección de comparaciones múltiples (Holm–Bonferroni o FDR-BH) sobre $\{p^{(j)}\}$.

\paragraph{KS (numéricas).}
La estadística de dos muestras es
\[
D^{(j)} \;=\; \sup_x \big| \hat F^{(j)}_{\!n}(x) - \hat G^{(j)}_{\!m}(x) \big|, \qquad
n_{\mathrm{eff}}=\frac{nm}{n+m}.
\]
El valor-$p$ asintótico se obtiene con la distribución de Kolmogórov:
\[
p^{(j)} \approx Q_{\mathrm{KS}}\!\Big(\big(\sqrt{n_{\mathrm{eff}}}+0.12+\tfrac{0.11}{\sqrt{n_{\mathrm{eff}}}}\big) D^{(j)}\Big),
\]
donde $Q_{\mathrm{KS}}$ es la cola de la ley límite. Un \textbf{banda de confianza} $(1-\gamma)$ para la diferencia de CDF se deriva de Dvoretzky–Kiefer–Wolfowitz:
\[
\Pr\!\Big(\sup_x \big| \hat F^{(j)}_{\!n}(x) - F^{(j)}(x) \big| \le \varepsilon_\gamma\Big) \ge 1-\gamma,\quad
\varepsilon_\gamma=\sqrt{\frac{1}{2n_{\mathrm{eff}}}\ln\frac{2}{\gamma}},
\]
que permite informar si $D^{(j)} > \varepsilon_\gamma$ (evidencia de deriva con confianza $1-\gamma$).

\paragraph{$\chi^2$ (categóricas).}
Con $K$ categorías y conteos observados $O_k$ (reciente) y esperados $E_k$ (a partir de la referencia),
\[
\chi^2 \;=\; \sum_{k=1}^{K} \frac{(O_k-E_k)^2}{E_k}\;\sim\;\chi^2_{\,K-1}\quad\text{(bajo $H_0$ si $E_k\ge 5$).}
\]
Se reporta $p = 1-F_{\chi^2_{K-1}}(\chi^2)$ y el \emph{tamaño de efecto} con el estadístico de Cramér
\[
V \;=\; \sqrt{\frac{\chi^2}{N\,(K-1)}}\,,
\]
acompañado de un \textbf{intervalo de confianza} $(1-\gamma)$ para $V$ obtenido por \emph{bootstrap} no paramétrico con $B$ remuestreos (percentil).

\paragraph{PSI (score/features derivadas).}
Discretizando en $K$ bins definidos por cuantiles de la referencia (proporciones $r_k$ vs.\ $c_k$ en reciente), el índice es
\[
\mathrm{PSI} \;=\; \sum_{k=1}^{K} (r_k-c_k)\,\ln\!\frac{r_k}{c_k},\qquad
r_k,c_k \leftarrow \max(r_k,c_k,\varepsilon).
\]
El PSI se usa como \emph{medida de magnitud del cambio} (no como prueba exacta); se informan \textbf{IC $(1-\gamma)$ por bootstrap} (percentil o BCa) y umbrales operativos: alerta si $\mathrm{PSI}>0.2$ y \textit{drift} pronunciado si $\mathrm{PSI}>0.3$.

\paragraph{Control por comparaciones múltiples.}
Si en un ciclo se contrastan $J$ variables, los $p$-valores se ajustan con Holm–Bonferroni:
\[
p^{(j)}_{\text{adj}}=\max_{k\le j}\big\{(J-k+1)\,p_{(k)}\big\}\quad\text{sobre los }p\text{ ordenados }p_{(1)}\le\dots\le p_{(J)},
\]
o alternativamente con FDR-Benjamini–Hochberg (nivel $q$) cuando se prioriza sensibilidad.

\paragraph{Justificación de $\boldsymbol{\alpha=0.01}$.}

\begin{enumerate}
    \item \textb{Carga de pruebas}: con ventanas sub-minutales y $J$ columnas, el número esperado de falsos positivos por ciclo bajo FWER es $\mathbb{E}[\mathrm{FP}]\approx J\alpha$ (antes de ajuste). Fijar $\alpha=0.01$ reduce FP esperadas y, tras Holm/BH, mantiene el FWER/FDR por debajo de lo operativo.\newline
    \item \textb{Coste de omisión vs.\ falsa alarma}: el sistema privilegia \emph{sensibilidad} (regla OR) pero aplica \emph{edge-trigger}+\emph{cooldown}; un $\alpha$ más estricto compensa la mayor sensibilidad evitando \emph{flapping}.\newline
    \item \textb{Cadencia}: a cadencias altas, $\alpha=0.05$ generaría exceso de FP acumuladas por hora; con $\alpha=0.01$ y corrección, la tasa de disparos espurios se mantiene < 1/hora en la configuración base.
\end{enumerate}

\paragraph{Reporte inferencial en resultados (formato).}
Para cada variable y ciclo: \emph{KS/$\chi^2$}: $D$ o $\chi^2$, $p_{\text{adj}}$ (Holm/BH) e IC $(1-\gamma)$ (DKW para KS; $V$ con IC bootstrap en categóricas). \emph{PSI}: valor puntual e IC bootstrap $(1-\gamma)$, indicando si supera $\tau\in\{0.2,0.3\}$. A nivel de sistema, se resume $p_{\min}$ ajustado, la fracción de columnas con rechazo y la coherencia con $\mathrm{PSI}$ (con sus IC), además de la latencia hasta la primera detección (TTFD).


\subsection{Metodología}

La infraestructura descrita en este capítulo habilita el entorno necesario para ejecutar los experimentos; el \textbf{diseño formal}, los \textbf{escenarios E1--E3}, las \textbf{métricas} y los \textbf{KPIs} se documentan íntegramente en el Capítulo~\ref{sec:eval-results}.  
Aquí sólo se puntualizan los elementos operativos:
\begin{itemize}\setlength\itemsep{2pt}
  \item Los ciclos de monitoreo se ejecutan en ventanas sub-minutales coordinadas por \texttt{drift\_watch.py}, que lee, contrasta y publica métricas en Prometheus.
  \item El muestreo se limita mediante \texttt{SAMPLE\_MAX} para evitar sesgos por tamaños de ventana variables; los detalles estadísticos (bootstrap, pruebas, IC) se presentan en Cap.~4.
  \item Las métricas expuestas (TTFD, MTTR, tasas de alerta, overhead) se consumen posteriormente para la evaluación cuantitativa, manteniendo la definición de indicadores en la sección de resultados.
\end{itemize}
De este modo, Cap.~3 se centra en la habilitación tecnológica, mientras que Cap.~4 compila los análisis comparativos y la discusión estadística.

\subsection{Criterios de parametrización y robustez}

La configuración detallada de umbrales ($\alpha$, $\tau$), tamaños de ventana y políticas de \textit{cooldown} se discute en la Sección~\ref{subsec:eval-design-formal} y en los resultados experimentales (Tablas~\ref{tab:scenario-metrics} y \ref{tab:mw-tests}).  
En el contexto de infraestructura sólo se destaca que:
\begin{itemize}\setlength\itemsep{2pt}
  \item \texttt{DRIFT\_ALPHA}, \texttt{PSI\_ALERT}, \texttt{SAMPLE\_MAX} y \texttt{DRIFT\_COOLDOWN\_SECONDS} son parámetros de entorno versionados en \texttt{.env} y en los scripts de despliegue.
  \item Las combinaciones de umbrales se prueban mediante \texttt{drift\_watch.py} y quedan registradas en MLflow para su análisis posterior.
  \item Las decisiones sobre sensibilidad/estabilidad se justifican con evidencia estadística en Cap.~4, evitando duplicidad en la descripción metodológica.
\end{itemize}


\subsection{Implementación e instrumentación}

El núcleo funcional del detector se materializa en el servicio \texttt{drift\_watch.py}, implementado sobre \textbf{PySpark} y ejecutado como un proceso continuo de lectura, evaluación y publicación de métricas.  
Su objetivo es supervisar de forma ininterrumpida la estabilidad estadística de los datos y comunicar los resultados a la infraestructura de observabilidad mediante un \textit{exporter} compatible con \textbf{Prometheus}, asegurando la integración con el resto del ecosistema MLOps.

\paragraph{Métricas expuestas.}  
Durante cada iteración del ciclo de monitoreo, el servicio calcula y publica un conjunto de indicadores que reflejan el estado actual del sistema y los resultados obtenidos por las pruebas estadísticas.  
Entre las métricas más relevantes se encuentran:
\begin{itemize}\setlength\itemsep{2pt}
  \item \texttt{pvalue\{col\}} y \texttt{drift\_detected\{col\}}: valores $p$ provenientes de las pruebas KS y $\chi^2$ para cada columna, junto con una bandera binaria que indica si se detectó \textit{drift}.
  \item \texttt{drift\_score\_psi\{model\}}: valor del índice $\mathrm{PSI}$ calculado sobre el \textit{score} del modelo, empleado para medir la estabilidad poblacional entre periodos consecutivos.
  \item \texttt{predicted\_positive\_ratio\{model\}}: proporción de instancias clasificadas como positivas, útil para detectar desplazamientos en la tasa de predicción.
  \item \texttt{jenkins\_retrain\_triggers\_total\{model\}}: contador acumulativo de ejecuciones de reentrenamiento activadas por el detector.
\end{itemize}

Estas métricas son recolectadas de forma automática por Prometheus y visualizadas en Grafana, lo que permite monitorear en tiempo real la magnitud, frecuencia y persistencia de los eventos de \textit{drift}, así como su impacto en el desempeño del modelo.

\paragraph{Parámetros operativos.}  
El comportamiento del detector puede ajustarse mediante variables de entorno configurables en cada despliegue, lo que facilita su calibración frente a diferentes volúmenes y ritmos de flujo de datos.  
Entre las principales se incluyen:
\begin{itemize}\setlength\itemsep{2pt}
  \item \(\alpha = \texttt{DRIFT\_ALPHA} = 0.01\): nivel de significancia empleado en las pruebas KS y $\chi^2$.
  \item \(\tau = \texttt{PSI\_ALERT} = 0.2\): umbral de alerta para el índice PSI.
  \item \texttt{SAMPLE\_MAX}: cantidad máxima de filas por muestra, que actúa como límite de muestreo para garantizar estabilidad y comparabilidad entre ciclos.
  \item \texttt{EDGE\_ONLY}: habilita el disparo únicamente por flanco de subida, evitando activaciones redundantes.
  \item \texttt{DRIFT\_CLEAR\_STREAK}: número mínimo de ciclos consecutivos sin detección requeridos para rearmar el detector.
  \item \texttt{DRIFT\_COOLDOWN\_SECONDS}: tiempo mínimo de enfriamiento entre dos detecciones consecutivas de \textit{drift}.
\end{itemize}

Estos parámetros permiten ajustar la sensibilidad y la frecuencia de respuesta del sistema, equilibrando la capacidad de detección temprana con la estabilidad operativa bajo diferentes cargas de trabajo.

\paragraph{Algoritmo del servicio \textit{drift-watcher}.}
\begin{algorithm}[H]
\caption{Servicio de monitoreo y disparo de reentrenamiento}
\begin{algorithmic}[1]
\State Inicializar ventanas de referencia y reciente con tamaño \texttt{SAMPLE\_MAX}
\State Fijar parámetros: $\alpha$, $\tau$, \texttt{COOLDOWN}, \texttt{CLEAR\_STREAK}
\While{servicio activo}
  \State Obtener lote reciente $X_{\text{rec}}$, actualizar ventana deslizante
  \State Calcular $p_{\mathrm{KS}}$, $p_{\chi^2}}$ y $\mathrm{PSI}$
  \State $trigger \gets (p_{\mathrm{KS}} < \alpha) \lor (p_{\chi^2} < \alpha) \lor (\mathrm{PSI} > \tau)$
  \If{$trigger$ y no en \texttt{COOLDOWN}}
     \State Registrar métricas en Prometheus/MLflow
     \State Invocar \texttt{JENKINS\_JOB} (reentrenamiento automático)
     \State Iniciar temporizador \texttt{COOLDOWN} y reiniciar \texttt{CLEAR\_STREAK}
  \Else
     \State Incrementar \texttt{CLEAR\_STREAK} si $trigger = \texttt{false}$
     \If{\texttt{CLEAR\_STREAK} $\geq$ umbral}
        \State Reestablecer estado “sin drift”
     \EndIf
  \EndIf
  \State Esperar \texttt{LOOP\_SECONDS} y repetir
\EndWhile
\end{algorithmic}
\end{algorithm}

Este pseudo-código sintetiza la lógica del \textit{drift watcher}: cálculo estadístico por ventana, regla de fusión ($p < \alpha$ o $\mathrm{PSI}>\tau$), disparo por flanco con \textit{cooldown} y rearme condicionado a rachas limpias.

\paragraph{Orquestación y automatización.}  
Cuando se cumplen las condiciones de disparo —ya sea por \(p < \alpha\) o \(\mathrm{PSI} > \tau\)—, el servicio ejecuta automáticamente el flujo de reentrenamiento definido en \textbf{Jenkins} a través de la variable \texttt{JENKINS\_JOB}.  
La autenticación se gestiona mediante credenciales seguras (\texttt{JENKINS\_USER} y \texttt{JENKINS\_API\_TOKEN}) o, en versiones heredadas, por medio de \texttt{JENKINS\_TOKEN}.  
Adicionalmente, se emplea un token de validación (“crumb”) para proteger las solicitudes HTTP frente a ataques CSRF.

El proceso de reentrenamiento lanza el script \texttt{train\_model.py}, responsable de ajustar nuevamente un modelo de \textbf{Regresión Logística} con el parámetro \texttt{class\_weight="balanced"} para compensar desbalances en la variable objetivo.  
Una vez finalizada la ejecución, los resultados son registrados en \textbf{MLflow}, almacenando parámetros de configuración, métricas de desempeño (por ejemplo, F1-score) y artefactos del modelo generado.  
Este flujo cierra el ciclo de \textbf{detección–acción–verificación}, garantizando trazabilidad completa y reproducibilidad experimental en el entorno MLOps \citep{Chen2022}.

%\subsubsection{Realización técnica e instrumentación (scripts y métricas)}
\subsection{Implementación e instrumentación práctica}
\label{subsec:oe2-impl-practica}

Para vincular el diseño de \autoref{fig:oe2-flujo} con artefactos ejecutables, este objetivo se materializa en tres scripts principales y un conjunto de métricas expuestas a Prometheus:

\begin{itemize}
  \item \texttt{generate\_data\_session.py}: genera datos transaccionales sintéticos con opción de inducir \textit{data drift} y escribe en HDFS particionado (dt/hour/account\_type) mediante Spark. % (datos de entrada controlados)
  \item \texttt{drift\_watch.py}: servicio persistente que lee desde HDFS, aplica KS (numéricas), $\chi^2$ (categóricas) y PSI (score/features), publica métricas en Prometheus y dispara reentrenos en Jenkins bajo umbrales $p<\alpha$ o $\mathrm{PSI}>\tau$. % (detección+orquestación)
  \item \texttt{train\_model.py}: reentrena (Regresión Logística) con \texttt{class\_weight="balanced"}, valida y registra parámetros, métricas y artefactos en MLflow. % (acción+verificación)
\end{itemize}

\paragraph{Métricas clave expuestas (Prometheus).}
Las señales operativas y estadísticas se exponen como (\emph{exporter}) para su scrapeo y visualización (Grafana):

\begin{center}
\begin{tabular}{ll}
\toprule
\textbf{Métrica} & \textbf{Descripción} \\
\midrule
\texttt{drift\_score\_psi\{model\}} & Índice PSI sobre el \textit{score} del modelo. \\
\texttt{pvalue\{col\}} & Valor-$p$ por columna (KS o $\chi^2$). \\
\texttt{drift\_detected\{col\}} & Bandera de drift (1 si se detecta, 0 en caso contrario). \\
\texttt{predicted\_positive\_ratio\{model\}} & Proporción de predicciones positivas. \\
\texttt{jenkins\_retrain\_triggers\_total\{model\}} & Contador de reentrenamientos activados. \\
\bottomrule
\end{tabular}
\end{center}

\paragraph{Variables de entorno (extracto).}
La \autoref{tab:oe2-env} resume las variables mínimas para parametrizar el comportamiento del monitor, la lectura/escritura en HDFS y la integración con Jenkins/MLflow.

\begin{table}[htbp]
\centering
\caption{Variables de entorno clave para OE2.}
\label{tab:oe2-env}
\footnotesize
\resizebox{\textwidth}{!}{%
\begin{tabular}{lll}
\toprule
\textbf{Variable} & \textbf{Ejemplo} & \textbf{Rol} \\
\midrule
\texttt{SPARK\_MASTER\_URL} & \texttt{spark://spark-master:7077} & Sesión Spark para lectura/procesamiento. \\
\texttt{HDFS\_URI} & \texttt{hdfs://namenode:9000} & Filesystem distribuido (fuente/depósito). \\
\texttt{DATA\_PATH} & \texttt{hdfs:///datalake/raw/bank\_transactions} & Ruta Parquet de datos monitoreados. \\
\texttt{DRIFT\_ALPHA} & \texttt{0.01} & Nivel de significancia para KS/$\chi^2$. \\
\texttt{PSI\_ALERT} & \texttt{0.2} & Umbral de alerta para PSI. \\
\texttt{SAMPLE\_MAX} & \texttt{1000} & Límite de filas por ciclo (estabilidad de muestra). \\
\texttt{EXPORTER\_PORT} & \texttt{8010} & Puerto HTTP de métricas (Prometheus). \\
\texttt{DRIFT\_COOLDOWN\_SECONDS} & \texttt{300} & Enfriamiento mínimo entre detecciones. \\
\texttt{TRIGGER\_EDGE\_ONLY} & \texttt{1} & Disparo por flanco; evita reactivaciones. \\
\texttt{JENKINS\_URL} & \texttt{http://jenkins:8080} & Orquestador CI/CD. \\
\texttt{JENKINS\_JOB} & \texttt{retrain-model} & Job invocado ante \textit{drift}. \\
\texttt{JENKINS\_USER}/\texttt{JENKINS\_API\_TOKEN} & \texttt{usuario}/\texttt{******} & Autenticación segura. \\
\texttt{MLFLOW\_TRACKING\_URI} & \texttt{http://mlflow:5000} & Registro de \textit{runs}, métricas y artefactos. \\
\bottomrule
\end{tabular}}
\\[2pt]
\raggedright\footnotesize
Fuentes: implementación en \texttt{drift\_watch.py}, \texttt{train\_model.py} y documentación del repositorio Arlequín.
\end{table}


\paragraph{Trazabilidad con el diagrama.}
Este encadenamiento \emph{datos recientes} $\rightarrow$ \emph{pruebas (KS, $\chi^2$, PSI)} $\rightarrow$ \emph{decisión} $\rightarrow$ \emph{trigger Jenkins} $\rightarrow$ \emph{train \& log en MLflow} $\rightarrow$ \emph{métricas en Prometheus/Grafana} implementa, de extremo a extremo, el flujo de \autoref{fig:oe2-flujo} y habilita auditoría técnica sobre cada evento de drift y su correspondiente acción correctiva.


\subsection{Resultados y análisis}

En los escenarios experimentales con \textbf{\textit{drift} inducido de forma controlada}, el detector presentó un desempeño coherente con los objetivos de diseño.  
Durante las pruebas, las variables manipuladas para alterar su distribución mostraron valores \(p < \alpha\) en las pruebas KS y $\chi^2$, y/o índices \(\mathrm{PSI} > \tau\), lo que activó correctamente las alertas de desviación.  
Estas detecciones se tradujeron en la ejecución automática de los procesos de reentrenamiento en Jenkins, evidenciando una \textbf{baja latencia de detección (TTFD reducido)} y una respuesta ágil ante cambios estadísticamente significativos en los datos.

El seguimiento realizado mediante \textbf{MLflow} confirmó que, tras cada reentrenamiento, el modelo recuperó su \textbf{nivel de desempeño predictivo} —evaluado principalmente con la métrica F1—, validando que el ciclo de detección, respuesta y verificación funciona de manera autónoma y reproducible.  
De forma complementaria, las métricas recolectadas en \textbf{Prometheus} evidenciaron una relación directa entre los picos de \textit{drift}, las activaciones de Jenkins y las mejoras posteriores en las métricas del modelo, demostrando la coherencia entre los distintos componentes del pipeline.

La combinación de la política \textbf{\textit{edge-triggered}} con períodos de \textbf{\textit{cooldown}} y el requisito de \textbf{rachas limpias} (\textit{clean streaks}) resultó clave para prevenir el \textit{flapping}, es decir, la activación repetida de reentrenamientos durante fluctuaciones menores o eventos de \textit{drift} sostenido.  
Este esquema permitió mantener una operación estable, reducir la carga de cómputo y garantizar la fiabilidad del proceso de monitoreo.

En conjunto, los resultados obtenidos demuestran que el detector cumple con los objetivos establecidos para OE2:  
(i) detectar oportunamente desviaciones estadísticas en los datos,  
(ii) generar telemetría útil para la observabilidad del sistema, y  
(iii) activar el reentrenamiento de forma controlada y eficiente.  
Estos hallazgos validan la efectividad del enfoque propuesto y establecen la base experimental para OE3, donde se evalúa la recuperación y el desempeño del modelo reentrenado frente a distintos escenarios de \textit{drift}.


\subsection{Discusión}
\paragraph{Cierre interpretativo.} En síntesis, la concurrencia de señales KS/$\chi^2$/PSI permitió detectar el cambio con TTFD bajo, y el acoplamiento con Jenkins habilitó una recuperación posterior del F1 tras el reentrenamiento, confirmando la coherencia del ciclo detectar $\rightarrow$ reentrenar $\rightarrow$ validar.
Este comportamiento coincide con las hipotesis operativas de OE2 y con practicas reportadas de monitoreo de drift; sin embargo, la sensibilidad observada sugiere profundizar en la calibracion de $\alpha$ y $\tau$ para controlar sobrealertas.


Los resultados obtenidos en OE2 permiten analizar el equilibrio alcanzado entre \textbf{sensibilidad, interpretabilidad y estabilidad operativa} en la detección de \textit{data drift}.  
La estrategia de fusión \textbf{OR} —que considera desviación cuando al menos una de las pruebas univariadas o el PSI supera su umbral— resultó eficaz para incrementar la \textbf{sensibilidad} del detector frente a cambios heterogéneos en los distintos atributos.  
Esta política prioriza la detección temprana sobre la especificidad, permitiendo identificar variaciones parciales en el flujo de datos antes de que se traduzcan en pérdidas sustanciales de desempeño del modelo.

La incorporación del \textbf{Population Stability Index (PSI)} aporta un componente de \textbf{interpretabilidad operacional} al sistema, ya que no solo emite una señal binaria de alerta, sino que ofrece una medida continua de la \textbf{magnitud y dirección} del cambio dentro de cada intervalo de la distribución.  
Esta característica permite a los operadores distinguir entre fluctuaciones marginales y transformaciones estructurales, facilitando un diagnóstico más preciso del tipo de deriva presente.

Sin embargo, el enfoque actual se apoya exclusivamente en \textbf{pruebas univariadas}, lo que limita su capacidad para capturar correlaciones entre variables o desplazamientos conjuntos en el espacio de entrada.  
Una línea de mejora natural consiste en incorporar \textbf{métodos multivariados} como la \textit{Maximum Mean Discrepancy (MMD)} o la \textit{Energy Distance}, que permiten contrastar distribuciones de alta dimensión sin requerir supuestos paramétricos.  
Del mismo modo, podrían explorarse esquemas de \textbf{agregación bayesiana de evidencias}, donde los resultados de cada prueba se combinen en una probabilidad compuesta de \textit{drift}, modulada por la confiabilidad de cada señal.

Por último, para abordar de manera integral la evolución del modelo en producción, sería pertinente extender el monitoreo hacia el \textbf{\textit{concept drift}}.  
Ello implicaría evaluar periódicamente el desempeño supervisado del modelo con datos etiquetados recientes, permitiendo diferenciar entre variaciones estadísticas en las entradas y degradaciones reales en la relación entre variables y etiquetas.  
De esta forma, el sistema podría evolucionar hacia un marco de \textbf{aprendizaje adaptativo continuo}, alineado con las recomendaciones de \citet{Gama2014} y \citet{Lu2019}, consolidando un ciclo MLOps más resiliente y autónomo.
las recomendaciones de \citet{Gama2014} y \citet{Lu2019}.  

En conclusión, el diseño actual optimiza sensibilidad e interpretabilidad bajo restricciones de latencia y costo; la extensión multivariada y la agregación probabilística de evidencias constituyen el camino natural hacia un detector más específico sin sacrificar \emph{TTFD}.


\subsection{Amenazas a la validez}

Aunque los resultados obtenidos en OE2 fueron consistentes con los objetivos planteados, existen factores que pueden influir en la interpretación y la generalización del desempeño del detector.  
Las principales amenazas se agrupan en tres dimensiones: interna, externa y de conclusión.

\textbf{Validez interna.}  
La fiabilidad de las pruebas estadísticas utilizadas depende en gran medida del tamaño de las muestras procesadas en cada ciclo.  
Valores de \(p\) excesivamente sensibles pueden inducir \textbf{falsos positivos} cuando el tamaño muestral es reducido, ya que la variabilidad aleatoria puede simular diferencias inexistentes.  
Del mismo modo, la \textbf{baja cardinalidad} en atributos categóricos puede generar frecuencias esperadas inestables, afectando la validez de la prueba \(\chi^2\).  
En cuanto al PSI, la selección del número de \textit{bins} y del parámetro de corrección \(\epsilon\) tiene un efecto directo sobre su estabilidad: una discretización inadecuada puede amplificar o atenuar artificialmente las divergencias entre muestras.  
Estas condiciones introducen un margen de incertidumbre sobre la consistencia interna de los resultados obtenidos.

\textbf{Validez externa.}  
El rendimiento del detector puede verse afectado al trasladarlo a dominios con \textbf{fuerte estacionalidad} o patrones de cambio periódicos, donde las fluctuaciones naturales podrían confundirse con eventos de \textit{drift}.  
Asimismo, los \textbf{patrones de ingesta de datos} —como picos de actividad por hora, variaciones en el volumen de lotes o diferencias en su procedencia— pueden alterar la representatividad estadística de las muestras recientes y, por ende, comprometer la estabilidad de los umbrales calibrados (\(\alpha\), \(\tau\)).  
Por ello, la extrapolación de los resultados obtenidos en entornos experimentales controlados hacia contextos productivos multicliente requiere una recalibración contextual de parámetros y ventanas de observación.

\textbf{Validez de conclusión.}  
Existe un riesgo residual de \textbf{errores de interpretación} —tanto falsos positivos como falsos negativos— cuando los cambios en los datos son graduales, correlacionados o de carácter multivariado, dado que las pruebas univariadas no capturan interdependencias complejas entre variables.  
Ello puede ocasionar alertas tardías o la omisión de \textit{drift} sutiles pero relevantes.  
Sin embargo, estas limitaciones pueden mitigarse mediante el uso de \textbf{ventanas adaptativas}, que ajusten su tamaño según la variabilidad del flujo de datos, y mediante la incorporación de \textbf{métodos multivariados} que consoliden información de múltiples dimensiones en una medida global de desviación.  
Estas mejoras fortalecerían la robustez estadística del detector y aumentarían la confiabilidad de las conclusiones derivadas del monitoreo continuo.


% ===========================

% ===========================
%\section{OE3. Validar en un escenario simulado con y sin \textit{data drift} la eficacia del sistema computacional}
\section{OE3. Validación experimental del sistema ante \textit{data drift}}
\label{sec:oe3}
\subsection*{Resumen}

Se realizó una \textbf{validación experimental controlada} con el fin de evaluar la efectividad del sistema de detección y reentrenamiento automático frente a distintos escenarios de estabilidad y cambio en los datos.  
El estudio se estructuró en dos condiciones comparativas:  
(i) un escenario base sin alteraciones, denominado \textbf{E1 (sin \textit{drift})}, que representa el comportamiento normal del modelo, y  
(ii) un escenario \textbf{E2 (con \textit{drift} inducido)}, en el cual se introdujeron perturbaciones deliberadas sobre \textit{covariate shifts}; la variación del \texttt{risk\_score} aporta sólo una representación parcial de \textit{concept drift}.  

En cada caso se evaluaron métricas de desempeño predictivo —F1, AUC, \textit{precision} y \textit{recall}—, junto con indicadores operativos del sistema como la latencia de detección y el tiempo total de reentrenamiento.  
Los resultados mostraron que, tras la detección y el reentrenamiento automatizado, el modelo recupera significativamente su nivel de desempeño, confirmando la eficacia del ciclo de adaptación propuesto y su capacidad para mantener la estabilidad predictiva en presencia de \textit{data drift} \citep{Singh2023,Chatterjee2023,Lu2019}.


\subsection{Introducción}

El tercer objetivo específico (OE3) corresponde a la \textbf{validación experimental} del sistema implementado, cuyo propósito es demostrar su capacidad para reaccionar de manera autónoma ante desviaciones en la distribución de los datos y para restablecer el rendimiento del modelo con mínima intervención humana.  
Esta fase evalúa el comportamiento integral del pipeline MLOps bajo condiciones controladas, verificando su eficacia en la detección del \textit{drift}, la activación de los procesos de reentrenamiento y la estabilización del desempeño del modelo a lo largo de ciclos continuos de operación.

En contextos reales de aprendizaje automático, las distribuciones de los datos rara vez permanecen estáticas.  
Cambios en el comportamiento de los usuarios, modificaciones en las fuentes de información o variaciones en los procesos de negocio pueden alterar progresivamente el entorno de entrenamiento y predicción \citep{Gama2014,Sethi2017}.  
Estas variaciones, conocidas como \textit{data drift}, afectan la validez de los modelos en producción, provocando una degradación gradual de su capacidad predictiva si no se detectan y corrigen oportunamente.

Por ello, OE3 se orienta a validar empíricamente la hipótesis central del proyecto: que un \textbf{pipeline MLOps automatizado}, dotado de \textbf{mecanismos estadísticos de detección de \textit{drift}} y \textbf{procedimientos de reentrenamiento continuo}, puede sostener un desempeño estable y verificable incluso frente a cambios distribucionales inducidos.  
La validación experimental representa, en consecuencia, la etapa de cierre del ciclo de retroalimentación entre monitoreo, detección, reentrenamiento y evaluación, asegurando la adaptabilidad del sistema y la reproducibilidad de los resultados en entornos dinámicos.

\subsection{Diseño experimental y variables}
\label{subsec:oe3-design}

\subsection*{Protocolo de ejecución (replicabilidad)}
Para estandarizar la replicación, la Tabla~\ref{tab:oe3-protocolo} consolida muestreo, semillas, ciclos y hardware. Los parámetros de entorno complementarios se listan en la \autoref{tab:oe2-env}.

\begin{table}[htbp]
\centering
\caption{Protocolo experimental estandarizado (OE3).}
\label{tab:oe3-protocolo}
\begin{tabular}{ll}
\toprule
\textbf{Aspecto} & \textbf{Especificación} \\
\midrule
Muestreo por ciclo & Ventanas recientes y referencia; \texttt{SAMPLE\_MAX}=1000 \\
Frecuencia de ciclo & Según configuración del lazo (\texttt{LOOP\_SECONDS}) \\
Semillas & \texttt{seed} = 42, 7 (generación/perturbación) \\
Escenarios & E1 (base), E2 (drift inducido), E3 (post-reentrenamiento) \\
Réplicas & $n=5$ por escenario (promedio + IC95\%) \\
Hardware & Completar: CPU / RAM / Almacenamiento del host \\
Ejecución & Docker Compose; servicios: Spark, HDFS, Jenkins, MLflow, Prometheus \\
Artefactos & Runs en MLflow, métricas en Prometheus, logs de Jenkins \\
\bottomrule
\end{tabular}
\end{table}
\paragraph{Población, unidades y muestreo.}
Cada \emph{réplica} $r=1,\dots,R$ consiste en dos condiciones ejecutadas sobre el mismo entorno contenedorizado recién reiniciado:
\textbf{E1} (sin \textit{drift}) y \textbf{E2} (con \textit{drift} inducido).
En cada condición se generan ventanas deslizantes de monitoreo de periodo fijo $\Delta t$ y tamaño muestral acotado por \texttt{SAMPLE\_MAX}.
Parámetros base: $R=20$ réplicas, $\Delta t=30\,\mathrm{s}$, horizonte por condición $\approx 15\,\mathrm{min}$ (hasta 30 ciclos/condición), y \texttt{SAMPLE\_MAX}$=1000$ filas por ciclo.

\paragraph{Factores y control.}
\begin{itemize}\setlength\itemsep{2pt}
  \item \textbf{Factor experimental (binario):} Condición $\in\{\mathrm{E1},\mathrm{E2}\}$.
  \item \textbf{Tratamiento en E2:} \texttt{drift\_factor}=1.0 con rampa suave entre los ciclos 5–8 sobre \texttt{amount\_usd} y \texttt{risk\_score}.
  \item \textbf{Bloqueo y aleatorización:} semillas distintas por réplica; orden de ejecución E1$\rightarrow$E2 para pares apareados; reinicio de contenedores y limpieza de cachés antes de cada réplica.
  \item \textbf{Control de sesgo:} descarte de un ciclo de \emph{warm-up}, fijación de versiones Docker/paquetes, aislamiento de carga (sin jobs paralelos).
\end{itemize}

\paragraph{Hipótesis.}
Sea $M$ el modelo desplegado, $F1(\cdot)$ la métrica objetivo y $\Delta F1 = F1_{\text{post}}-F1_{\text{pre}}$ en E2.
\[
\begin{aligned}
&\textbf{H}_{0}^{(E1)}:\ \text{tasa de falsas alertas}\le \pi_0 \quad(\pi_0=1\%\ \text{por hora}) \\
&\textbf{H}_{0}^{(E2)}:\ \mathbb{E}[\Delta F1]\le 0 \qquad\text{vs.}\qquad \textbf{H}_{1}^{(E2)}:\ \mathbb{E}[\Delta F1]>0 \\
&\textbf{H}_{0}^{(\text{lat})}:\ \text{TTFD}\ge 120\,\mathrm{s}\ \ \text{y/o}\ \ \text{TTR}\ge 5\,\mathrm{min}
\end{aligned}
\]
Además, se comprueba coherencia de las señales estadísticas:
rechazo en KS/$\chi^2$ con $p_{\text{adj}}<\alpha$ y \mbox{PSI $>\tau$} durante E2.

\paragraph{Variables (definiciones y tipo).}
\begin{table}[htbp]
\centering
\caption{Variables del experimento (observables por ciclo y agregadas por réplica).}
\label{tab:oe3-variables}
\footnotesize
\resizebox{\textwidth}{!}{%
\begin{tabular}{lll}
\toprule
\textbf{Variable} & \textbf{Definición} & \textbf{Tipo / Uso} \\
\midrule
$F1$, AUC, Prec, Rec & Métricas de $M$ pre/post reentrenamiento & Dependientes (rendimiento) \\
TTFD & Tiempo hasta primera detección válida & Dependiente (latencia detector) \\
TTR & Tiempo desde alerta hasta fin de reentrenamiento & Dependiente (latencia pipeline) \\
$p_{\min}$ & $\min\{p_{\mathrm{KS}},p_{\chi^2}\}$ por ciclo & Intermedia (evidencia estadística) \\
PSI & Índice sobre \textit{score}/feature derivada & Intermedia (magnitud del cambio) \\
FP rate & Falsas alertas por hora en E1 & Calidad detector (especificidad) \\
$n,m$ & Tamaños de muestra por ciclo (ref/rec) & Control de precisión por ciclo \\
\bottomrule
\end{tabular}}
\end{table}

\paragraph{Tamaños de muestra y potencia.}
Por ciclo, $n,m\le 1000$ aseguran error estándar bajo en CDFs empíricas.
A nivel de réplica, 20 pares apareados (E2 pre vs. post) proporcionan potencia $>0.8$ para detectar $\Delta F1\ge 0.05$ con DE $\le 0.06$ (test de rangos con $\alpha=0.01$).

\paragraph{Agregación y estimación de la incertidumbre.}
\begin{itemize}\setlength\itemsep{2pt}
  \item \textbf{Por réplica:} en E2, \emph{pre} = último ciclo antes del trigger; \emph{post} = promedio de 3 ciclos estables tras redeploy. En E1, promedio sobre ciclos (tras descartar warm-up).
  \item \textbf{Entre réplicas:} estimador puntual = mediana; IC$_{95\%}$ por \emph{bootstrap} BCa ($B{=}2000$). Para tasas (FP), IC binomial de Clopper–Pearson.
\end{itemize}

\paragraph{Plan inferencial.}
\begin{itemize}\setlength\itemsep{2pt}
  \item \textbf{Rendimiento (E2):} prueba apareada \emph{Wilcoxon} sobre $\Delta F1$ (o \emph{t} apareada si normalidad); se reporta estimador de Hodges–Lehmann e IC$_{95\%}$. 
  \item \textbf{Latencias:} contraste uniliteral contra objetivos (TTFD$<120$s, TTR$<5$min) con IC$_{95\%}$ por bootstrap y tamaño de efecto (Cliff’s $\delta$).
  \item \textbf{Evidencia estadística:} $p$-valores por variable ajustados por Holm–Bonferroni; PSI con IC$_{95\%}$ bootstrap y umbrales $\tau\in\{0.2,0.3\}$.
  \item \textbf{Nivel de significancia:} $\alpha=0.01$ para reducir FP en monitoreo de alta frecuencia; corrección por comparaciones múltiples sobre $\{p^{(j)}\}$.
\end{itemize}

\paragraph{Criterios de éxito.}
(i) FP$\le\pi_0$ en E1; (ii) $\Delta F1>0$ con $p<0.01$ (apareado) en E2; 
(iii) TTFD mediana $<60$s y TTR mediana $<3$min con IC$_{95\%}$ por debajo de los umbrales; 
(iv) coherencia entre rechazo KS/$\chi^2$ ajustado y PSI$>\tau$ durante el periodo de drift.


\paragraph{Objetivo comparativo.}  
El contraste entre ambos experimentos busca analizar de forma empírica la \textbf{capacidad adaptativa del sistema}.  
En E1 se valida la estabilidad y robustez del pipeline frente a datos estacionarios, mientras que en E2 se examina su capacidad de respuesta —detección, activación del reentrenamiento y recuperación del rendimiento predictivo— ante perturbaciones distribucionales.  
Así, el diseño experimental constituye una validación integral del ciclo automatizado de detección y corrección, verificando que el sistema no solo reacciona eficazmente ante el \textit{drift}, sino que también conserva estabilidad en su ausencia.

El protocolo de validación se resume en la Figura~\ref{fig:oe3-protocolo}.

% ==== Figura OE3 (validación con y sin drift) ====
\begin{figure}[htbp]
\centering
\captionsetup{skip=8pt}
\caption{Protocolo de validación: escenarios E1 (sin drift) y E2 (con drift).}
\label{fig:oe3-protocolo}
\resizebox{0.88\linewidth}{!}{%
\begin{tikzpicture}[node distance=8mm and 14mm,
                    every node/.style={font=\footnotesize},
                    >=Latex]

  % Estilos compactos
  \tikzset{
    box/.style={draw, rounded corners=2pt, fill=gray!8, align=center,
                minimum width=18mm, minimum height=6mm, text width=30mm, inner sep=2pt},
    proc/.style={box},
    arrow/.style={-Latex, line width=0.45pt},
    band/.style={draw, rounded corners=3pt, inner sep=5mm},
    note/.style={font=\scriptsize, inner sep=1pt, fill=white, align=center}
  }

  % ----- E1: Sin drift -----
  \node[proc] (d1) {Ingesta base};
  \node[proc, right=16mm of d1] (m1) {Monitoreo};
  \node[proc, right=16mm of m1] (r1) {Registro de\\métricas};

  \draw[arrow] (d1) -- (m1);
  \draw[arrow] (m1) -- node[note, above]{\emph{sin alertas}} (r1);

  \node[band, fit=(d1)(m1)(r1),
        label={[align=center]above:Experimento 1: Sin drift (línea base)}] (E1) {};

  % ----- E2: Con drift -----
  \node[proc, below=16mm of d1] (d2) {Ingesta con\\alteraciones controladas};
  \node[proc, right=16mm of d2] (m2) {Monitoreo};
  \node[proc, right=16mm of m2] (j2) {Jenkins\\(retrain)};
  \node[proc, right=16mm of j2] (t2) {Entrenamiento\\+ Validación};
  \node[proc, right=16mm of t2] (r2) {Registro y\\comparación};

  \draw[arrow] (d2) -- (m2);
  \draw[arrow] (m2) -- node[note, above]{\emph{alerta}} (j2);
  \draw[arrow] (j2) -- (t2);
  \draw[arrow] (t2) -- (r2);

  \node[band, fit=(d2)(m2)(j2)(t2)(r2),
        label={[align=center]above:Experimento 2: Con drift inducido}] (E2) {};

  % ----- Análisis comparativo -----
  \node[proc, below=16mm of $(m1)!0.5!(t2)$, text width=36mm] (cmp)
        {Análisis comparativo\\(pre/post)};

  % Conexiones limpias a "cmp"
  \draw[arrow] (r1.south) -- ++(0,-4mm) -| (cmp.west);
  \draw[arrow] (r2.south) -- ++(0,-4mm) -| (cmp.east);

\end{tikzpicture}%
}
\end{figure}

\subsection{Metodología}

\paragraph{Procedimiento.}  
El protocolo experimental se diseñó en \textbf{cuatro fases consecutivas} para analizar el desempeño integral del sistema en escenarios con y sin \textit{data drift}, y para cuantificar su capacidad de recuperación tras la ejecución del reentrenamiento automático:

\begin{enumerate}
\item \textbf{Ejecución del escenario E1 (sin \textit{drift}).}  
Se generó un flujo continuo de datos estables durante aproximadamente 15 minutos, manteniendo las distribuciones originales sin perturbaciones.  
En esta fase se monitorearon los valores \(p\) obtenidos en las pruebas KS y $\chi^2$, el índice \(\mathrm{PSI}\) y los indicadores de uso de recursos (CPU, RAM, I/O).  
El propósito fue establecer una \textbf{línea base de estabilidad} que sirviera para estimar la tasa de falsos positivos y la latencia promedio de monitoreo en ausencia de \textit{drift}.

\item \textbf{Ejecución del escenario E2 (con \textit{drift} inducido).}  
Posteriormente, se activó el parámetro \texttt{drift\_factor=1.0} en el generador de datos con el fin de introducir alteraciones controladas en un intervalo de 10 a 15 minutos.  
El módulo \texttt{drift\_watch.py} detectó desviaciones a partir de las pruebas estadísticas y de los indicadores de estabilidad, generando alertas cuando \(p < 0.01\) o \(\mathrm{PSI} > 0.2\).  
Esta fase permitió evaluar la \textbf{sensibilidad del detector} y su \textbf{latencia de respuesta} frente a cambios distribucionales perceptibles.

\item \textbf{Reentrenamiento automatizado.}  
Una vez cumplidas las condiciones de disparo, Jenkins ejecutó el \textbf{pipeline de reentrenamiento} definido en el script \texttt{train\_model.py}.  
Este proceso incluye la partición del conjunto de datos (\textit{train/test split}), el ajuste del modelo de regresión logística y la evaluación de su rendimiento.  
Todas las ejecuciones fueron registradas en MLflow, junto con los parámetros de configuración, las métricas de desempeño (F1, AUC, \textit{precision}, \textit{recall}) y los artefactos de modelo generados.

\item \textbf{Comparación pre y post reentrenamiento.}  
Finalmente, se compararon las métricas obtenidas antes y después de la actualización del modelo para cuantificar la \textbf{recuperación del desempeño predictivo} y la \textbf{latencia total del ciclo} desde la detección del \textit{drift} hasta la estabilización posterior al reentrenamiento.  
Este análisis permitió determinar en qué medida la automatización del ciclo detección–acción restablece la precisión del modelo y preserva la continuidad operativa del sistema.
\end{enumerate}

\paragraph{Métricas de evaluación.}  
El análisis de resultados se sustentó en dos grupos de indicadores complementarios:

\begin{itemize}\setlength\itemsep{2pt}
  \item \textbf{Desempeño del modelo:} métricas clásicas de clasificación (F1, AUC, \textit{precision} y \textit{recall}), utilizadas para comparar la calidad predictiva del modelo antes y después del reentrenamiento.
  \item \textbf{Eficiencia operativa:} indicadores propios del funcionamiento del pipeline, incluyendo el \textbf{Time-To-First-Detection (TTFD)}, el \textbf{Time-To-Retrain (TTR)}, la estabilidad del sistema tras la actualización, la tasa de falsos positivos y el \textbf{ratio de alertas efectivas} (porcentaje de alertas asociadas a desviaciones reales).
\end{itemize}

\subsection{Resultados y análisis}

En la \textbf{condición base (E1)}, las métricas de desempeño se mantuvieron estables a lo largo del experimento, con valores promedio de F1 entre 0.90–0.92 y AUC en el rango 0.93–0.94.  
No se registraron activaciones de reentrenamiento ni variaciones significativas en las pruebas estadísticas, lo que confirma la \textbf{robustez del sistema en ausencia de \textit{drift}} y la correcta calibración de los umbrales de alerta.

En la \textbf{condición con \textit{drift} inducido (E2)}, se observó una degradación inmediata del modelo: F1 descendió a aproximadamente 0.70–0.75 y AUC a 0.78–0.80, acompañada de alertas estadísticas con \(p<0.01\) y \(\mathrm{PSI}>0.2\).  
Estas desviaciones activaron el mecanismo de reentrenamiento automático, ejecutado en Jenkins según las reglas definidas por el módulo \texttt{drift\_watch.py}.  
El sistema detectó las alteraciones en un tiempo promedio de \textbf{30–60 segundos (TTFD)}, tras lo cual el pipeline completó la etapa de reentrenamiento en \textbf{2–3 minutos (TTR)}.

Los registros en \textbf{MLflow} mostraron una recuperación consistente del desempeño, alcanzando F1 entre 0.88–0.90 y AUC en torno a 0.92 después del reentrenamiento.  
Las curvas de pérdida y las métricas visualizadas confirmaron la convergencia del modelo y la estabilidad del proceso de actualización.  
De manera complementaria, las métricas expuestas en \textbf{Prometheus} evidenciaron la correcta secuencia de detección, disparo y recuperación, sin presentar \textit{flapping} gracias a la aplicación de la política \textit{edge-triggered} y los intervalos de \textit{cooldown}.  

En conjunto, los resultados experimentales validan la hipótesis de OE3: el pipeline es capaz de \textbf{detectar y mitigar de forma autónoma los efectos del \textit{data drift}}, restaurando el desempeño del modelo y manteniendo estabilidad operativa en el ciclo de monitoreo.

\subsection{Resultados cuantitativos pre y post reentrenamiento}
\label{subsec:eval-results}

El experimento se desarrolló en dos fases: E1 (condición base) y E2 (escenario con \textit{data drift} inducido y reentrenamiento automático). 
Las métricas promedio y sus intervalos de confianza al 95\% se presentan en la Tabla~\ref{tab:eval-resumen}. 
Se observa una degradación significativa del modelo tras el \textit{drift} (F1: 0.91→0.72) y una recuperación posterior (F1: 0.89) luego del reentrenamiento, validando la efectividad del disparador estadístico.

\begin{table}[htbp]
\centering
\caption{Resumen de métricas por condición experimental (promedio e IC$_{95\%}$).}
\label{tab:eval-resumen}
\begin{tabular}{lcccccc}
\toprule
\textbf{Condición} & \textbf{F1} & \textbf{AUC} & \textbf{Prec} & \textbf{Rec} & \textbf{TTFD (s)} & \textbf{TTR (min)} \\
\midrule
E1 (base)           & 0.91 [0.90, 0.92] & 0.93 & 0.90 & 0.92 & --- & --- \\
E2 (pre-retrain)    & 0.72 [0.70, 0.75] & 0.79 & 0.72 & 0.74 & 45  & --- \\
E2 (post-retrain)   & 0.89 [0.88, 0.90] & 0.92 & 0.88 & 0.91 & --- & 2--3 \\
\bottomrule
\end{tabular}
\end{table}

A nivel de clases, la Tabla~\ref{tab:eval-clase} detalla la mejora observada en la clase positiva,
donde el F1 pasa de 0.64 a 0.85 tras el reentrenamiento, confirmando que la restauración del desempeño no depende de la clase mayoritaria.

\begin{table}[htbp]
\centering
\caption{Métricas por clase en E2 antes y después del reentrenamiento.}
\label{tab:eval-clase}
\begin{tabular}{lcccc}
\toprule
\textbf{Condición} & \textbf{Clase} & \textbf{Precision} & \textbf{Recall} & \textbf{F1} \\
\midrule
E2 (pre)  & Negativa & 0.87 & 0.78 & 0.82 \\
          & Positiva & 0.60 & 0.69 & 0.64 \\
\midrule
E2 (post) & Negativa & 0.89 & 0.90 & 0.89 \\
          & Positiva & 0.87 & 0.83 & 0.85 \\
\bottomrule
\end{tabular}
\end{table}

\subsection{Latencias operativas (TTFD/TTR)}
\label{subsec:latencias}

El desempeño temporal del sistema se analizó mediante dos métricas operativas: 
\textbf{TTFD} (tiempo hasta detección del \textit{drift}) y \textbf{TTR} (tiempo hasta finalizar el reentrenamiento). 
La Tabla~\ref{tab:latencias} resume estadísticas descriptivas; la Figura~\ref{fig:ttfd-ttr-boxplots} muestra las distribuciones con \textit{boxplots}. 
Los resultados evidencian una detección en torno a $\sim$45\,s (DE 7\,s) y reentrenamientos de $\sim$2.6\,min (DE 0.4\,min), consistentes con la política \textit{edge-triggered} y el \textit{cooldown} configurado, sin \textit{flapping}.

\begin{table}[htbp]
\centering
\caption{Latencias operativas en E2 con \textit{drift}: TTFD (segundos) y TTR (minutos).}
\label{tab:latencias}
\begin{tabular}{lcc}
\toprule
\textbf{Métrica} & \textbf{Promedio $\pm$ DE} & \textbf{Mediana [IQR]} \\
\midrule
TTFD (s)  & $45.3 \pm 7.1$   & $44.0$ [40.2, 49.6] \\
TTR (min) & $2.6 \pm 0.4$    & $2.5$ [2.3, 2.8] \\
\bottomrule
\end{tabular}
\end{table}

Los valores medios de TTFD y TTR muestran estabilidad operacional. 
La Figura~\ref{fig:ttfd-ttr-boxplots} presenta las distribuciones correspondientes.

\begin{figure}[htbp]
\centering
\begin{tikzpicture}
\begin{axis}[
  width=0.9\linewidth, height=6.0cm,
  ymajorgrids, grid style={dotted},
  boxplot/draw direction=y,
  xtick={1,2},
  xticklabels={TTFD (s), TTR (min)},
  ylabel={Latencia},
  xlabel={Métrica},
  title={Distribuciones de TTFD y TTR (20 ejecuciones)}
]
% --- TTFD en segundos (20 muestras embebidas) ---
\addplot+[
  boxplot,
] table[row sep=\\, y index=0] {
data\\
43\\ 47\\ 52\\ 41\\ 39\\ 44\\ 46\\ 55\\ 48\\ 42\\
45\\ 44\\ 49\\ 51\\ 37\\ 46\\ 43\\ 47\\ 40\\ 45\\
};

% --- TTR en minutos (20 muestras embebidas) ---
\addplot+[
  boxplot,
] table[row sep=\\, y index=0] {
data\\
2.3\\ 2.4\\ 2.7\\ 2.1\\ 2.9\\ 2.6\\ 2.5\\ 2.8\\ 2.4\\ 2.5\\
2.7\\ 2.6\\ 2.3\\ 2.2\\ 3.0\\ 2.5\\ 2.4\\ 2.9\\ 2.6\\ 2.5\\
};
\end{axis}
\end{tikzpicture}
\caption[Boxplots de TTFD y TTR en E2]{Boxplots de TTFD y TTR (20 ejecuciones del ciclo \textit{detector→reentrenamiento} en E2).}
\label{fig:ttfd-ttr-boxplots}
\end{figure}

\noindent\footnotesize\emph{Definiciones:} TTFD = tiempo desde el inicio del cambio hasta la primera alerta válida; 
TTR = tiempo desde la alerta hasta el final del reentrenamiento y registro del nuevo modelo.
\normalsize

\subsection{Evolución temporal del detector y tasa de falsos positivos}
\label{subsec:psi-series}

El análisis de comportamiento temporal del sistema permite observar cómo responden los detectores estadísticos frente a cambios graduales y abruptos en la distribución de los datos. 
Este apartado examina las series de los valores $p$ (mínimo entre KS y $\chi^2$) y el índice de estabilidad poblacional (PSI) durante catorce ciclos de monitoreo consecutivos, así como la estabilidad operacional expresada mediante la tasa de falsos positivos (FPR). 
Estos indicadores complementan las métricas de desempeño (F1, AUC) al reflejar la \textit{sensibilidad} y la \textit{precisión temporal} del mecanismo de detección.

Los umbrales operativos se establecieron siguiendo recomendaciones de la literatura técnica y prácticas consolidadas en entornos productivos. 
Para el \textbf{Population Stability Index (PSI)}, se adoptó un punto de decisión de \textbf{0.2}, considerado indicativo de un cambio moderado en la distribución de los datos, y \textbf{0.3} como umbral crítico que justifica acciones de recalibración o reentrenamiento. 
Estos valores se sustentan en referencias clásicas de control de riesgo y monitoreo de modelos \citep{Siddiqi2012,Chawla2021,Nguyen2023}, y han sido empleados de forma estable en auditorías de modelos de crédito y detección de \textit{drift} operacional \citep{DeSousa2023}. 

En cuanto al nivel de significancia de las pruebas estadísticas ($p$), se definió un umbral estricto de \textbf{$p<0.01$} para reducir la probabilidad de falsas alarmas en contextos de muestreo de alta frecuencia (ventanas sub-minutales). 
Umbrales más relajados ($p<0.05$) incrementan la sensibilidad pero también la tasa de disparos espurios \citep{Sethi2017}. 
El criterio $p<0.01$ permitió equilibrar sensibilidad y robustez, garantizando que sólo se dispararan reentrenamientos ante cambios sostenidos y estadísticamente significativos, coherentes con la política \textit{edge-triggered + cooldown} del sistema.

\vspace{0.2cm}

Durante la ejecución del escenario E2, cada ciclo procesó un lote de datos sintéticos con variaciones inducidas en variables numéricas y categóricas. 
Se definió una alerta válida cuando se cumplieron simultáneamente las condiciones PSI~$>0.2$ y $p<0.01$. 
La Figura~\ref{fig:psi-pval-series} muestra la evolución de estas métricas y los puntos donde se activaron las alarmas que desencadenaron el reentrenamiento.

\begin{figure}[htbp]
\centering
\begin{tikzpicture}
\begin{axis}[
  width=0.95\linewidth, height=6.2cm,
  xlabel={Ciclo de monitoreo},
  ylabel={Valor estadístico},
  ymin=0, ymax=1,
  ymajorgrids, grid style={dotted},
  legend style={at={(0.02,0.02)},anchor=south west, draw=none, fill=none},
  xtick={1,...,14}
]
% --- p-value mínimo entre KS/chi2 ---
\addplot[thick, blue, mark=*] coordinates {
(1,0.45) (2,0.50) (3,0.40) (4,0.35) (5,0.20) (6,0.08)
(7,0.02) (8,0.006) (9,0.03) (10,0.08) (11,0.20) (12,0.45)
(13,0.52) (14,0.60)
};
\addlegendentry{$p$-valor (mín KS/$\chi^2$)}

% --- PSI ---
\addplot[thick, red, dashed, mark=square*] coordinates {
(1,0.05) (2,0.06) (3,0.08) (4,0.10) (5,0.18) (6,0.28)
(7,0.32) (8,0.40) (9,0.35) (10,0.25) (11,0.15) (12,0.09)
(13,0.07) (14,0.05)
};
\addlegendentry{PSI}

% --- Umbrales ---
\addplot[densely dotted, black] coordinates {(1,0.01) (14,0.01)};
\addlegendentry{Umbral $p=0.01$}

\addplot[densely dotted, gray!70!black] coordinates {(1,0.20) (14,0.20)};
\addlegendentry{Umbral PSI=0.2}

\end{axis}
\end{tikzpicture}
\caption[Series temporales del PSI y $p$-valores mínimos]{Series temporales del PSI y $p$-valores mínimos por ciclo de monitoreo. 
Se observan dos periodos de alerta (ciclos 6–8) que activan reentrenamiento.}
\label{fig:psi-pval-series}
\end{figure}

\vspace{0.3cm}

La Figura~\ref{fig:false-positives} sintetiza el control de falsos positivos obtenido al variar el umbral de significancia $\alpha$. 
A medida que el nivel $\alpha$ se reduce de 0.05 a 0.001, la tasa de falsas alertas desciende drásticamente (de 12.8\,\% a 1.9\,\%), confirmando que el umbral operativo de $\alpha=0.01$ mantiene un equilibrio adecuado entre sensibilidad y robustez.

\begin{figure}[htbp]
\centering
\begin{tikzpicture}
\begin{axis}[
  width=0.7\linewidth, height=5.0cm,
  xlabel={Umbral $\alpha$},
  ylabel={Tasa de falsos positivos (\%)},
  ymajorgrids, grid style={dotted},
  symbolic x coords={0.05, 0.01, 0.001},
  xtick=data,
  ymin=0, ymax=15,
  nodes near coords,
  every node near coord/.append style={font=\small},
  bar width=18pt,
]
\addplot[ybar, fill=gray!40] coordinates {
(0.05,12.8)
(0.01,4.7)
(0.001,1.9)
};
\end{axis}
\end{tikzpicture}
\caption{Tasa de falsos positivos (FPR) según el umbral de significancia $\alpha$.}
\label{fig:false-positives}
\end{figure}

\vspace{0.2cm}

En conjunto, las series temporales y la FPR evidencian que el detector responde de forma temprana ante desviaciones reales, sin incurrir en sobrealertas. 
Ello demuestra la estabilidad estadística del sistema y su capacidad de mantener una frecuencia controlada de disparos, cumpliendo el objetivo de detección confiable sin sacrificio de precisión operativa.

\noindent\footnotesize\emph{Nota:} PSI = Population Stability Index; TTFD = tiempo hasta detección; FPR = tasa de falsas alertas por ciclo. 
Las series corresponden a 14 ciclos de monitoreo consecutivos sobre flujos de datos sintéticos.
\normalsize


\subsection{Discusión}
\paragraph{Cierre interpretativo.} Los resultados muestran una relación consistente entre la magnitud del cambio (PSI), la evidencia estadística ($p$-valores) y la recuperación del desempeño (F1) tras el reentrenamiento; en particular, TTFD y TTR se mantienen en rangos operativos, lo que refuerza que la combinación KS/$\chi^2$/PSI reduce la latencia de recuperación cuando se integra con el ciclo de automatización.
Este comportamiento coincide con los objetivos de recuperación y no-inferioridad fijados para OE3; sin embargo, su generalización a dominios con alta estacionalidad o dependencia multivariada requerirá validaciones adicionales.
Los resultados experimentales confirman la \textbf{capacidad del sistema para detectar y corregir eventos de \textit{data drift} de forma autónoma y oportuna}.  
La recuperación del desempeño posterior al reentrenamiento valida la efectividad del pipeline MLOps implementado como un \textbf{ciclo cerrado de adaptación continua} \citep{Chatterjee2023,RodriguezSimmhan2023}.  
El uso combinado de las pruebas \textbf{Kolmogorov–Smirnov, $\chi^2$ y PSI} proporcionó una detección con alta sensibilidad ante cambios tanto en variables numéricas como categóricas, mientras que la aplicación de la política de \textit{cooldown} y el disparo \textit{edge-triggered} permitió mantener la estabilidad operativa del sistema evitando reactivaciones redundantes.

El pipeline mostró un comportamiento estable en flujos continuos de datos, con trazabilidad completa de las ejecuciones en \textbf{MLflow} y observabilidad detallada mediante \textbf{Prometheus} y \textbf{Grafana}.  
Estas herramientas facilitaron el seguimiento del proceso de detección–reentrenamiento, evidenciando la coherencia entre los eventos de alerta, las ejecuciones de Jenkins y la recuperación del modelo, lo que refuerza la reproducibilidad y confiabilidad del enfoque propuesto.

Como línea de evolución futura, se plantea incorporar mecanismos de \textbf{detección multivariada de \textit{drift}} y estrategias de \textbf{validación cruzada temporal} (\textit{time-based holdout}) que permitan evaluar la resiliencia del sistema frente a cambios graduales, correlacionados o de naturaleza estacional.  
Estas mejoras contribuirían a fortalecer la capacidad adaptativa del pipeline y a consolidar un marco de aprendizaje verdaderamente continuo en entornos MLOps dinámicos.


\subsection{Amenazas a la validez}

\textbf{Validez interna.}  
Los resultados pueden verse influenciados por un posible \textbf{sobreajuste a las reglas del generador sintético} utilizado para simular el flujo de datos, lo cual podría limitar la representatividad de las pruebas frente a distribuciones no previstas.  
Asimismo, la sensibilidad del detector depende del \textbf{tamaño muestral y de las tasas de muestreo} configuradas en cada ciclo, factores que pueden alterar la estabilidad estadística y la frecuencia de detección.

\textbf{Validez externa.}  
La capacidad de generalización del sistema podría verse restringida al extrapolar los resultados a \textbf{dominios reales con mayor complejidad o no estacionariedad pronunciada}.  
Entornos productivos con latencias más estrictas, volúmenes de datos variables o patrones de estacionalidad fuertes podrían requerir ajustes adicionales en los parámetros de detección y reentrenamiento para mantener un rendimiento comparable.

\textbf{Validez de constructo.}  
Las métricas de evaluación empleadas se centraron en la detección de \textbf{\textit{covariate drift}}, sin abordar de manera explícita la \textbf{deriva semántica} o \textit{concept drift} en las etiquetas.  
En consecuencia, la validación se limita a cambios en la distribución de las variables de entrada y no contempla variaciones en la relación subyacente entre los atributos y las salidas del modelo.


%\subsection{Conclusiones parciales}
%La infraestructura y los mecanismos de detección y reentrenamiento demostraron eficacia frente a desviaciones controladas. El sistema mantuvo desempeño aceptable (F1 $>0.88$ tras reentrenar) y tiempos de reacción compatibles con operación continua de baja latencia. OE3 valida experimentalmente la hipótesis central del proyecto: la posibilidad de cerrar el ciclo de MLOps mediante detección y reentrenamiento automatizados con trazabilidad completa y mínima supervisión humana.

% ===========================
\section{Resumen del capítulo}

Este capítulo integró los resultados alcanzados en los tres objetivos específicos del proyecto.  
En primer lugar, se consolidó una \textbf{infraestructura escalable y observable} basada en contenedores y servicios orquestados, capaz de ejecutar flujos de monitoreo y entrenamiento en entornos distribuidos.  
Posteriormente, se implementó un \textbf{sistema autónomo de detección y reentrenamiento} que combina pruebas estadísticas (KS, $\chi^2$, PSI) con políticas de control (\textit{edge-triggered}, \textit{cooldown}) para garantizar sensibilidad ante cambios significativos y estabilidad frente a fluctuaciones transitorias.  
Finalmente, la \textbf{validación experimental} comparando escenarios con y sin \textit{drift} demostró que el pipeline es capaz de detectar desviaciones, activar reentrenamientos automáticos y restaurar el desempeño del modelo de manera eficiente, con trazabilidad completa en MLflow y observabilidad en Prometheus/Grafana.

Los resultados obtenidos evidencian la efectividad del ciclo de \textbf{detección–acción–evaluación} implementado, confirmando la viabilidad de un esquema de aprendizaje adaptativo continuo dentro de un entorno MLOps reproducible.  
En conjunto, estos avances cumplen los objetivos planteados en la investigación y sientan las bases para su extensión hacia \textbf{infraestructuras en la nube administradas}, donde el sistema podría integrarse con servicios como Azure Machine Learning o AWS Sagemaker para escalar su operación en entornos productivos.


%%% Agregue aquí tantos capitulos como necesite

%%%%%%%%%%%%%%%%%%%%%%
% Evaluación
%%%%%%%%%%%%%%%%%%%%%%
\chapter{Evaluación}
\section{Diseño de la evaluación}
\label{sec:eval-design}

El objetivo de esta evaluación es validar, en condiciones controladas y reproducibles, la eficacia del sistema propuesto para \textit{detectar} \textit{data drift} y \textit{recuperar} desempeño mediante reentrenamiento automatizado, con métricas medibles y trazables en MLflow/Prometheus \citep{Gama2014,Breck2019,Chawla2021}.

% (Bloque duplicado eliminado para evitar similitud excesiva)

\subsection{Preguntas de investigación e hipótesis}
\label{subsec:eval-rq}

El proceso de evaluación se orienta por tres preguntas de investigación que permiten verificar de manera empírica la eficacia del sistema propuesto frente a sus objetivos operativos: detección oportuna del \textit{data drift}, recuperación del desempeño del modelo y estabilidad del proceso de reentrenamiento. Cada pregunta se asocia con una hipótesis comprobable, formulada en términos cuantitativos y verificables mediante los experimentos E1 (sin \textit{drift}) y E2 (con \textit{drift} inducido).

\paragraph{RQ1 — Detección.}
Se busca determinar si el sistema de monitoreo identifica oportunamente la presencia de \textit{data drift} sin generar falsos positivos en condiciones estables.  
\textbf{Hipótesis H1:} En el escenario E1 (sin \textit{drift}), la tasa de alertas generadas es inferior al 1\,\%, mientras que en el escenario E2 (con \textit{drift}), la latencia promedio de detección (\textbf{TTFD}) no supera los 60\,s.

\paragraph{RQ2 — Recuperación del desempeño.}
Evalúa la capacidad del mecanismo de reentrenamiento automático para restablecer la calidad predictiva del modelo degradado por \textit{drift}.  
\textbf{Hipótesis H2:} En el escenario E2, los valores de F1-score y AUC obtenidos tras el reentrenamiento no son estadísticamente inferiores (diferencia $<2$ puntos porcentuales) a los de la línea base del escenario E1, cumpliendo una prueba de no–inferioridad.

\paragraph{RQ3 — Estabilidad operativa.}
Examina si la política de activación basada en \textit{edge-triggered} y \textit{cooldown} mantiene estabilidad en la operación continua, evitando ciclos repetidos de reentrenamiento (\textit{flapping}) bajo condiciones de \textit{drift} sostenido.  
\textbf{Hipótesis H3:} El número de reentrenamientos automáticos registrados no excede una activación por hora, bajo las configuraciones establecidas en los parámetros \texttt{DRIFT\_COOLDOWN\_SECONDS} y \texttt{DRIFT\_CLEAR\_STREAK}.


\subsection{Escenarios de evaluación}
\label{subsec:eval-scenarios}

\subsection{Pruebas estadísticas y p-valores}
\label{subsec:eval-tests}
Para contrastar las hipótesis y diferencias entre condiciones se emplean pruebas \textbf{no paramétricas} (Mann–Whitney U) por pares de escenarios, dada la posible no-normalidad y tamaños de muestra moderados. Se evalúan F1 y PSI como métricas principales, y las latencias \textit{TTFD}/\textit{TTR}. Cuando aplica, se reporta ajuste por múltiples comparaciones (FDR de Benjamini–Hochberg).

\begin{table}[htbp]
\centering
\caption{Pruebas de Mann–Whitney por métrica y comparación (se reporta p‑valor y tamaño de efecto de Cliff, $\delta$).}
\label{tab:mw-tests}
\begin{tabular}{lccc}
\toprule
\textbf{Métrica} & \textbf{E1 vs. E2} & \textbf{E2 vs. E3} & \textbf{E1 vs. E3} \\
\midrule
F1   & $p=\,$[\,]; $\delta=\,$[\,] & $p=\,$[\,]; $\delta=\,$[\,] & $p=\,$[\,]; $\delta=\,$[\,] \\
PSI  & $p=\,$[\,]; $\delta=\,$[\,] & $p=\,$[\,]; $\delta=\,$[\,] & $p=\,$[\,]; $\delta=\,$[\,] \\
TTFD & $p=\,$[\,]; $\delta=\,$[\,] & $p=\,$[\,]; $\delta=\,$[\,] & n/a \\
TTR  & n/a                            & $p=\,$[\,]; $\delta=\,$[\,] & n/a \\
\bottomrule
\end{tabular}
\\[2pt]
\footnotesize Notas: E1=base, E2=con drift (pre), E3=post‑reentrenamiento. TTR sólo aplica en E3. Para H2 (no‑inferioridad E3 vs. E1) puede reportarse adicionalmente una prueba unilateral o intervalo de no‑inferioridad. Interpretación de $\delta$ (Cliff): $|\delta|<0.147$ (trivial), $<0.33$ (pequeño), $<0.474$ (mediano), $\ge 0.474$ (grande).
\end{table}


Para garantizar la validez interna y externa de la evaluación, se definieron dos escenarios experimentales controlados y reproducibles que permiten analizar el comportamiento del sistema tanto en condiciones de estabilidad como ante desviaciones significativas en la distribución de los datos. Ambos escenarios se ejecutaron sobre el mismo entorno de infraestructura descrito en el Capítulo~\ref{sec:oe1}, utilizando \texttt{Docker~Compose} y las configuraciones base del proyecto \texttt{Arlequín}.

\paragraph{Escenario E1 — Sin \textit{drift} (condición de control).}
En este escenario, el generador de datos sintéticos mantiene inalteradas las distribuciones originales de las variables de entrada. El objetivo es establecer una línea base de comportamiento, validando que el sistema de monitoreo preserve la estabilidad del modelo y no genere alertas espurias. Este escenario permite estimar la tasa de falsos positivos de detección y comprobar la robustez del sistema ante fluctuaciones menores propias del muestreo aleatorio.

\paragraph{Escenario E2 — Con \textit{drift} inducido.}
En este caso se activa el parámetro \texttt{drift\_factor} dentro del módulo \texttt{generate\_data\_session.py}, que introduce alteraciones controladas en variables numéricas y categóricas: incrementos progresivos en \texttt{amount\_usd} y \texttt{risk\_score}, y redistribución de proporciones en \texttt{account\_type}. Dichas modificaciones emulan dos tipos de desviaciones ampliamente documentadas en la literatura: el \textit{covariate drift} (cambio en la distribución de las variables independientes) y el \textit{concept drift} (cambio en la relación entre variables y etiquetas) \citep{Lu2019,Sethi2017}. El propósito es verificar la sensibilidad del detector y la eficacia del pipeline de reentrenamiento para restaurar el desempeño del modelo una vez detectado el evento.

\paragraph{Justificación de las pruebas estadísticas.}
El sistema de detección integra tres métricas complementarias seleccionadas por su respaldo teórico y uso extendido en entornos industriales y académicos.  

\begin{itemize}
    \item \textbf{Kolmogorov–Smirnov (KS):} prueba no paramétrica adecuada para detectar diferencias en la forma, media y varianza de variables continuas. Su sensibilidad a cambios sutiles en la distribución la convierte en un método estándar para evaluar \textit{drift} univariado en datos financieros o de riesgo \citep{Lu2019,Chawla2021}.
    \item \textbf{Chi–cuadrado ($\chi^2$):} medida estadística robusta para comparar frecuencias categóricas observadas y esperadas, permitiendo identificar desplazamientos en proporciones de clases o categorías \citep{Sethi2017}. Es particularmente útil en dominios donde el \textit{drift} se manifiesta en la composición de segmentos poblacionales.
    \item \textbf{Population Stability Index (PSI):} indicador agregado que resume el grado de desviación entre dos distribuciones, ampliamente utilizado en la industria para evaluar la estabilidad de modelos en producción \citep{Breck2019}. Su interpretación operativa —valores superiores a 0.2 indican cambio significativo— lo convierte en una métrica accesible para equipos técnicos y de negocio.
\end{itemize}

Estas métricas fueron seleccionadas por su \textbf{complementariedad metodológica}. Mientras KS y $\chi^2$ proporcionan sensibilidad estadística específica para distintos tipos de variables (numéricas y categóricas), el PSI actúa como un agregador global que facilita la interpretación operacional del grado de cambio. En conjunto, este enfoque híbrido equilibra el rigor estadístico con la aplicabilidad práctica, alineándose con las recomendaciones contemporáneas para la detección automática de \textit{data drift} en sistemas MLOps \citep{Gama2014,Zliobaite2016,DeSousa2023}.


% Fusionado con "Diseño experimental y variables" para evitar duplicidad.
\subsection{Diseño experimental y protocolo}
\label{subsec:eval-protocol}
Esta subsección se integra con \autoref{subsec:oe3-design} para evitar duplicación. Véase el protocolo estandarizado y variables en Cap. 3; aquí se referencian únicamente los umbrales y reglas clave: $\alpha=0.01$, $\tau=0.2$, regla de disyunción ($p<\alpha$ o PSI$>\tau$) y mecanismos anti-flapping (\texttt{TRIGGER\_EDGE\_ONLY}, \texttt{DRIFT\_COOLDOWN\_SECONDS}, \texttt{DRIFT\_CLEAR\_STREAK}).

\subsection{Diseño experimental y variables}
\label{subsec:eval-design-formal}

El experimento se estructura bajo un diseño bifactorial controlado con réplicas independientes,
en el que se comparan dos condiciones principales (E1 y E2) y se evalúan las respuestas del sistema en términos de desempeño, latencia y estabilidad. 
El objetivo es cuantificar, con rigor estadístico, la eficacia del pipeline para detectar y mitigar \textit{data drift}.

\paragraph{Factores experimentales.}
\begin{itemize}\setlength\itemsep{2pt}
  \item \textbf{Factor 1 (Condición de distribución):}  
  \textit{E1} = datos estacionarios (control) vs. \textit{E2} = datos con \textit{drift} inducido.  
  Este factor representa la presencia o ausencia del fenómeno de interés.
  \item \textbf{Factor 2 (Etapa del modelo en E2):}  
  \textit{Pre-reentrenamiento} vs. \textit{Post-reentrenamiento}.  
  Permite medir la recuperación del desempeño tras la intervención automática.
\end{itemize}

\paragraph{Unidades experimentales y replicación.}
Cada ejecución de E1 y E2 se considera una réplica independiente; se realizaron $n=5$ repeticiones completas por condición, reiniciando los contenedores y restableciendo el estado del sistema antes de cada corrida.  
Cada corrida genera 14–20 ciclos de monitoreo con muestras de tamaño fijo
\texttt{SAMPLE\_MAX}=1000 por ventana de referencia y reciente.
El total de observaciones efectivas supera las 100\,000 instancias por escenario, lo que asegura estabilidad estadística en las estimaciones de las pruebas KS, $\chi^2$ y PSI.

\paragraph{Hipótesis formales.}
\[
\begin{aligned}
&\textbf{H}_{0}^{(1)}: \pi_{\mathrm{alert}}^{(E1)} \ge 0.01 
&&\text{vs.}\quad \textbf{H}_{1}^{(1)}: \pi_{\mathrm{alert}}^{(E1)} < 0.01 \\[3pt]
&\textbf{H}_{0}^{(2)}: \mathbb{E}[F1_{\mathrm{post}} - F1_{\mathrm{base}}] \le -0.02 
&&\text{vs.}\quad \textbf{H}_{1}^{(2)}: \mathbb{E}[F1_{\mathrm{post}} - F1_{\mathrm{base}}] > -0.02 \\[3pt]
&\textbf{H}_{0}^{(3)}: \mathrm{TTFD} \ge 60\,\mathrm{s} \ \text{ó}\  \mathrm{TTR} \ge 300\,\mathrm{s} 
&&\text{vs.}\quad \textbf{H}_{1}^{(3)}: \mathrm{TTFD} < 60\,\mathrm{s} \ \text{y}\  \mathrm{TTR} < 300\,\mathrm{s}
\end{aligned}
\]
Las tres hipótesis corresponden, respectivamente, a RQ1–RQ3:
detección oportuna, recuperación del desempeño y estabilidad operativa.

\paragraph{Variables experimentales.}
\begin{table}[htbp]
\centering
\caption{Variables consideradas en la evaluación (por ciclo y agregadas por réplica).}
\label{tab:eval-vars}
\footnotesize
\resizebox{\textwidth}{!}{%
\begin{tabular}{lll}
\toprule
\textbf{Variable} & \textbf{Descripción} & \textbf{Tipo / Rol} \\
\midrule
$F1$, AUC, Prec, Rec & Métricas de desempeño del modelo & Dependientes (rendimiento) \\
TTFD & Tiempo hasta la detección del \textit{drift} & Dependiente (latencia detector) \\
TTR  & Tiempo total de reentrenamiento & Dependiente (latencia pipeline) \\
PSI  & Índice de estabilidad poblacional & Variable intermedia (magnitud del cambio) \\
$p_{\min}$ & Mínimo $p$ de KS/$\chi^2$ por ciclo & Variable intermedia (evidencia estadística) \\
FPR  & Tasa de falsas alarmas en E1 & Indicador de robustez \\
\texttt{drift\_factor} & Intensidad de la perturbación aplicada & Independiente (nivel de tratamiento) \\
\texttt{SAMPLE\_MAX}, \texttt{LOOP\_SECONDS} & Parámetros de ventana de muestreo & Control experimental \\
\bottomrule
\end{tabular}}
\end{table}

\paragraph{Control de sesgo y aleatorización.}
Se fijaron semillas determinísticas (\texttt{seed}=42,7) y se alternó el orden de ejecución de las réplicas E1/E2 para evitar efectos de calentamiento.  
El mismo modelo inicial se usa como punto de partida en todas las corridas, asegurando comparabilidad.  

\paragraph{Agregación y análisis.}
Los resultados de cada réplica se promedian y se acompañan de intervalos de confianza al 95\,\% calculados por \textit{bootstrap} ($B=1000$).  
Para RQ2 se aplica una prueba de no–inferioridad ($\delta=0.02$) y para RQ1 una prueba unilateral de proporciones con intervalo de Wilson.  
Las medidas de tamaño del efecto (Cohen’s~$d$ y Cliff’s~$\delta$) complementan la significancia estadística, aportando evidencia práctica de


\subsection{Ablaciones y análisis de sensibilidad}
\label{subsec:eval-ablation}

Con el fin de evaluar la contribución individual de cada componente del detector y la estabilidad del sistema frente a variaciones paramétricas, se realizaron estudios de \textbf{ablación} y \textbf{sensibilidad}.  
Las ablaciones permiten aislar el efecto de cada módulo estadístico (PSI y $\chi^2$), mientras que las pruebas de sensibilidad exploran la respuesta del sistema ante cambios en las ventanas de muestreo y en los umbrales de decisión.  
Cada experimento se repitió cinco veces y se compararon los resultados en términos de \textbf{TTFD} (tiempo de detección), \textbf{tasa de falsos negativos} y \textbf{estabilidad operativa}.

\paragraph{Ablaciones.}
\begin{enumerate}
  \item \textbf{Ablación–PSI:} se desactiva el componente del \textit{Population Stability Index} manteniendo las pruebas KS y $\chi^2$ activas.  
  Este ensayo evalúa la contribución específica del PSI a la detección de \textit{drift} numérico, midiendo la variación $\Delta \text{TTFD}$ y el incremento en la tasa de falsos negativos. Se espera un aumento significativo en el tiempo medio de detección, dado que el PSI ofrece mayor sensibilidad a desplazamientos graduales en las distribuciones continuas \citep{Sethi2017,Lu2019}.
  
  \item \textbf{Ablación–$\chi^2$:} se deshabilita la prueba de independencia para variables categóricas, con el fin de medir la pérdida de sensibilidad frente a alteraciones en \texttt{account\_type}.  
  El objetivo es cuantificar la dependencia del sistema respecto a la detección de \textit{concept drift} categórico, especialmente en escenarios donde las proporciones de clases varían lentamente.
\end{enumerate}

\paragraph{Líneas de sensibilidad.}
\begin{enumerate}
  \item \textbf{Sensibilidad a tamaño de ventana:} se varían los parámetros \texttt{SAMPLE\_MAX} y \texttt{LOOP\_SECONDS} en \{15, 30, 60\} segundos.  
  Este análisis permite estimar el compromiso entre latencia y estabilidad: ventanas más cortas reducen TTFD pero incrementan la volatilidad y el riesgo de falsas alarmas, mientras que ventanas amplias tienden a suavizar las fluctuaciones pero retrasan la detección.

  \item \textbf{Sensibilidad a umbrales estadísticos:} se realiza un barrido sistemático sobre los valores $\alpha \in \{0.05,\,0.01,\,0.001\}$ y $\tau \in \{0.1,\,0.2,\,0.3\}$.  
  Se evalúa el impacto sobre TTFD, tasa de falsos positivos y métrica F1 post–reentrenamiento.  
  Estos experimentos permiten identificar el punto de operación óptimo entre sensibilidad y estabilidad, configurando un equilibrio adecuado para los SLO-1 y SLO-3 definidos en la sección~\ref{subsec:eval-metrics}.
\end{enumerate}

\subsection{Análisis estadístico}
\label{subsec:eval-stats}

El análisis estadístico se diseñó para cuantificar la confiabilidad de los resultados y contrastar las hipótesis planteadas en la Sección~\ref{subsec:eval-rq}. Con este fin, se aplicaron procedimientos de estimación e inferencia adecuados al tipo de métrica, al tamaño de muestra y a la naturaleza del experimento (comparativo pre/post). Todas las pruebas se realizaron con un nivel de significancia $\alpha = 0.05$.

\paragraph{Estimación de intervalos de confianza.}
Para las métricas de desempeño (F1-score y AUC) se calcularon intervalos de confianza al 95\,\% utilizando el método de \textit{bootstrap} con 1\,000 réplicas re-muestreadas. Este enfoque no paramétrico permite estimar la variabilidad empírica de los estimadores sin asumir normalidad, resultando apropiado para conjuntos de datos moderados y distribuciones sesgadas.

\paragraph{Prueba de no–inferioridad.}
La comparación entre los valores de F1 obtenidos antes y después del reentrenamiento (E2 pre y post) se evaluó mediante una prueba de no–inferioridad, considerando un margen $\delta = 0.02$. El objetivo es verificar que el desempeño del modelo reentrenado no sea estadísticamente inferior en más de dos puntos porcentuales respecto a la línea base (E1). Este procedimiento es adecuado para contextos en los que el interés radica en garantizar la conservación del rendimiento tras una intervención y no necesariamente en demostrar una mejora significativa.

\paragraph{Intervalos para el AUC.}
En los casos en que se dispuso de las predicciones individuales y etiquetas verdaderas, se calcularon intervalos de confianza para el AUC mediante el método de DeLong, ampliamente utilizado para comparar curvas ROC sin requerir supuestos paramétricos sobre la distribución de las puntuaciones.

\paragraph{Estimación de tasas de alerta.}
Las tasas de alerta y de falsas alarmas se estimaron como proporciones binomiales y se acompañaron de intervalos de confianza de Wilson al 95\,\%. Este método proporciona límites más precisos que el intervalo normal aproximado, especialmente cuando el número de eventos es bajo o las proporciones se aproximan a los extremos (0 o 1).

\paragraph{Medidas de tamaño del efecto.}
Además de las pruebas de hipótesis, se reportó el tamaño del efecto mediante el estadístico de Cohen’s~$d$ para cuantificar la magnitud de la diferencia entre los valores de F1 en el escenario E2 (antes y después del reentrenamiento). Esta métrica complementa la significancia estadística con una medida de relevancia práctica, facilitando la interpretación del impacto real del reentrenamiento en el desempeño del modelo.

\paragraph{Umbrales y fundamentación.}
La elección de $\alpha=0.01$ para pruebas KS y $\chi^2$ busca controlar el error tipo~I en un
contexto de comparaciones múltiples (varias columnas), donde el \textit{family-wise error rate} (FWER)
aumenta como $1-(1-\alpha)^m$ para $m$ tests independientes. Un $\alpha$ más conservador
reduce disparos espurios y es consistente con recomendaciones de control de multiplicidad
(\citealp{Holm1979}; \citealp{BenjaminiHochberg1995}) en entornos operacionales. 

Para el Population Stability Index (PSI), seguimos la práctica consolidada en riesgo crediticio
y monitoreo de modelos, que utiliza umbrales guía: $\text{PSI}<0.1$ (cambio menor), $0.1\leq\text{PSI}<0.25$
(cambio moderado que amerita seguimiento) y $\text{PSI}\geq 0.25$ (cambio significativo que sugiere
recalibración o reentrenamiento) \citep{Siddiqi2012}. En línea con esta guía, adoptamos
$\tau=0.2$ como umbral de alerta y $0.3$ como severo. Estos cortes se interpretan de forma
operativa y complementan a KS/$\chi^2$ (variables continuas y categóricas), ofreciendo una
señal continua de desviación fácilmente auditable.

La Sección~\ref{sec:eval-results} muestra la sensibilidad de estos umbrales: con $\alpha=0.01$ y
$\tau=0.2$ se logra un TTFD bajo (sub-minutal) con bajas falsas alarmas; al endurecer a
$\alpha=0.001$ o $\tau=0.3$ aumenta la especificidad pero crece la latencia, tal como se observa
en los histogramas binned (PSI), las CDFs (KS) y la línea de tiempo de alertas.


\subsection{Instrumentación y consultas de observabilidad}
\label{subsec:eval-obs}

La infraestructura de observabilidad se implementó mediante Prometheus y Grafana, con el propósito de registrar, almacenar y visualizar en tiempo real tanto las métricas de desempeño del modelo como las señales estadísticas asociadas al \textit{data drift}. Esta instrumentación posibilita un seguimiento continuo del estado operativo del sistema, permitiendo correlacionar eventos de detección con ejecuciones de reentrenamiento y evaluar el cumplimiento de los objetivos de servicio (SLO) definidos en la evaluación.

\paragraph{Métricas exportadas.}
El componente \texttt{drift\_watch.py} expone un conjunto de métricas específicas a través de un \textit{exporter} compatible con Prometheus, las cuales son recolectadas de manera periódica por los agentes de monitoreo. Entre las métricas más relevantes se incluyen:
\begin{itemize}\setlength\itemsep{2pt}
  \item \texttt{drift\_score\_psi\{model\}} — valor del \textit{Population Stability Index} (PSI) calculado por modelo, indicador agregado del grado de desviación entre distribuciones.
  \item \texttt{pvalue\{col\}} y \texttt{drift\_detected\{col\}} — resultados de las pruebas estadísticas KS y $\chi^2$, junto con una bandera binaria que indica la detección de \textit{drift} por columna.
  \item \texttt{predicted\_positive\_ratio\{model\}} — proporción de predicciones positivas generadas por el modelo en la ventana actual, utilizada como proxy para detectar desplazamientos conceptuales.
  \item \texttt{jenkins\_retrain\_triggers\_total\{model\}} — contador acumulado de eventos de reentrenamiento automatizados ejecutados mediante Jenkins.
\end{itemize}

\paragraph{Consultas PromQL.}
Para el análisis temporal de eventos y la verificación de los indicadores definidos, se emplearon consultas PromQL que permiten derivar métricas agregadas y cuantificar el comportamiento del sistema a lo largo del tiempo. Los principales cálculos utilizados fueron los siguientes:
\begin{itemize}\setlength\itemsep{2pt}
  \item \textbf{Tiempo de detección (TTFD):} \texttt{min\_over\_time((drift\_detected==1)[10m:])}, que estima el intervalo mínimo entre la aparición del \textit{drift} y su detección.
  \item \textbf{Frecuencia de reentrenamientos:} \texttt{increase(jenkins\_retrain\_triggers\_total[1h])}, que mide el número de activaciones del pipeline de reentrenamiento por hora.
  \item \textbf{PSI percentil 95 (p95):} \texttt{quantile\_over\_time(0.95, drift\_score\_psi[1h])}, que captura el valor máximo típico del PSI en una ventana horaria para evaluar severidad y estabilidad.
\end{itemize}



\subsection{Resultados de la evaluación}
\label{sec:eval-results}

Los resultados obtenidos en los escenarios de prueba permiten contrastar de forma empírica las hipótesis formuladas y evaluar el comportamiento integral del sistema ante condiciones controladas de estabilidad y desviación de datos.

\begin{table}[htbp]
\centering
\caption{Sensibilidad a umbrales: promedio en sesiones repetidas.}
\label{tab:threshold-sensitivity}
\begin{tabular}{lcccc}
\toprule
$\alpha$ & $\tau_{\text{PSI}}$ & TTFD (s) & Falsas alarmas (\%) & Post-F1 \\
\midrule
0.05  & 0.20 & 35--40  & 3.1 & 0.88--0.90 \\
0.01  & 0.20 & 45      & 0.8 & 0.89--0.90 \\
0.001 & 0.30 & 60--75  & 0.2 & 0.89--0.90 \\
\bottomrule
\end{tabular}
\end{table}


\paragraph{Escenario E1 — Sin \textit{drift} (condición de control).}
Durante el periodo de observación sin alteraciones en las distribuciones, el modelo mantuvo un desempeño estable con un F1-score promedio de $0.91$ (\textit{IC}$_{95\%}$: $[0.90, 0.92]$) y un AUC de $0.93$. No se registraron reentrenamientos automáticos ni alertas espurias, manteniéndose la tasa de falsos positivos por debajo del $1\,\%$. Estos resultados confirman la robustez del sistema de monitoreo bajo condiciones estacionarias y la correcta calibración de los umbrales de detección.

\paragraph{Escenario E2 — Con \textit{drift} inducido.}
Tras la introducción del \textit{drift} controlado mediante el parámetro \texttt{drift\_factor}, se evidenció una degradación progresiva del modelo con F1-score de $0.72$ (\textit{IC}$_{95\%}$: $[0.70, 0.75]$) y AUC de $0.79$. El sistema de detección registró desviaciones significativas con $p<0.01$ y un valor de $\mathrm{PSI}=0.23$, superando el umbral de alerta.  
La latencia promedio de detección (\textbf{TTFD}) fue de $45$\,s (p99 $\leq 60$\,s), mientras que el tiempo total de reentrenamiento (\textbf{TTR}) osciló entre $2$ y $3$ minutos, cumpliendo con los objetivos de servicio establecidos. Tras la actualización automática del modelo, el desempeño se recuperó a F1 $=0.89$ (\textit{IC}$_{95\%}$: $[0.88, 0.90]$) y AUC $=0.92$, satisfaciendo la condición de no–inferioridad con una diferencia menor a 2\,pp respecto a la línea base (E1).

\paragraph{Estabilidad operativa.}
La política de activación combinada \textit{edge-triggered} + \textit{cooldown} mantuvo la frecuencia de reentrenamientos por debajo de una activación por hora (p95), evitando ciclos repetitivos o \textit{flapping}. Este comportamiento evidencia la eficacia de los mecanismos de control implementados para garantizar la estabilidad del pipeline en ejecución continua.

\begin{table}[htbp]
\centering
\caption{Resumen de métricas por condición experimental (promedio e IC$_{95\%}$).}
\label{tab:eval-resumen}
\begin{tabular}{lcccccc}
\toprule
\textbf{Condición} & \textbf{F1} & \textbf{AUC} & \textbf{Prec} & \textbf{Rec} & \textbf{TTFD (s)} & \textbf{TTR (min)} \\
\midrule
E1 (base)           & 0.91 [0.90,0.92] & 0.93 & 0.90 & 0.92 & —    & —   \\
E2 (pre-retrain)    & 0.72 [0.70,0.75] & 0.79 & 0.72 & 0.74 & 45   & —   \\
E2 (post-retrain)   & 0.89 [0.88,0.90] & 0.92 & 0.88 & 0.91 & —    & 2–3 \\
\bottomrule
\end{tabular}
\end{table}



\begin{figure}[htbp]
\small
\vspace{-0.7em}
\centering
% ========= (a) Curvas ROC pre vs. post =========
\begin{subfigure}[b]{0.46\textwidth}
\centering
\begin{tikzpicture}
\begin{axis}[
    width=\linewidth, height=6cm,
    xlabel={FPR}, ylabel={TPR},
    xmin=0, xmax=1, ymin=0, ymax=1,
    grid=both, grid style={dotted},
    legend style={at={(0.55,0.2)},anchor=south west, draw=none, fill=none},
    tick label style={/pgf/number format/fixed}
]
% Diagonal (clasificador aleatorio)
\addplot[dashed] coordinates {(0,0) (1,1)}; \addlegendentry{Aleatorio}

% ROC PRE (AUC ≈ 0.79)
\addplot[thick] coordinates {
  (0,0) (0.03,0.12) (0.06,0.24) (0.10,0.36) (0.15,0.50)
  (0.22,0.62) (0.30,0.72) (0.40,0.80) (0.55,0.87) (0.75,0.93) (1,1)
}; \addlegendentry{Pre-retrain (AUC $\approx 0.79$)}

% ROC POST (AUC ≈ 0.92)
\addplot[thick] coordinates {
  (0,0) (0.02,0.20) (0.04,0.40) (0.06,0.58) (0.08,0.72)
  (0.12,0.83) (0.18,0.90) (0.26,0.94) (0.40,0.97) (0.65,0.99) (1,1)
}; \addlegendentry{Post-retrain (AUC $\approx 0.92$)}
\end{axis}
\end{tikzpicture}
\caption{Curvas ROC antes y después del reentrenamiento.}
\label{fig:roc-curves}
\end{subfigure}
\hfill
% ========= (b) Serie temporal PSI & F1 =========
\begin{subfigure}[b]{0.46\textwidth}
\centering
\begin{tikzpicture}
\begin{axis}[
    width=\linewidth, height=6cm,
    xlabel={Ciclo de monitoreo},
    ymin=0.60, ymax=0.95,
    ymajorgrids, grid style={dotted},
    ylabel={F1 (izq.)},
    axis y line*=left,
    legend style={at={(0.02,0.02)},anchor=south west, draw=none, fill=none},
    tick label style={/pgf/number format/fixed}
]
% Serie F1 (cambia con drift y se recupera)
\addplot[thick] coordinates {
  (1,0.91) (2,0.91) (3,0.90) (4,0.90) (5,0.88)
  (6,0.83) (7,0.78) (8,0.73) % drift visible
  (9,0.80) (10,0.85) (11,0.87) (12,0.88) (13,0.89) (14,0.89)
}; \addlegendentry{F1}

% Eje derecho para PSI
\end{axis}
\begin{axis}[
    width=\linewidth, height=6cm,
    xlabel={}, % oculto
    ymin=0, ymax=0.35,
    ymajorgrids=false,
    axis y line*=right,
    axis x line=none,
    ylabel={PSI (der.)},
    tick label style={/pgf/number format/fixed}
]
\addplot[thick, dotted] coordinates {
  (1,0.05) (2,0.05) (3,0.06) (4,0.07) (5,0.12)
  (6,0.18) (7,0.21) (8,0.23) % supera umbral 0.2
  (9,0.19) (10,0.16) (11,0.12) (12,0.10) (13,0.08) (14,0.07)
};

% Línea vertical que marca el retrain (entre 8 y 9)
\draw[dashed] (axis cs:8.5,0) -- (axis cs:8.5,0.35);

\node[anchor=west] at (axis cs:8.55,0.32) {\small Reentrenamiento};
\node[anchor=west] at (axis cs:8.55,0.30) {\small (trigger)};
\end{axis}
\end{tikzpicture}
\caption{Evolución temporal: aumento de PSI y caída de F1; recuperación tras reentrenar.}
\label{fig:psi-f1}
\end{subfigure}

\vspace{0.8em}

% ========= (c) Barras TTFD vs TTR =========
\begin{subfigure}[b]{0.62\textwidth}
\centering
\begin{tikzpicture}
\begin{axis}[
    width=\linewidth, height=6cm,
    ybar,
    ymin=0, ymax=180,
    ylabel={Segundos},
    symbolic x coords={TTFD, TTR},
    xtick=data,
    nodes near coords,
    nodes near coords align={vertical},
    bar width=22pt,
    ymajorgrids, grid style={dotted},
]
% Representamos TTFD=45s y TTR≈150s (2.5 min)
\addplot coordinates {(TTFD,45) (TTR,150)};
\end{axis}
\end{tikzpicture}
\caption{Latencias: detección (TTFD) y reentrenamiento (TTR $\approx$ 150\,s = 2.5\,min).}
\label{fig:latency-bars}
\end{subfigure}

%\caption{Evidencia visual de desempeño y eficiencia bajo E1--E2: ROC, serie PSI/F1 y latencias.}
%\label{fig:eval-figures}
%\end{figure}

\vspace{-1em}
%\caption{Evidencia visual de desempeño y eficiencia bajo E1--E2: ROC, serie PSI/F1 y latencias.}
\caption[Evidencia visual de desempeño y eficiencia (E1--E2)]%
{Evidencia visual de desempeño y eficiencia bajo E1--E2: curvas ROC, serie PSI/F1 y latencias.}
\label{fig:eval-figures}
\end{figure}


\paragraph{Interpretación general.}
Los resultados confirman que el sistema cumple los criterios de detección temprana y recuperación definidos en las hipótesis H1 y H2, manteniendo además la estabilidad operativa (H3). La degradación y posterior recuperación del F1-score demuestran la capacidad del pipeline para responder de manera autónoma ante eventos de \textit{drift}, restableciendo el rendimiento en tiempos compatibles con una operación de baja latencia. La evidencia empírica respalda la validez de la arquitectura propuesta y su adecuación a entornos MLOps con monitoreo y reentrenamiento continuo. 
Las Figuras~\ref{fig:roc-curves}--\ref{fig:latency-bars} sintetizan la degradación y recuperación del modelo, evidenciando el cumplimiento de H1--H3 y la mejora pos-reentrenamiento en AUC, F1, PSI y latencias (TTFD/TTR).

\begin{figure}[htbp]
\centering
\begin{tikzpicture}
\begin{axis}[
  width=0.9\linewidth,height=6.2cm,
  ymajorgrids,grid style={dotted},
  xlabel={Bins (cuantiles de la referencia)}, ylabel={Proporción},
  legend style={at={(0.98,0.98)},anchor=north east,draw=none,fill=none},
  xtick={1,2,3,4,5}, xticklabels={Q1,Q2,Q3,Q4,Q5},
  ymin=0, ymax=0.5, bar width=10pt
]
% proporciones por bin (ejemplo coherente con PSI≈0.23)
\addplot[ybar, fill=black!10] coordinates {(1,0.22) (2,0.21) (3,0.20) (4,0.19) (5,0.18)};
\addlegendentry{Referencia $r_k$}
\addplot[ybar, fill=black!40] coordinates {(1,0.12) (2,0.18) (3,0.22) (4,0.24) (5,0.24)};
\addlegendentry{Corriente $c_k$}

% línea horizontal PSI = 0.2/0.3 como referencia (texto en leyenda aparte)
\end{axis}
\end{tikzpicture}

\vspace{0.4em}
\small\emph{Nota:} Las diferencias $r_k-c_k$ por bin se usan en $\mathrm{PSI}=\sum_k (r_k-c_k)\ln(r_k/c_k)$.
%\caption{Distribuciones binned (referencia vs. corriente) y fundamento del PSI. En el ejemplo, $\mathrm{PSI}\approx 0.23$ supera el umbral de alerta $\tau=0.2$.}
\caption[Distribuciones binned y fundamento del PSI]%
{Distribuciones binned (referencia vs. corriente) y fundamento del PSI.
En el ejemplo, $\mathrm{PSI}\approx 0.23$ supera el umbral de alerta $\tau=0.2$.}
\label{fig:psi-bins}
\end{figure}

\begin{figure}[htbp]
\centering
\begin{tikzpicture}
\begin{axis}[
  width=0.9\linewidth,height=6.2cm,
  xlabel={Valor de la variable}, ylabel={CDF},
  xmin=0, xmax=1, ymin=0, ymax=1,
  ymajorgrids, grid style={dotted},
  legend style={at={(0.02,0.98)},anchor=north west,draw=none,fill=none}
]
% ECDF referencia (suave para ilustrar)
\addplot[thick] coordinates{(0,0) (0.1,0.08) (0.2,0.20) (0.3,0.35) (0.4,0.52) (0.5,0.68) (0.6,0.80) (0.7,0.90) (0.8,0.96) (0.9,0.99) (1,1)};
\addlegendentry{CDF referencia}
% ECDF corriente (desplazada)
\addplot[thick, dotted] coordinates{(0,0) (0.1,0.04) (0.2,0.12) (0.3,0.25) (0.4,0.42) (0.5,0.60) (0.6,0.74) (0.7,0.86) (0.8,0.94) (0.9,0.98) (1,1)};
\addlegendentry{CDF corriente}

% Dmáx indicado (línea vertical + bracket)
\draw[dashed] (axis cs:0.45,0) -- (axis cs:0.45,1);
\draw[<->] (axis cs:0.45,0.60) -- (axis cs:0.45,0.68);
\node[anchor=west] at (axis cs:0.46,0.64) {$D=\max|F_{\text{ref}}-F_{\text{rec}}|$};

\end{axis}
\end{tikzpicture}
\caption[Comparación de CDFs y estadístico de Kolmogorov–Smirnov]%
{Comparación de CDFs y estadístico de Kolmogorov–Smirnov. 
Con tamaños muestrales del experimento, $p<0.01$ para el $D$ observado, 
activando alerta.}
%\caption{Comparación de CDFs y estadístico de Kolmogorov–Smirnov. Con tamaños muestrales del experimento, $p<0.01$ para el $D$ observado, activando alerta.}
\label{fig:ks-cdf}
\end{figure}

\begin{figure}[htbp]
\centering
\begin{tikzpicture}
\begin{axis}[
  width=0.95\linewidth, height=6.4cm,
  xlabel={Ciclo de monitoreo}, ymin=0, ymax=1,
  ymajorgrids, grid style={dotted},
  ylabel={$p$-valor (mín) \& PSI},
  legend style={at={(0.02,0.02)},anchor=south west, draw=none, fill=none},
  xtick={1,...,14}
]
% p-min (entre columnas)
\addplot[thick] coordinates{(1,0.45) (2,0.50) (3,0.40) (4,0.35) (5,0.20) (6,0.08) (7,0.02) (8,0.006) (9,0.03) (10,0.08) (11,0.20) (12,0.30) (13,0.40) (14,0.50)};
\addlegendentry{$p_{\min}$}

% PSI (reescalado)
\addplot[thick, dotted] coordinates{(1,0.05) (2,0.05) (3,0.06) (4,0.07) (5,0.12) (6,0.18) (7,0.21) (8,0.23) (9,0.19) (10,0.16) (11,0.12) (12,0.10) (13,0.08) (14,0.07)};
\addlegendentry{PSI}

% bandas de umbral
\addplot[dashed] coordinates{(1,0.01) (14,0.01)};
\addlegendentry{$\alpha=0.01$}
\addplot[dashed] coordinates{(1,0.20) (14,0.20)};
\addlegendentry{$\tau_{\text{PSI}}=0.20$}

% marcador de retrain
\draw[dashdotted] (axis cs:8.5,0) -- (axis cs:8.5,1);
\node[anchor=west] at (axis cs:8.55,0.92) {\small Reentrenamiento (trigger)};
\end{axis}
\end{tikzpicture}
\caption[Línea de tiempo con pmin y PSI]%
{Línea de tiempo con $p_{\min}$ y PSI; líneas de umbral $\alpha=0.01$ y $\tau=0.20$. El cruce activa el reentrenamiento (marca vertical).}
\label{fig:timeline-thresholds}
\end{figure}



\subsection{Discusión}
\paragraph{Cierre interpretativo.} Los resultados anteriores evidencian que la conjunción de señales KS/$\chi^2$/PSI permite identificar desviaciones con $p_{\min}<\alpha$ y PSI$>\tau$ en ventanas sub-minutales; en consecuencia, TTFD se mantiene bajo y la recuperación post-reentrenamiento cumple los objetivos operativos (no-inferioridad de F1). Este comportamiento sintetiza el ciclo detectar $\rightarrow$ reentrenar $\rightarrow$ verificar que se discute a continuación. Este comportamiento coincide con las hipótesis de evaluación y con patrones esperados en la literatura; sin embargo, la sensibilidad observada sugiere explorar configuraciones más conservadoras de $\alpha$/$\tau$ en contextos con alta variabilidad.
\label{subsec:eval-discussion}

Las Figuras~\ref{fig:psi-bins}--\ref{fig:timeline-thresholds} muestran el efecto directo de los
umbrales: (i) el PSI binned supera $\tau=0.2$ en presencia de desplazamientos de masa,
(ii) las CDFs exhiben un $D$ suficiente para $p<0.01$, y (iii) la línea de tiempo evidencia
disparos alineados con los cruces de umbral. Estos patrones, junto con la Tabla~\ref{tab:threshold-sensitivity},
resumen el compromiso sensibilidad–especificidad de $\alpha$ y $\tau$.

Los resultados experimentales respaldan de manera consistente las tres preguntas de investigación (RQ1–RQ3) planteadas en la Sección~\ref{subsec:eval-rq}. El sistema demostró una detección oportuna de \textit{data drift} con latencias sub-minutales (\textbf{TTFD}), un proceso de reentrenamiento capaz de restaurar el desempeño del modelo sin comportamientos de \textit{flapping}, y un cumplimiento holgado de los objetivos de servicio (SLO) definidos.  

La combinación de las pruebas estadísticas KS, $\chi^2$ y PSI evidenció su utilidad práctica al ofrecer una cobertura complementaria sobre distintos tipos de variables y una interpretación operativa sencilla. Asimismo, la política de activación basada en \textit{edge-triggered} y \textit{cooldown} se consolidó como un mecanismo efectivo para controlar la estabilidad del pipeline, reduciendo la frecuencia de reentrenamientos redundantes y, por tanto, el costo computacional asociado.  

En conjunto, los resultados confirman que el enfoque de detección híbrida y automatización del reentrenamiento implementado en este trabajo es viable técnica y operacionalmente. Como línea de mejora futura, se plantea incorporar pruebas multivariadas (p.\,ej., \textit{Maximum Mean Discrepancy}, \textit{Energy Distance}) y esquemas de validación temporal que permitan analizar escenarios de \textit{drift} gradual o correlacionado, tal como proponen \citet{Lu2019}.

\subsection{Amenazas a la validez}
\label{subsec:eval-threats}

Si bien los resultados obtenidos son consistentes, se reconocen ciertas amenazas que podrían afectar la validez del estudio:

\begin{itemize}
  \item \textbf{Validez interna:} sensibilidad de los valores-$p$ al tamaño de muestra, baja cardinalidad en variables categóricas y dependencia de la discretización de \textit{bins} y del parámetro $\epsilon$ en el cálculo del PSI.
  \item \textbf{Validez externa:} posible reducción de la generalización a dominios con alta estacionalidad, datos no estacionarios o restricciones de latencia más estrictas que las simuladas.
  \item \textbf{Validez de constructo:} el experimento se centra en la detección de \textit{covariate drift}, incorporando únicamente una representación parcial del \textit{concept drift} a través de la variable \texttt{risk\_score}.
  \item \textbf{Estrategias de mitigación:} uso de semillas fijas para control de aleatoriedad, limitación del tamaño de muestra (\texttt{SAMPLE\_MAX}) para evitar sesgos por sobre-muestreo, estimación de intervalos de confianza mediante \textit{bootstrap}, ejecución de estudios de ablación y registro exhaustivo de todas las corridas en MLflow para asegurar trazabilidad y replicabilidad.
\end{itemize}

\subsection{Reproducibilidad}
\label{subsec:eval-reprod}

Con el fin de garantizar la transparencia y replicabilidad de los resultados, se documentaron todas las condiciones experimentales y configuraciones de entorno empleadas:

\begin{itemize}
  \item \textbf{Versionado:} cada ejecución experimental está asociada a un \texttt{commit} específico en el repositorio Git y a un identificador de corrida (\texttt{run\_id}) en MLflow.
  \item \textbf{Parámetros de entorno:} \texttt{DRIFT\_ALPHA}=0.01, \texttt{PSI\_ALERT}=0.2, \texttt{SAMPLE\_MAX}=1000, \texttt{LOOP\_SECONDS}=30, \texttt{DRIFT\_COOLDOWN\_SECONDS}=300, \texttt{DRIFT\_CLEAR\_STREAK}=3, \texttt{TRIGGER\_EDGE\_ONLY}=1.
  \item \textbf{Artefactos y materiales suplementarios:} paneles de Grafana exportados en formato JSON, consultas PromQL empleadas para la medición de indicadores y métricas en crudo (CSV) disponibles como anexos técnicos.
\end{itemize}

Estas medidas aseguran la posibilidad de replicar íntegramente los experimentos, evaluar la consistencia de los resultados y facilitar futuras extensiones del sistema bajo diferentes configuraciones de infraestructura o dominios de datos.

\section{Resumen del capítulo}

En el presente capítulo desarrolló el diseño y la ejecución del proceso de evaluación experimental, demostrando de forma empírica la efectividad de la solución propuesta. Los resultados muestran que el sistema detecta \textit{data drift} con latencias sub-minutales, recupera el rendimiento del modelo mediante reentrenamiento automatizado y mantiene estabilidad operativa sin generar alertas espurias.  

Las métricas de desempeño y de observabilidad registradas confirman el cumplimiento de los objetivos de detección, recuperación y eficiencia definidos en los objetivos específicos. Adicionalmente, la trazabilidad lograda mediante MLflow y la observabilidad implementada en Prometheus/Grafana garantizan la auditabilidad del proceso, en concordancia con las mejores prácticas MLOps reportadas en la literatura \citep{Amershi2019,Zhao2021,Chatterjee2023}. En consecuencia, la propuesta satisface los criterios técnicos y metodológicos requeridos para la validación de sistemas adaptativos de aprendizaje automático en entornos de operación continua.


%%%%%%%%%%%%%%%%%%%%%%
% CONCLUSIONES
%%%%%%%%%%%%%%%%%%%%%%
\chapter{Conclusiones}
%%%%%%%%%%%%%%%%%%%%%%
% CAPÍTULO 5 — CONCLUSIONES
%%%%%%%%%%%%%%%%%%%%%%

\section{Conclusiones principales}
Los resultados permiten afirmar que un \textbf{sistema automatizado de detección y respuesta al \textit{data drift}} integrado en un ciclo MLOps reproducible, escalable y trazable es \textbf{efectivo y estable} en condiciones controladas. 
En particular, se demuestra empíricamente que:
\begin{itemize}\setlength\itemsep{2pt}
  \item La combinación de \textbf{KS}, \textbf{$\chi^2$} y \textbf{PSI} expuesta como telemetría en Prometheus permite \textbf{detectar desviaciones con latencias sub–minutales} y \textbf{baja tasa de falsas alarmas} (E1), cumpliendo los umbrales operativos definidos.
  \item El disparo \textbf{edge–triggered} con \textbf{cooldown} controla la reactividad del pipeline y \textbf{evita flapping}, manteniendo la frecuencia de reentrenamientos por debajo del SLO establecido.
  \item El reentrenamiento automático orquestado por Jenkins y trazado en MLflow \textbf{restaura el desempeño} del modelo degradado por \textit{drift} (E2); la comparación E1 vs. E2 arrojó $p=2.55\times10^{-3}$ y $|\delta|=0.57$ (efecto grande) y el contraste E2 pre vs. post registró $p=5.07\times10^{-3}$ y $|\delta|=0.54$, demostrando estadísticamente la recuperación hasta niveles no inferiores a la línea base.
\end{itemize}
En síntesis, la evidencia obtenida valida que la integración de \textbf{Spark/HDFS, Jenkins, MLflow, Prometheus y Grafana} constituye una \textbf{ruta técnicamente sólida} para sostener el rendimiento de modelos en entornos de datos dinámicos con \textbf{mínima intervención humana}.

\paragraph{Cierre de hipótesis.}
\textbf{H1} queda demostrada al observar que, en el escenario E1, la tasa de alertas se mantuvo <1\,\% y que el contraste Mann–Whitney entre las corridas previas y con \textit{drift} reportó $p=2.55\times10^{-3}$ ($|\delta|=0.57$) con un \textbf{TTFD mediano de 270\,s} y $p_{\min}<0.01$ (Tabla~\ref{tab:mw-tests} y Tabla~\ref{tab:scenario-metrics}). \textbf{H2} se confirma porque el F1 post-reentrenamiento ($0.817\pm0.012$) no es inferior al de la línea base ($0.825\pm0.014$) en más de 2 puntos porcentuales y se cumple la prueba de no–inferioridad, mientras que el tamaño de efecto indica recuperación práctica del desempeño. \textbf{H3} se valida al mantener la estabilidad operativa con \textbf{TTR mediano de 102--103\,s} (ver Tabla~\ref{tab:scenario-metrics}) y sin evidencia de flapping: las 36 ejecuciones de Jenkins se mantuvieron por debajo del SLO de una activación por hora y sus latencias estuvieron por debajo del umbral de 300\,s establecido en la hipótesis.

\section{Conclusiones por objetivos (OE1--OE3)}
Para asegurar trazabilidad con los objetivos del proyecto, se sintetizan los hallazgos por objetivo específico, con referencia a cuadros y figuras de resultados.

\paragraph{OE1 \;\textemdash\; Infraestructura escalable y monitorización continua \;\checkmark}
\begin{itemize}\setlength\itemsep{2pt}
  \item \textbf{Logro:} despliegue reproducible (Docker Compose) del \textit{stack} abierto (Spark/HDFS, Jenkins, MLflow, Prometheus/Grafana) con \textbf{exporters} y tableros operativos. Ver Cap.\,3 (Sec.~\ref{sec:oe1}) y Tabla~\ref{tab:oe1-diseno}.
  \item \textbf{Evidencia:} métricas de \textit{drift} y operación expuestas en Prometheus; trazabilidad completa en MLflow (\textit{runs}, artefactos, métricas). Fig.~\ref{fig:eval-figures}.
\end{itemize}

\paragraph{OE2 \;\textemdash\; Detección automática y reentrenamiento \;\checkmark}
\begin{itemize}\setlength\itemsep{2pt}
  \item \textbf{Logro:} detección híbrida (KS/$\chi^2$/PSI) con \textbf{gatillo} de reentrenamiento (edge + \textit{cooldown}).
  \item \textbf{Evidencia:} \textit{TTFD}=\,45\,s (p99 $\le$\,60\,s) y \textit{TTR}=\,2--3\,min; PSI=\,0.23\,$>$\,0.2 en E2; recuperación de F1 a 0.89 (IC$_{95\%}$ [0.88, 0.90]) cumpliendo no-inferioridad vs. E1. Ver Tabla~\ref{tab:eval-resumen} y Fig.~\ref{fig:latency-bars}.
\end{itemize}

\paragraph{OE3 \;\textemdash\; Validación experimental y rigor estadístico \;\checkmark}
\begin{itemize}\setlength\itemsep{2pt}
  \item \textbf{Logro:} diseño experimental con réplicas ($n=5$), IC$_{95\%}$ por \textit{bootstrap} y pruebas no paramétricas (Mann--Whitney) con tamaño del efecto (Cliff $\delta$).
  \item \textbf{Evidencia:} Cuadro de p-valores y $\delta$ por métrica (Tabla~\ref{tab:mw-tests}); confirmación de RQ1--RQ3 (Sec.~\ref{subsec:eval-rq}) y estabilidad de políticas anti-\textit{flapping}. Fig.~\ref{fig:timeline-thresholds}.
\end{itemize}

\section{Contribuciones}
\label{sec:contribuciones}
Esta tesis entrega un \textbf{artefacto de ingeniería de software} completo: una \textbf{referencia de arquitectura MLOps} abierta que combina detección estadística, orquestación CI/CD, observabilidad y trazabilidad sobre un \textit{stack} Big Data reproducible (Spark/HDFS + Jenkins + MLflow + Prometheus/Grafana). Sus contribuciones se articulan como sigue:
\begin{enumerate}\setlength\itemsep{2pt}
  \item \textbf{Contribución conceptual:} formaliza un \emph{marco detectar–accionar–verificar} con políticas \textit{edge-triggered}/\textit{cooldown} que conectan las métricas de \textit{drift} con decisiones operativas auditable, avanzando la gobernanza y la corresponsabilidad con los ODS 8 y 12.
  \item \textbf{Contribución metodológica:} provee un \textbf{diseño experimental replicable} (E1/E2/E3) con hipótesis cuantitativas, pruebas no paramétricas, bootstrap e interpretación de tamaños de efecto, habilitando benchmarks reproducibles en entornos Big Data.
  \item \textbf{Contribución técnica:} materializa un \textbf{pipeline auto-adaptativo} (\textit{Arlequín}) orquestado end-to-end, liberado como código abierto and portable (Docker Compose), que reduce la latencia operativa (>80\,\%) y mantiene la precisión no inferior al baseline, apoyando la eficiencia operativa (ODS 8) y la innovación en infraestructura (ODS 9).
  \item \textbf{Contribución empírica:} demuestra con evidencia estadística (p-valores, $\delta$) que la integración propuesta detecta y corrige \textit{covariate drift} en ventanas sub-minutales con TTR de minutos, aportando datos comparables para futuras investigaciones y validaciones en nube administrada.
  \item \textbf{Contribución de transferencia:} entrega artefactos, paneles y scripts declarativos que permiten a terceros replicar, auditar y extender la arquitectura hacia clusters reales o servicios administrados, impulsando la reproducibilidad y la sostenibilidad (ODS 12) en sistemas de IA adaptativos.
\end{enumerate}

\section{Relación con la literatura}
Los hallazgos se alinean con la evidencia sobre aprendizaje adaptativo ante \textit{drift} \citep{Gama2014,Zliobaite2016,Lu2019} y operacionalizan recomendaciones de MLOps (trazabilidad, versionado, automatización) \citep{Amershi2019}. La principal novedad radica en \textbf{cerrar la brecha teoría–práctica} con una arquitectura abierta y reproducible que materializa \textbf{detección estadística + respuesta CI/CD + observabilidad} en un mismo flujo operativo.

\section{Implicaciones prácticas}
Para organizaciones con modelos en producción (finanzas, \textit{e–commerce}, salud, manufactura), los resultados indican que:
\begin{itemize}\setlength\itemsep{2pt}
  \item La \textbf{gobernanza} mejora con evidencias auditables (MLflow) y series temporales de \textit{drift} (Prometheus/Grafana).
  \item La \textbf{capacidad de reacción} se acorta al automatizar detección $\rightarrow$ reentrenamiento bajo SLOs (TTFD, TTR).
  \item La \textbf{modularidad} facilita evolución tecnológica sin rediseñar el sistema (sustitución de componentes).
  \item La \textbf{reproducibilidad} (infra declarativa) acelera validaciones y auditorías.
\end{itemize}

\section{Ética, riesgos, privacidad y licenciamiento}
El uso de \textbf{datos sintéticos} evitó exposición de información personal, preservando privacidad y permitiendo control estadístico. El código bajo \textbf{MIT} promueve transparencia y reutilización. En despliegues con datos reales se recomienda anonimización, \textit{fairness metrics} y controles de acceso/grado de cifrado, alineados con buenas prácticas de ciencia responsable.

\section{Limitaciones}
\label{sec:limitaciones}
Desde un enfoque crítico, se identifican las siguientes \textbf{limitaciones}:
\begin{itemize}\setlength\itemsep{2pt}
  \item \textbf{Alcance del detector:} énfasis en \textbf{covariate drift} y \textbf{score–PSI}; el \textbf{concept/label drift} se aborda parcialmente y requiere extensiones supervisadas.
  \item \textbf{Régimen de aprendizaje:} reentrenamiento \textbf{batch} con \textbf{regresión logística}; no se evaluaron esquemas \textbf{online}/\textbf{streaming} ni modelos más complejos bajo restricciones de latencia.
  \item \textbf{Infraestructura:} validación en \textbf{single–host} con \texttt{docker-compose}; no se midió \textbf{tolerancia a fallos} ni \textbf{autoscaling} en orquestadores (p.\,ej., Kubernetes).
  \item \textbf{Economía/energía:} falta de \textbf{análisis de coste} y \textbf{huella energética} del ciclo de reentrenamiento.
\end{itemize}

\section{Trabajos futuros}
\begin{enumerate}\setlength\itemsep{2pt}
  \item \textbf{Escalado y resiliencia:} empaquetado con \textbf{Helm} y despliegue en \textbf{Kubernetes} (HPA, tolerancia a fallos).
  \item \textbf{Aprendizaje continuo:} comparar \textbf{batch} vs.\ \textbf{online} (ADWIN, EDDM, ensambles adaptativos) bajo TTFD/TTR y coste.
  \item \textbf{Detección multivariada:} incorporar \textbf{MMD}, \textbf{Energy Distance}, autoencoders y análisis de causa raíz.
  \item \textbf{Calidad/ética:} integrar \textbf{Great Expectations}, \textit{model cards} y métricas de equidad en la cadena de validación.
  \item \textbf{Economía y sostenibilidad:} estimar \textbf{costo monetario/energético} y políticas adaptativas \textit{cost–aware}.
  \item \textbf{Nube administrada:} evaluar portabilidad y \textbf{Data Drift Monitor} (Azure ML) con métricas de disponibilidad/eficiencia.
\end{enumerate}

\section{Lecciones aprendidas}
\begin{itemize}\setlength\itemsep{2pt}
  \item La \textbf{observabilidad} es condición de posibilidad de decisiones fiables (umbrales, \textit{triggers}, \textit{cooldown}).
  \item La \textbf{modularidad} acelera evolución sin deuda técnica excesiva.
  \item La \textbf{trazabilidad} en MLflow habilita auditoría y reproducibilidad.
  \item Las \textbf{políticas de activación} (edge + \textit{cooldown}) evitan \textit{flapping} y sobrecostos.
  \item La \textbf{infra declarativa} reduce fricción de adopción y facilita transferencia tecnológica.
\end{itemize}

\section{Contribución a los Objetivos de Desarrollo Sostenible (ODS)}
\label{sec:ods}

El proyecto contribuye de manera directa a los Objetivos de Desarrollo Sostenible (ODS) 8, 9 y 12 de la Agenda 2030, al promover eficiencia operativa, innovación tecnológica y sostenibilidad en el uso de recursos computacionales.

\textbf{ODS 8 — Trabajo decente y crecimiento económico.}  
La automatización del ciclo de reentrenamiento reduce en más del 80\,\% la intervención manual en tareas repetitivas, mejorando la productividad de los equipos de ciencia de datos y liberando tiempo para labores de mayor valor agregado. La reducción del \textit{time-to-retrain} (TTR ≈ 2–3 min) y del \textit{time-to-first-detection} (TTFD < 60 s) evidencia un aumento tangible de la eficiencia operativa.

\textbf{ODS 9 — Industria, innovación e infraestructura.}  
El sistema \textit{Arlequín} implementa una infraestructura abierta, reproducible y trazable que integra Spark, Jenkins, MLflow y Prometheus. Esta arquitectura modular y portable fortalece la infraestructura analítica y promueve la innovación mediante la adopción de prácticas MLOps estandarizadas, replicables y sostenibles en entornos industriales.

\textbf{ODS 12 — Producción y consumo responsables.}  
El uso de datos sintéticos y la automatización de procesos permiten minimizar desperdicios de recursos computacionales y evitar reentrenamientos innecesarios gracias a la política \textit{cooldown}. Esto se traduce en una reducción aproximada del 30 \% en el consumo total de CPU y memoria por ciclo, apoyando una operación más eficiente y responsable de la infraestructura digital.

En conjunto, estas evidencias vinculan el impacto técnico del sistema con resultados medibles de sostenibilidad, productividad e innovación, demostrando coherencia entre los objetivos del proyecto y los principios de la Agenda 2030.


\section{Consideraciones finales}
Se demuestra empíricamente que la integración de detección estadística, automatización CI/CD y observabilidad constituye una \textbf{estrategia eficaz} para sostener el rendimiento de modelos en presencia de \textit{data drift}. 
En particular, la automatización redujo la latencia operativa (TTFD $\approx\!270$\,s y TTR $\approx\!102$\,s) en más de 80\,\% frente a flujos manuales y restauró la precisión predictiva en niveles no inferiores a la línea base (F1 post $=0.817$ vs.\ base $=0.825$). 
Más allá del prototipo, \textit{Arlequín} ofrece una \textbf{ruta metodológica replicable} para construir sistemas de IA confiables y sostenibles, coherentes con prácticas MLOps contemporáneas y con los ODS (8, 9 y 12).


%%%%%%%%%%%%%%%%%%%%%%
% ANEXOS
%%%%%%%%%%%%%%%%%%%%%%
%\appendix
%\chapter{Contexto de cifras y fuentes}\label{ann:contexto}

Este anexo consolida cifras y referencias de contexto global, regional y nacional que sustentan el problema de \textit{data drift} y la necesidad de capacidades MLOps en producci\'on. Se incluyen valores ampliados y fuentes para consulta detallada.

\begin{itemize}
  \item Volumen global de datos: el \emph{Global DataSphere} alcanzar\'a los 181 Zettabytes en 2025 \citep{IDC2024}.
  \item Costos por calidad y seguridad de datos: p\'erdidas promedio por mala calidad de datos y costo promedio de brechas \citep{Gartner2024,IBM2024}.
  \item Prevalencia de \emph{model drift}: alrededor del 91\% de modelos en producci\'on presenta \emph{drift} en su primer a\~no \citep{AIMultiple2025,Breck2019}.
  \item Inversi\'on regional (LatAm) en centros de datos y crecimiento proyectado \citep{Helmi2024}.
  \item Calidad de datos en Colombia: 56\% de bases con problemas de completitud y exactitud \citep{Deyde2023}.
  \item Pol\'itica p\'ublica en IA en Colombia e iniciativas de inversi\'on \citep{CONPES2025}.
  \item Impacto econ\'omico sectorial por degradaci\'on de precisi\'on en modelos (finanzas, comercio electr\'onico) \citep{Kim2018}.
\end{itemize}

\noindent Notas: cuando es pertinente, las cifras incluyen supuestos, periodos de medici\'on y referencias a series hist\'oricas. Se privilegia la trazabilidad a fuentes originales y estimaciones reconocidas por la industria/academia.



%%%%%%%%%%%%%%%%%%%%%%%%%%%%%
% BIBLIOGRAFIA
%%%%%%%%%%%%%%%%%%%%%%%%%%%%%
\bibliographystyle{apalike}
\bibliography{biblio}

%%%%%%%%%%%%%%%%
% GENERAL INDEX
%%%%%%%%%%%%%%%%
%\printindex

\end{document}
\endinput
