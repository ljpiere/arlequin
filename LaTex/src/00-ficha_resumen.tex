\thispagestyle{empty}
\begin{center}
    \Large{Ficha Resumen \\ Trabajo de Grado Maestría en Ingeniería de Software}
\end{center}

\textbf{TíTULO:} {Implementación de un Sistema Automatizado de Reentrenamiento mediante la Detección de Data Drift en un Ciclo MLOps para Entornos Big Data}
\begin{enumerate}
    \item Énfasis: Ingeniería de Software
    \item Área de trabajo: Ciencias de la Computación y la Ingeniería de Sistemas, con un enfoque específico en Big Data, Machine Learning y MLOps 
    \item Tipo de proyecto: Aplicado
    \item Estudiante: Leyton Jean Piere Castro Clavijo
    \item Correo electrónico: leytoncastro@javerianacali.edu.co
    \item Dirección y teléfono: Cr. 9A \#43-38 Ibagué \- Tolima, 318 785 7029
    \item Director: Felipe Buitrago Carmona
    \item Vinculación del director: Externo
    \item Correo electrónico del director: felipe.buitrago@ucaldas.edu.co
    \item Co-Director (Si aplica):
    \item Grupo o empresa que lo avala (Si aplica):
    \item Otros grupos o empresas: Ninguno
    \item Palabras clave: Data Drift, MLOps, Big Data, Automatización, Machine Learning
    \item ODS que aplica al proyecto (Agenda 2030): ODS 8, ODS 9, ODS 12.
    \item Fecha de inicio: 23 de Junio de 2025
    \item Resumen:  El presente proyecto propone un sistema automatizado de reentrenamiento de modelos de \textit{machine learning} que responde a la detección temprana del \textit{data drift} en entornos Big Data. En la actualidad, las organizaciones enfrentan pérdidas promedio de 12,9 millones de USD anuales por mala calidad de datos y 4,88 millones por brechas de información \citep{Gartner2024,IBM2024}, lo que evidencia la necesidad de soluciones que garanticen precisión y resiliencia en sistemas basados en datos. El sistema desarrollado monitorea continuamente los flujos de entrada de los modelos en producción; cuando identifica cambios estadísticamente significativos en su distribución, activa un \textit{pipeline} automatizado que reentrena, valida y despliega una nueva versión del modelo. De esta forma, se obtiene un modelo actualizado, trazable y auditable, que mantiene la exactitud predictiva ante condiciones cambiantes. Basado en prácticas MLOps, el ciclo integra herramientas abiertas como Docker para contenerización, Jenkins para orquestación, Spark para procesamiento distribuido y MLflow para trazabilidad experimental. Los resultados, validados en escenarios con y sin \textit{drift}, demuestran una recuperación completa del desempeño del modelo (F1-score) y una reducción significativa del tiempo de respuesta operativa. El proyecto contribuye a los \textbf{ODS 8, 9 y 12}: fomenta la eficiencia y productividad mediante automatización inteligente (ODS 8), impulsa innovación e infraestructura digital reproducible (ODS 9) y promueve la sostenibilidad operativa al reducir retrabajos y desperdicio computacional (ODS 12).
\end{enumerate}
