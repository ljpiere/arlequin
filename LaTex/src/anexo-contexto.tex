\chapter{Contexto de cifras y fuentes}\label{ann:contexto}

Este anexo consolida cifras y referencias de contexto global, regional y nacional que sustentan el problema de \textit{data drift} y la necesidad de capacidades MLOps en producci\'on. Se incluyen valores ampliados y fuentes para consulta detallada.

\begin{itemize}
  \item Volumen global de datos: el \emph{Global DataSphere} alcanzar\'a los 181 Zettabytes en 2025 \citep{IDC2024}.
  \item Costos por calidad y seguridad de datos: p\'erdidas promedio por mala calidad de datos y costo promedio de brechas \citep{Gartner2024,IBM2024}.
  \item Prevalencia de \emph{model drift}: alrededor del 91\% de modelos en producci\'on presenta \emph{drift} en su primer a\~no \citep{AIMultiple2025,Breck2019}.
  \item Inversi\'on regional (LatAm) en centros de datos y crecimiento proyectado \citep{Helmi2024}.
  \item Calidad de datos en Colombia: 56\% de bases con problemas de completitud y exactitud \citep{Deyde2023}.
  \item Pol\'itica p\'ublica en IA en Colombia e iniciativas de inversi\'on \citep{CONPES2025}.
  \item Impacto econ\'omico sectorial por degradaci\'on de precisi\'on en modelos (finanzas, comercio electr\'onico) \citep{Kim2018}.
\end{itemize}

\noindent Notas: cuando es pertinente, las cifras incluyen supuestos, periodos de medici\'on y referencias a series hist\'oricas. Se privilegia la trazabilidad a fuentes originales y estimaciones reconocidas por la industria/academia.

\section{Detalle te\'orico de detectores de \textit{data drift}}\label{ann:detectors}

Esta secci\'on complementa la discusi\'on de la Secci\'on~\ref{subsec:cuadro-detectores}, documentando los aspectos m\'as t\'ecnicos que fueron resumidos en el cuerpo principal.
\begin{itemize}
  \item \textbf{Cobertura de variables.} KS supone variables continuas y requiere ordenamiento; $\chi^2$ opera sobre tablas de contingencia y se vuelve inestable si las frecuencias esperadas son bajas; PSI admite num\'ericas y categ\'oricas siempre que el \textit{binning} mantenga al menos 5\% de soporte por caj\'on; KL exige discretizaci\'on o estimaci\'on de densidad para variables continuas.
  \item \textbf{Modo y latencia.} En ventanas \textit{batch} se trabaja con muestras independientes de referencia/actual; el retardo equivale a la longitud de la ventana. En flujo continuo, ADWIN y EDDM mantienen estructuras adaptativas (colas con p\'erdida exponencial y contadores de distancia entre errores) que permiten reaccionar en $\mathcal{O}(1)$--$\mathcal{O}(\log n)$ por evento.
  \item \textbf{Costo computacional.} PSI y $\chi^2$ dependen de conteos ($\mathcal{O}(k)$); KS implica ordenar o calcular CDFs emp\'iricas ($\mathcal{O}(n\log n)$); KL suma un costo adicional por suavizado/regularizaci\'on para evitar probabilidades nulas; ADWIN amortiza memoria proporcional a la cantidad de niveles de confianza.
  \item \textbf{Calibraci\'on y gobernanza.} PSI utiliza umbrales operativos ($<0.1$ estable, 0.1--0.25 bajo observaci\'on, $>0.25$ severo). KS y $\chi^2$ requieren ajustar $\alpha$ y controlar multiplicidad (Holm/BH) cuando se monitorean numerosas columnas. KL necesita $\epsilon$ de suavizado; los detectores \textit{online} exigen definir niveles de confianza y tama\~no m\'inimo de ventana adaptativa.
  \item \textbf{Extensi\'on multivariada.} Se recomienda combinar pruebas univariantes con control de tasa de descubrimiento falso (FDR) para mantener interpretabilidad. Para dependencias fuertes puede recurrirse a MMD, Energy Distance o a clasificadores meta que distingan entre referencia y corriente; dichos casos superan el alcance de esta tesis y se listan como trabajo futuro.
\end{itemize}
