\chapter{Contexto de cifras y fuentes}\label{ann:contexto}

Este anexo consolida cifras y referencias de contexto global, regional y nacional que sustentan el problema de \textit{data drift} y la necesidad de capacidades MLOps en producci\'on. Se incluyen valores ampliados y fuentes para consulta detallada.

\begin{itemize}
  \item Volumen global de datos: el \emph{Global DataSphere} alcanzar\'a los 181 Zettabytes en 2025 \citep{IDC2024}.
  \item Costos por calidad y seguridad de datos: p\'erdidas promedio por mala calidad de datos y costo promedio de brechas \citep{Gartner2024,IBM2024}.
  \item Prevalencia de \emph{model drift}: alrededor del 91\% de modelos en producci\'on presenta \emph{drift} en su primer a\~no \citep{AIMultiple2025,Breck2019}.
  \item Inversi\'on regional (LatAm) en centros de datos y crecimiento proyectado \citep{Helmi2024}.
  \item Calidad de datos en Colombia: 56\% de bases con problemas de completitud y exactitud \citep{Deyde2023}.
  \item Pol\'itica p\'ublica en IA en Colombia e iniciativas de inversi\'on \citep{CONPES2025}.
  \item Impacto econ\'omico sectorial por degradaci\'on de precisi\'on en modelos (finanzas, comercio electr\'onico) \citep{Kim2018}.
\end{itemize}

\noindent Notas: cuando es pertinente, las cifras incluyen supuestos, periodos de medici\'on y referencias a series hist\'oricas. Se privilegia la trazabilidad a fuentes originales y estimaciones reconocidas por la industria/academia.

